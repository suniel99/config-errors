
\begin{table*}[t]
\vspace{1mm}
\centering
\small{
\setlength{\tabcolsep}{.96\tabcolsep}
\begin{tabular}{|l||c|c|c|c|c|c|}
\hline
 Error ID. Program& Root Cause Configuration Option & \#Options & \multicolumn{3}{|c|}{Rank of the Root Cause Configuration Option}\\
 \cline{4-6}
 & & & \ourtool & \prevtool~\cite{Zhang:2013:ADS}  & \conftool~\cite{Rabkin:2011:PPC} \\
 %&  Options &  Options&  Options \\
 \hline
 \hline
 1. Randoop& \CodeIn{usethreads} & \randoopoptnum & \randooprank & \n & \x \\
 2. Weka & \CodeIn{m\_numFolds} & \wekaoptnum &  \wekarank & 9 & 1 \\
 3. Synoptic& \CodeIn{dumpInitialGraphDotFile} & \synopticoptnum & \synopticrankfirst & \n & \x \\
 4. Synoptic& \CodeIn{dumpInitialGraphPngFile} &\synopticoptnum & \synopticranksecond & \n & \x \\
 5. JChord& \CodeIn{chord.ssa} & \jchordoptnum & \jchordrankfirst & 3& \x \\
 6. JChord& \CodeIn{chord.print.methods} & \jchordoptnum & \jchordranksecond & \n & \x \\
 7. JMeter&\CodeIn{output\_format} & \jmeteroptnum & \jmeterrank & & \x \\
 8. Javalanche& \CodeIn{project.tests} &  \javalancheoptnum & \javalancherank& & \x \\
\hline
\hline
 Average &  & 49.1 & \averagerank & & 47.5\\
\hline
\end{tabular}
}
\vspace{-2mm}
\Caption{{\label{tab:errors} 
Experimental results. Column ``Root Cause Configuration Option''
shows the actual root cause configuration option.
Column ``\#Options'' shows the number
of configuration options in the new program version, taken from
Table~\ref{tab:experiment-sub}. 
Column ``Rank of the Root Cause Configuration Option'' shows the
absolute rank of the actual root
cause configuration option in each technique's output (lower is better).
%``\N'' means the technique does not identify the root casue configuration options.
``\x'' means the technique is not applicable (i.e., requiring a crashing
point), and ``\n'' means the technique does not identify the actual root cause.
When computing the average rank, each ``\x'' or ``\n'' is treated as
half of the number of configuration options, because a user would need to examine
on average half of the avaiable options to find the root cause.
Column ``\ourtool'' shows the results
of using our technique. Columns ``\prevtool'' and ``\conftool'' shows
the results of using two existing techniques, respectively.
}
}
\end{table*}
