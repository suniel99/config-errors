
\section{Diagnosis Technique}
\label{sec:technique}

In \ourtool, we model a configuration as a set of key-value
pairs, where the keys are strings and the values have
arbitrary type. As an example, in the error-fixing
configuration setting \CodeIn{output\_format = XML} for JMeter in Section~\ref{sec:evolerror},
\CodeIn{output\_format} is the configuration option name,
and \CodeIn{XML} is the value. This key-value abstraction
is used by the POSIX system environment, the Java
Properties API, and the Windows Registry.

\subsection{Overview}

Figure~\ref{fig:overview} sketches the high-level workflow of
\ourtool. To diagnose an error, \ourtool first asks the user to demonstrate
the different behaviors by executing the same input values and
configuration on two instrumented program versions (Section~\ref{sec:profiling}). 
After that, \ourtool analyzes the desired and undesired
execution profiles produced by user demonstration, and identifies
predicates that two executions behave differently (Section~\ref{sec:comparison}).
Then, \ourtool analyzes the identified execution differences
to reason about which configuration option may cause such differences,
and reports these to the user (Section~\ref{sec:rootcause}).

\subsection{Instrumentation and Profiling}
\label{sec:profiling}

\ourtool first instruments both the old and new versions
of the target program offline by inserting code to monitor
the execution of each predicate at runtime. A predicate
is a Boolean expression in a conditional or loop statement,
whose evaluation result determines whether to execute the
following statement or not. A predicate's run-time outcome
affects the program control flow. 

Then, \ourtool asks the user to demonstrate the different
behaviors on two instrumented versions. Demonstration is
one of the simplest way for a user to describe their problem;
it is easier than writing specifications or scripts of any form.

Executing the instrumented program produces an execution profile,
which consists of a sequence of executed predicates and their
evaluation results. Such captured predicate behaviors are by
no means complete in recording the full execution trace. However,
they capture control flows a program is taking. Thus, using the
recorded predicate execution result, \ourtool could faithfully
derive the full control flow. 

\subsection{Execution Trace Comparison}
\label{sec:comparison}


\begin{figure}[t]
\textbf{Axuiliary methods}:

sameStatement($\mathit{s}$, $\mathit{s'}$): return whether two statements
$\mathit{s}$ and $\mathit{s'}$ are the same statements. \todo{What does
  ``same statement'' mean?  The same source code?  The same subexpressions
  too? (For example, does this return true for if statement only if the
  predicate, then clause, and else clause are all identical?)}

BFS($\mathit{s}$, $\mathit{cfg}$): return an ordered list of reachable successive statements from statement $\mathit{s}$ in $\mathit{cfg}$ by Breath-First Search (BFS).

firstMatch($\mathit{stmtList_1}$, $\mathit{stmtList_2}$): return the first matched statement pair between $\mathit{stmtList_1}$ and $\mathit{stmtList_2}$ (iterating $\mathit{stmtList_1}$ first).
\todo{What is a ``matched statement pair''?}


\textbf{Input}: two methods from two software versions: $\mathit{m_{old}}$ and $m_{new}$,

\quad a maximum lookahead value $\mathit{lh}$ (Our experiment uses $\mathit{lh}=5$.).\\
\textbf{Output}: matched statements between old and new versions.
\vspace{-4mm}%
matchStatements($\mathit{m_{old}}$, $\mathit{m_{new}}$, $\mathit{lh}$)\\
\begin{algorithmic}[1]
\STATE $\mathit{matchedStmts}$ $\leftarrow$ new Map$\langle$Stmt, Stmt$\rangle$
\STATE $\mathit{cfg_{old}}$ $\leftarrow$ constructControlFlowGraph($\mathit{m_{old}}$)
\STATE $\mathit{cfg_{new}}$ $\leftarrow$ constructControlFlowGraph($\mathit{m_{new}}$)
\STATE $\mathit{stack}$ $\leftarrow$ new Stack$\langle$Pair$\langle$Stmt, Stmt$\rangle$$\rangle$
\STATE $\mathit{stack}$.push($\mathit{cfg_{old}}$.$\mathit{entry}$, $\mathit{cfg_{new}}$.$\mathit{entry}$)
\WHILE{$\mathit{stack}$ is not empty}
\STATE $\langle$$\mathit{stmt_{old}}$, $\mathit{stmt_{new}}$$\rangle$ $\leftarrow$ $\mathit{stack}$.pop()
%\IF{$\mathit{stmt_{old}}$ or $\mathit{stmt_{new}}$ has already been matched}
\IF{$\mathit{matchedStmts}$.keys().contains($\mathit{stmt_{old}}$) \\ \quad || $\mathit{matchedStmts}$.values().contains($\mathit{stmt_{new}}$)}
\STATE \textbf{continue}
\ENDIF
\IF{sameStatement($\mathit{stmt_{old}}$, $\mathit{stmt_{new}}$)}
\STATE $\mathit{matchedStmts}$[$\mathit{stmt_{old}}$] $\leftarrow$ $\mathit{stmt_{new}}$
\ELSE
\STATE $\mathit{stmtList_{old}}$ $\leftarrow$ BFS($\mathit{stmt_{old}}$, $\mathit{lh}$)
\STATE $\mathit{stmtList_{new}}$ $\leftarrow$ BFS($\mathit{stmt_{new}}$, $\mathit{lh}$)
\STATE $\langle$$\mathit{stmt_{old}}$, $\mathit{stmt_{new}}$$\rangle$ $\leftarrow$ firstMatch($\mathit{stmtList_{old}}$, $\mathit{stmtList_{new}}$)
\IF{$\langle$$\mathit{stmt_{old}}$, $\mathit{stmt_{new}}$$\rangle$ $\neq$ null}
\STATE $\mathit{stack}$.push($\langle$$\mathit{stmt_{old}}$, $\mathit{stmt_{new}}$$\rangle$)
\ENDIF
\ENDIF
\ENDWHILE
\RETURN $\mathit{matchedStmts}$
\end{algorithmic}
\vspace{-4mm}
\caption{Algorithm for matching statements from two methods.
\label{fig:matching}
\todo{This algorithm is suspicious to me.  For one thing, the size of the
  stack is always $\le 1$ --- why use a stack in that case?  For another
  thing, as soon as \emph{any} statement satisfies sameStatement(), then
  the algorithm terminates.  Is there a bug in the algorithm?}
}
\end{figure}


In this step, \ourtool compares two execution profiles
produced by the desired and undesired program behaviors, and
then summarizes the predicates that behave differently
across two versions.

\subsubsection{Matching Predicates}

For each predicate recorded in the desired execution profile,
\ourtool matches it in the new program version, and identified
the matched predicate in the undesired execution profile.

To match a predicate, \ourtool first matches its hosting method
by using two strategies in the following order:

\begin{enumerate}
\item \textbf{Similar method name}
\item \textbf{Similar method content} Use a JDiff-like algorithm
as shown in Figure~\ref{fig:matching}.
\end{enumerate}

\todo{more details needed above}

After determining the matched method, \ourtool identifies
the matched predicate by examing the similarity in the
instructions of its two outgoing branches.

\todo{above is the basic idea. need more elaboration}


\subsubsection{Summarizing Differences}

Using the matched predicate information, \ourtool further
summarizes the behaviorally-different predicates between
two executions. It summarizes differences in three aspects:

\begin{enumerate}
\item Predicates that only executed in the desired execution
\item Predicates that only executed in the undesired execution
\item Predicates that executed in two executions, but have
different behaviors. \todo{elaborate different behaviors in details}
\end{enumerate}

Differences in predicate behaviors indicate different executed paths
between two versions. Such differences provide evidence of
which part of the program might be behaving abnormally and why.

%is any path not taken by the actual
%program execution that starts at a conditional
%branch instruction for which the branch condition
%is affected by one or more configuration options.


\subsection{Root Cause Analysis}
\label{sec:rootcause}

%An alternate path is any path not taken by the actual
%program execution that starts at a conditional
%branch instruction for which the branch condition
%is affected by one or more configuration options.

In this step, \ourtool attribute execution differences
to one or more root cause configuration options.
The key idea is to identify configuration options that
may affect the behaviorally-deviated predicates, and rank
such options by the degree of how they may cause
the execution differences.

\subsubsection{Identifying Affecting Configuration Options}

\ourtool uses thin slicing to identify the affecting
configuration options for each predicate.

\todo{illustrate why thin slicing is a good choice, and
give an example.}


\subsubsection{Finding Root Causes}

Essentially, this step answers the question:
``how likely is a configuration
option cause the execution differences?''.
%\ourtool uses two heuristics: xxx.
%These heuristics cause real root causes to rank
%higher than false positives.


%\ourtool next determines why each different path
%gets executed. \ourtool associates each block
%with a set of root causes, More specifically,
%it uses thin slicing to identify a
%set of configuration options for each deviated
%execution path as the root causes.


To distinguish the likelihood of each configuration
option being the root cause, \ourtool further analyzes
the cascading effects of each configuration option.
Specifically, \ourtool propagates the effect of
a configuration option to other program statements
according to the dependencies introduced via data and control flow.
The more behaviorally-different predicate a
configuration option can affect, the more likely it is
the root cause.

To do so, \ourtool associates each configuration option
with a floating-point weight as its effects propagates
in the program.  The weight represents the strength of
the causual relationship between the configuration option
and the execution differences.

The detailed algorithm is shown in Figure~\ref{fig:diagalg}.
\todo{need explanation}

\ourtool uses the number of executed instructions
as a metric to characterize the effect.

%Two configuration options are considered equal
%root causes even if one has a direct causal
%relationship to a location (e.g., the value
%in memory was read directly from the configuration value)
%and another has a nebulous relationship (e.g., 
%its effects is propagated along a long chain
%of conditional assignments).

%Data flow dependencies are treated to be more likely
%to lead to the root cause than control flow
%dependencies. Control flow dependencies are assumed
%to be more likely to the root cause if they occur
%later in the execution (i.e., closer to the
%deviated execution paths).

%Assign control flow dependence only half the weight
%of the weight introduced by data flow dependencies.
%Further, each nested conditional branches reduces
%the weight by prior branch in the nest one half.

%what about 
%if(a) \{
%
%    if(b) \{
%        //should a and b equally important
%     \}
%    \}



\begin{figure}[t]
\textbf{Auxiliary methods:}

\quad getAffectingOptions($\mathit{p}$, $\mathit{V}$): return all configuration options that affect predicate $\mathit{p}$ in the software version $\mathit{V}$

\quad getExecutedStmtNum($\mathit{p}$, $\mathit{V}$, $\mathit{T}$): return the number of executed statements (determined by predicate $\mathit{p}$) in trace $\mathit{T}$

\textbf{Input}: two software version: $\mathit{V_{old}}$ and $\mathit{V_{new}}$, 

\quad the map of all behaviorally-deviated predicates: $\mathit{predMap}$, produced by Figure~\ref{fig:identify}.

\textbf{Output}: {a ranked list of likely root cause configuration options}
\vspace{-4mm}%
recommendOptions($\mathit{V_{old}}$, $\mathit{V_{new}}$, $\mathit{predMap}$)\\
\begin{algorithmic}[1]
\STATE $\mathit{optionMap}$ $\leftarrow$ new Map$\langle$Option, Float$\rangle$
\FOR{each $\langle$$\mathit{p_{old}}$, $\mathit{p_{new}}$$\rangle$ in $\mathit{predMap}$.keys()}
\STATE $\mathit{v}$ $\leftarrow$ $\mathit{predMap}$[$\langle$$\mathit{p_{old}}$, $\mathit{p_{new}}$$\rangle$]
\IF{$\mathit{p_{old}}$ $\neq$ $\mathit{null}$}
\STATE $\mathit{options_{old}}$ $\leftarrow$ getAffectingOptions($\mathit{p_{old}}$, $\mathit{V_{old}}$)
\STATE $\mathit{v}$ $\leftarrow$ $\mathit{v}$ $\times$ getExecutedStmtNum($\mathit{p_{old}}$, $\mathit{V_{old}}$, $\mathit{T_{old}}$)
\FOR{each Option $\mathit{option}$ in $\mathit{options_{old}}$}
\STATE $\mathit{optionMap}$[$\mathit{option}$] = $\mathit{optionMap}$[$\mathit{option}$] + $\mathit{v}$
\ENDFOR
\ENDIF
\IF{$\mathit{p_{new}}$ $\neq$ $\mathit{null}$}
\STATE $\mathit{options_{new}}$ $\leftarrow$ getAffectingOptions($\mathit{p_{new}}$, $\mathit{V_{new}}$)
\STATE $\mathit{v}$ $\leftarrow$ $\mathit{v}$ $\times$ getExecutedStmtNum($\mathit{p_{new}}$, $\mathit{V_{new}}$, $\mathit{T_{new}}$)
\FOR{each Option $\mathit{option}$ in $\mathit{options_{new}}$}
\STATE $\mathit{optionMap}$[$\mathit{option}$] = $\mathit{optionMap}$[$\mathit{option}$] + $\mathit{v}$
\ENDFOR
\ENDIF
\ENDFOR
\RETURN $\mathit{optionMap}$.sortedKeys()
\vspace{-2mm}
\end{algorithmic}
\caption{Algorithm for recommending configuration options.
\label{fig:recommend}
}
\end{figure}


\todo{mention implementation details about recursive, avoid double count}


%\ourtool also tracks implicit control flow dependencies.

A predicate's execution depends on the value of
the configuration option, and the associated weight
indicates the strength of the dependency.

%\ourtool assumes that control flow dependencies are more likely
%to lead to the root cause if they
%occur closer to the predicate being executed.

%This represents the belief that the execution of
%the basic block is affected by XXX.
%Since these are two independent probabilities:
%potentially changing either of the two options
%might cause the basic block to not have been executed.
%Thus, the weights of configuration options associated
%with a basic block need not sum to one.

\ourtool determines the root cause of each behaviorally-different
predicate.  It attributes the cost of the difference 
to the associated configuration options.
It calculates a cost for the difference by first summing
the total number of instructions executed via this
predicate. 

%It attributes the divergence
%to root cause configuration options by multiplying
%the cost of the divergence by the weights of the configuration
%options that are relevant to the divergence.
%the 

\todo{repeatedly applying thin slicing is different than full slicing.}


\subsection{Discussion}

We next discuss some design issues in \ourtool.

\vspace{1mm}

\noindent \textbf{\textit{Fixing configuration errors vs. Fixing software bugs.}}
It appears that software configuration lies in the gray zone between
the software developers of software users.
The responsibility for creating correct configurations
lies with both parties; the developer should create
intuitive configuration logic, build logic that detects
errors, and convey configuration knowledge to users
effectively; \todo{evolve} the end-user should imbibe the
knowledge and manage cross-version configurations.
This shared responsibility is non-trivial to efficiently
achieve. For example, there is no obviously ``correct'' way to
build configuration logic; also, unlike fixing a bug once,
every software user has to be informed on the right way to
configure the system. Perhaps as a result, misconfigurations
have been one of the dominant causes of system issues and
is likely to continue so.

\todo{a few more tradeoffs can be discussed here.}

\vspace{1mm}
\noindent \textbf{\textit{\ourtool's current limitations.}}
There are three major limitations in the \ourtool technique.
First, \ourtool currently assumes that only one
configuration option has an incorrect value.
If fixing a particular configuration error
requires changing values of two configuration options,
then \ourtool may not identify both of them.
Second, \ourtool assumes the different behaviors
on two program versions can be demonstrated by users,
and does not support diagnosing non-deterministic
errors. For non-deterministic errors, \ourtool
could potentially leverage a deterministic replay
system~\cite{Huang:2013:CRL, Jin:2012:BRF} to faithfully reproduce the behaviors.
Third, \ourtool focuses on identifying root cause
configuration options that can change behaviors of
the analyzed program rather than the underlying OS
or runtime system. Configuration options, such
as \CodeIn{-Xmx} used in launch a Java program for
specifying JVM's heap size, are not supported
in \ourtool.
