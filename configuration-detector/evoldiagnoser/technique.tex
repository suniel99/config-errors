
\section{Technique}
\label{sec:technique}

\ourtool models a configuration as a set of key-value
pairs, where the keys are strings and the values have
arbitrary type. 
This key-value abstraction
is used by the POSIX system environment, the Java
Properties API, and the Windows Registry.

%As an example, in the error-fixing
%configuration setting \CodeIn{output\_format = XML} for JMeter in Section~\ref{sec:evolerror},
%\CodeIn{output\_format} is the configuration option name,
%and \CodeIn{XML} is the value. 

\subsection{Overview}

\ourtool is based on two key insights. First,
a program's control flow, rather than data flow, often propagates the majority of
the effects of a configuration option and determines
a program's execution path.
Second, the control flow differences of two execution
traces can approximate the program behavior differences
of two versions, and provide evidence
for which parts of the program might be behaving
abnormally and why.
%undesirably.

Based on these two insights, \ourtool uses three
steps to link different behaviors across program
versions to a specific configuration option that caused the difference.
Figure~\ref{fig:overview} sketches the high-level workflow of
\ourtool. 
%To recommend configuration options that
%fix the undesired behavior, 
In the first step, \ourtool asks the
user to demonstrate the different behaviors, using the same input and
configuration, on two \ourtool-instrumented program versions
(Section~\ref{sec:profiling}).
Then, \ourtool analyzes the two execution traces produced
by user demonstration, and identifies the control flow differences between
them. In particular, \ourtool identifies program predicates
that behave differently across two executions
(Section~\ref{sec:comparison}).
After that, \ourtool uses a lightweight dependence
analysis technique, called thin slicing~\cite{Sridharan:2007},
to statically reason about which configuration
options may cause the control flow differences.
Finally, \ourtool reports a ranked list of 
suspicious configuration options to the user (Section~\ref{sec:rootcause}).

\subsection{Instrumentation and Demonstration}
\label{sec:profiling}

\ourtool first instruments both the old and new program versions
to monitor the program  execution at runtime. \ourtool's instrumentation
consists of two parts:

\vspace{-2mm}

\begin{itemize}
\item For each program predicate, \ourtool inserts one
statement before and one statement after it
to monitor how frequent the predicate is executed, and
how often the predicate evaluates to true, respectively. In our
context, a predicate is a Boolean expression in a
conditional or loop statement,
whose evaluation result affects the program
control flow by determining whether to execute the
following statement or not.


\item For each of the other statements, \ourtool inserts
one statement before it to monitor whether the statement
gets executed or not at runtime.
\end{itemize}


\vspace{-1mm}

After instrumentation, \ourtool asks the user to demonstrate the different
behaviors on the two instrumented program versions, by using
the same input and configuration. Demonstration is
one of the simplest ways for an end-user to describe her problem;
and it is easier than writing specifications or scripts of any form.

Executing the instrumented program produces an execution trace,
which consists of a sequence of executed statements as well
as the evaluation result of each predicate.
The captured execution trace by \ourtool is by no means complete
in recording the full execution trace;  it only
captures the control flows a program is taking. As demonstrated
in our experiments, such control flow information serves
good approximation to determine which parts of the program behave
abnormally and why.
% are
%responsible for the behavioral difference and why.


%predicates and their
%evaluation results. Such captured predicate behaviors are by
%no means complete in recording the full execution trace. However,
%they capture control flows a program is taking. Thus, using the
%recorded predicate execution result, \ourtool could faithfully
%derive the full execution path. 



\subsection{Execution Trace Comparison}
\label{sec:comparison}


%\todo{Are variants 2 and three really variants of \ourtool, or are they
%  completely different techniques that should be presented as such?  In
%  particular, can you characterize variant 2 (here and in the table) as
%  Tarantula, possibly with some small enhancements.  I have a similar
%  question about variant 3.  In any event, make clearer what part of the
%  architecture is replaced by each variant.}

Another possible way to diagnose a configuration error is to leverage
the existing fault localization techniques, by treating the undesired
execution as a failing run and all correct executions (in the database)
as passing runs. We next compare \ourtool with two state-of-the-art
techniques: % in error diagnosis:

\begin{itemize}
\item \textbf{Statement-level Coverage Analysis}. This technique treats statements covered
by the undesired execution profile as potentially buggy, and statements
covered the correct execution profiles as correct.
Then, it leverages a well-known fault localization technique,
Tarantula~\cite{Jones:2002}, to rank the likelihood of each
statement being buggy, and queries the results of thin slice
to identify its affecting configuration options as the root causes.

\item \textbf{Method-level Invariant Analysis}. This technique stores invariants detected
by Daikon~\cite{Ernst:1999} from correct execution profiles in the database.
When a configuration error occurs, this technique detects invariants from the undesired execution profile;
and compares them with those stored in the database.
It treats a method to have suspicious behaviors if its observed invariants
from the undesired execution profile are different from the invariants stored
in the database~\cite{McCamant:2003}. Finally, this technique ranks a method's suspiciousness by
the number of different invariants, and queries the results of thin slice
to identify its affecting configuration options as the root causes. 
\end{itemize}


The experimental results are shown in Figure~\ref{tab:results} (Columns
``Coverage Analysis'' and ``Invariant Analysis'').
Both techniques produce less
accurate results; and for some errors, even fail to identify
the actual root causes.

In Coverage Analysis, the statement-level abstraction is \textit{too fine-grained}.
Many statements have the exactly same coverage by the failing/passing executions,
and thus have the same suspiciousness score as computed by Tarantula~\cite{Jones:2002}.
Furthermore, the underlying Tarantula technique only records whether a
statement has been executed or not but does not record how a statement is 
executed (e.g, how often a predicate is evaluated to true). The combination
of these two factors causes the low accuracy.


In Invariant Analysis, the method-level abstraction is \textit{too coarse-grained}.
A typical invariant detection technique like Daikon~\cite{Ernst:1999}
only checks program states at method entries and exits, resulting in likely invariants that
correspond to pre- and post-conditions. Thus, invariant detection is less
senstivie to the local control flow changes within a method (e.g., a predicate's
true ratio). In our study, Invariant Analysis failed to diagnose several errors because it
reported the same invariants for the method containing behavioral-deviated predicates.



%method-level granularity is too coarse
%is not sensitive enough on small control flow changes. for example, the invariant is the same


This experiment suggests that applying existing fault localization
techniques to diagnose configuration errors may be insufficient, due
to different abstractions and algorithms. It also indicates that
focusing on the relevant predicates' behaviors as \ourtool does
 can be a good choice.


\subsection{Configuration Option Recommendation}
\label{sec:rootcause}

% -*- LaTeX -*-

%An alternate path is any path not taken by the actual
%program execution that starts at a conditional
%branch instruction for which the branch condition
%is affected by one or more configuration options.


\begin{figure}[t]
\textbf{Auxiliary methods:}

\quad getAffectingOptions($\mathit{p}$, $\mathit{V}$): return all configuration options that affect predicate $\mathit{p}$ in the software version $\mathit{V}$

\quad getExecutedStmtNum($\mathit{p}$, $\mathit{V}$, $\mathit{T}$): return the number of executed statements (determined by predicate $\mathit{p}$) in trace $\mathit{T}$

\textbf{Input}: two software version: $\mathit{V_{old}}$ and $\mathit{V_{new}}$, 

\quad the map of all behaviorally-deviated predicates: $\mathit{predMap}$, produced by Figure~\ref{fig:identify}.

\textbf{Output}: {a ranked list of likely root cause configuration options}
\vspace{-4mm}%
recommendOptions($\mathit{V_{old}}$, $\mathit{V_{new}}$, $\mathit{predMap}$)\\
\begin{algorithmic}[1]
\STATE $\mathit{optionMap}$ $\leftarrow$ new Map$\langle$Option, Float$\rangle$
\FOR{each $\langle$$\mathit{p_{old}}$, $\mathit{p_{new}}$$\rangle$ in $\mathit{predMap}$.keys()}
\STATE $\mathit{v}$ $\leftarrow$ $\mathit{predMap}$[$\langle$$\mathit{p_{old}}$, $\mathit{p_{new}}$$\rangle$]
\IF{$\mathit{p_{old}}$ $\neq$ $\mathit{null}$}
\STATE $\mathit{options_{old}}$ $\leftarrow$ getAffectingOptions($\mathit{p_{old}}$, $\mathit{V_{old}}$)
\STATE $\mathit{v}$ $\leftarrow$ $\mathit{v}$ $\times$ getExecutedStmtNum($\mathit{p_{old}}$, $\mathit{V_{old}}$, $\mathit{T_{old}}$)
\FOR{each Option $\mathit{option}$ in $\mathit{options_{old}}$}
\STATE $\mathit{optionMap}$[$\mathit{option}$] $\leftarrow$ $\mathit{optionMap}$[$\mathit{option}$] + $\mathit{v}$
\ENDFOR
\ENDIF
\IF{$\mathit{p_{new}}$ $\neq$ $\mathit{null}$}
\STATE $\mathit{options_{new}}$ $\leftarrow$ getAffectingOptions($\mathit{p_{new}}$, $\mathit{V_{new}}$)
\STATE $\mathit{v}$ $\leftarrow$ $\mathit{v}$ $\times$ getExecutedStmtNum($\mathit{p_{new}}$, $\mathit{V_{new}}$, $\mathit{T_{new}}$)
\FOR{each Option $\mathit{option}$ in $\mathit{options_{new}}$}
\STATE $\mathit{optionMap}$[$\mathit{option}$] $\leftarrow$ $\mathit{optionMap}$[$\mathit{option}$] + $\mathit{v}$
\ENDFOR
\ENDIF
\ENDFOR
\RETURN $\mathit{optionMap}$.sortedKeys()
\vspace{-2mm}
\end{algorithmic}
\caption{Algorithm for recommending configuration options.
\label{fig:recommend}
}
\end{figure}


In this step, \ourtool attributes the control flow differences
to one or more root cause configuration options.
The key idea is to identify configuration options that
may affect the behaviorally-deviated predicates, and then rank
these options by the deviation value (computed by the
{deviation} method in Figure~\ref{fig:recommend})
and the number of executed statements they control (computed
by the {getExecutedStmtNum} auxiliary method in Figure~\ref{fig:recommend}).

%\subsubsection{Attributing Trace Differences to Configuration Options}

To identify the configuration options that can affect
a predicate, a straightforward way is to use program slicing~\cite{Weiser:1981}
to compute a forward slice from the initialization statement
of a configuration option, and then check whether the predicate is
in the slice. Unfortunately, traditional
full slicing~\cite{Weiser:1981} is infeasible due
to its conservatism, such as the need of handling pointers
and the need of following both data and control dependences.
%As been experimentally demonstrated in our
%previous work~\cite{}, traditional full slicing includes
%too much of the program, and can significantly affect an analysis's
%accuracy. 


To address this limitation, \ourtool uses thin slicing~\cite{Sridharan:2007}
to identify configuration options that \textit{directly} affect
a predicate. Different from traditional full slicing,
thin slicing \textit{only} follows the data flow dependencies
from the seed (i.e., the initialization statement of a
configuration option), and ignores control flow dependencies
as well as uses of base pointers. This property separates
pointer computations from the flow of configuration option
values and naturally connects a configuration option with its
directly affected statements. Section~\ref{sec:alternative}
empirically demonstrates that traditional full slicing includes
too much of the program, and can significantly affect an analysis's
accuracy; while thin slicing is a better
choice.


To distinguish the likelihood of each configuration
option being the root cause, \ourtool associates each
configuration option with a weight, which represents the strength of
the causal relationship between the configuration option
and the execution differences.
A larger weight value indicates that a configuration option
can potentially attributes more to the control
flow differences as its value propagates in the program, and thus
the configuration option is more likely to be the root cause.
%which is the number
%of affected statements, as its effects propagates
%in the program.  The weight. 
%If a configuration option
%potentially affects more statements that are decided
%by a behaviorally-different predicate, it is more likely
%to be the root cause.

Figure~\ref{fig:recommend} sketches the configuration option
recommendation algorithm.
For each predicate in both versions,
\ourtool first attributes the control flow difference 
to its affecting configuration options (lines 4 and 12). Then,
\ourtool computes the number of executed statements controlled
by that predicate (lines 5 and 13). To obtain the number of
executed statements controlled by a predicate,
the getExecutedStmtNum method first statically examines the
source code to compute the immediate post-dominant statement
of a predicate, and then traverses the execution trace to count
the number of statements that
are executed between the predicate and its post-dominant
statement. 
\ourtool mulitples the a predicate's cross-version deviation value 
with the number of executed statements, and updates the
weight of each affecting configuration option (lines 5--8
and 13--16).
%\ourtool first iterates each predicate pair with different
%behaviors between two versions (line 2), and then attribute
%the behavior difference to its affecting configuration options (lines 4--17).
%For each predicate pair, \ourtool analyzes the predicate
%in the old and new program versions separately in lines 4--10
%and lines 11--17, respectively. For a predicate, \ourtool
%first identifies all configuration options that can affect its
%behavior (lines 5 and 12). The weight of each configuration option
%is accumulated by summing up the result of multiplying the
%deviation value of its affected predicate with the number of
%executed statements determined by that predicate (lines 6 and 13).
Finally, \ourtool ranks all affecting configuration options
in decreasing order, and outputs a ranked list of suspicious
options that might be responsible for the
behavior differences (line 18). 

In \ourtool, if two configuration options have the
same weights, \ourtool prefers the configuration option
having more statements in its thin slice. This heuristic
is based on the intuition that configuration options affecting
more statements seem more likely to be relevant to the behavior
differences.

%In the output list, if an option affects 
%a predicates in the old program version,
%it indicates that option may have a different effect on the new
%version (e.g., the option has
%been renamed), and the user should switch to a different option.
%If an option affects predicates in the new version or in both
%versions, it indicates that the user should change its
%existing value.

%\todo{need more clarification of above.}


%Finally, the algorithm sums
%first summing
%the total number of instructions executed via this
%predicate. It attributes the divergence
%to root cause configuration options by multiplying
%the cost of the divergence by the weights of the configuration
%options that are relevant to the divergence.


%\ourtool statically analyzes
%the effects of each configuration option when it is
%assigned to a different value.
%Specifically, \ourtool propagates the effect of
%a configuration option to other program statements
%based on control flow dependencies in the program. 



%Two configuration options are considered equal
%root causes even if one has a direct causal
%relationship to a location (e.g., the value
%in memory was read directly from the configuration value)
%and another has a nebulous relationship (e.g., 
%its effects is propagated along a long chain
%of conditional assignments).

%Data flow dependencies are treated to be more likely
%to lead to the root cause than control flow
%dependencies. Control flow dependencies are assumed
%to be more likely to the root cause if they occur
%later in the execution (i.e., closer to the
%deviated execution paths).

%Assign control flow dependence only half the weight
%of the weight introduced by data flow dependencies.
%Further, each nested conditional branches reduces
%the weight by prior branch in the nest one half.

%what about 
%if(a) \{
%
%    if(b) \{
%        //should a and b equally important
%     \}
%    \}



%\todo{mention implementation details about recursive, avoid double count}


%\ourtool also tracks implicit control flow dependencies.

%A predicate's execution depends on the value of
%the configuration option, and the associated weight
%indicates the strength of the dependency.

%\ourtool assumes that control flow dependencies are more likely
%to lead to the root cause if they
%occur closer to the predicate being executed.

%This represents the belief that the execution of
%the basic block is affected by XXX.
%Since these are two independent probabilities:
%potentially changing either of the two options
%might cause the basic block to not have been executed.
%Thus, the weights of configuration options associated
%with a basic block need not sum to one.

%\ourtool determines the root cause of each behaviorally-different
%predicate.  



%Essentially, this step answers the question:
%``how likely is a configuration
%option cause the execution differences?''.

%\ourtool uses two heuristics: xxx.
%These heuristics cause real root causes to rank
%higher than false positives.


%\ourtool next determines why each different path
%gets executed. \ourtool associates each block
%with a set of root causes, More specifically,
%it uses thin slicing to identify a
%set of configuration options for each deviated
%execution path as the root causes.



\subsection{Discussion}
\label{sec:tech_discuss}

We next discuss some design issues in \ourtool.

\vspace{1mm}

\noindent \textbf{\textit{Fixing configuration errors 
vs. Localizing regression bugs.}}
The problem addressed in this paper is significantly different
than the traditional regression bug localization problem~\cite{dd, autoflow}.
A regression bug occurs when developers have made a mistake,
which causes software to violate its specification after a session of code changes.
By contrast, in our problem, the software behavior on the new version
is still as \textit{designed} but \textit{undesired}. 
%\ourtool addresses
%this problem from a specific angel by recommending configuration options
%to fix the undesired behavior.
Further, compared to
regression bugs introduced by software developers,
software configuration lies in the gray zone between
the software developers of software users. The responsibility for creating
correct configurations lies with both parties; the developer should create
intuitive configuration logic, build logic that detects
errors, and convey configuration knowledge to users
effectively. This shared responsibility makes recommend configuration options
different than localizing a regression bug, and causes many existing
techniques not applicable. We will discuss related techniques below
and in Section~\ref{sec:related}, and describe an empirical comparison
in Section~\ref{sec:evaluation}.


\vspace{1mm}
\noindent \textbf{\textit{Why not use a dynamic analysis to recommend
configuration options?}}
\ourtool uses thin slicing to statically identify responsible configuration
options for a behaviorally-deviated predicate. Another possible way is to use a pure
dynamic analysis to assess the causality of how a configuration option
may affect the control flow. State-of-the-art
techniques such as Delta Debugging~\cite{dd}, value replacement~\cite{failuredoc},
and dual slicing~\cite{Sumner:2013:CCE}
use a similar idea: they repeatedly replace a variable value with other alternaives,
and then re-execute the program to check whether the outcome is desired.
There are two major challenges that prevent these dynamic analyses
from being used. First, it is
difficult to find a valid replacement value for a configuration option.
For Boolean type option, it is trivial to find alternative values.
However, for many configurtion options with string or regular expression types, it
can be hard to determine good alternative values without a specification.
Second, automatically checking program outcomes requires
a testing oracle, which is often not available in practice, and end-users
should not be  expected to provide it. To address these challenges,
\ourtool approximates the program behavioral differences by the
control flow differences of two executions, and then statically reasons
about the responsible configuration options.

%However, a major challenge is that it is difficult for
%\todo{illustrate more clearly above}. Investigating how
%to combine static and dynamic analyses 


\vspace{1mm}
\noindent \textbf{\textit{\ourtool's current limitations.}}
There are four major limitations in the our \ourtool technique.
First, \ourtool currently assumes that only one
configuration option is responsible for the undeisred behavior,
although it can diagnose two errors caused by two independent options
at a time (Section~\ref{sec:evaluation}).
If fixing a particular configuration error
requires changing values of two configuration options,
then \ourtool may not identify both of them.
Second, \ourtool assumes the different behaviors
on two program versions are not caused by non-determinism.
For non-deterministic behaviors, \ourtool
could potentially leverage a deterministic replay
system~\cite{Huang:2013:CRL, Jin:2012:BRF} to faithfully reproduce the behaviors.
Third, \ourtool only matches one predicate in the old
program version to one predicate in the new program version.
If a predicate evolves into multiple predicates in the new
version, \ourtool may output less useful results. 
%We did not see such cases
%in our experiment, but we speculate that \ourtool
%may not produce useful matching results.
Fourth, \ourtool focuses on identifying root cause
configuration options that can change the functional behaviors of
the target program.
% rather than the underlying OS
%or runtime system. 
Configuration options that essentially affect the underlying
OS or runtime system, such as the \CodeIn{-Xmx} option used to
specify JVM's heap size when launching a Java program,
are not supported in \ourtool.
