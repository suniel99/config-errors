\begin{figure}[t]
\textbf{Axuiliary methods}:

sameStatement($\mathit{s}$, $\mathit{s'}$): return whether two statements
$\mathit{s}$ and $\mathit{s'}$ are the same statements. \todo{What does
  ``same statement'' mean?  The same source code?  The same subexpressions
  too? (For example, does this return true for if statement only if the
  predicate, then clause, and else clause are all identical?)}

BFS($\mathit{s}$, $\mathit{cfg}$): return an ordered list of reachable successive statements from statement $\mathit{s}$ in $\mathit{cfg}$ by Breath-First Search (BFS).

firstMatch($\mathit{stmtList_1}$, $\mathit{stmtList_2}$): return the first matched statement pair between $\mathit{stmtList_1}$ and $\mathit{stmtList_2}$ (iterating $\mathit{stmtList_1}$ first).
\todo{What is a ``matched statement pair''?}


\textbf{Input}: two methods from two software versions: $\mathit{m_{old}}$ and $m_{new}$,

\quad a maximum lookahead value $\mathit{lh}$ (Our experiment uses $\mathit{lh}=5$.).\\
\textbf{Output}: matched statements between old and new versions.
\vspace{-4mm}%
matchStatements($\mathit{m_{old}}$, $\mathit{m_{new}}$, $\mathit{lh}$)\\
\begin{algorithmic}[1]
\STATE $\mathit{matchedStmts}$ $\leftarrow$ new Map$\langle$Stmt, Stmt$\rangle$
\STATE $\mathit{cfg_{old}}$ $\leftarrow$ constructControlFlowGraph($\mathit{m_{old}}$)
\STATE $\mathit{cfg_{new}}$ $\leftarrow$ constructControlFlowGraph($\mathit{m_{new}}$)
\STATE $\mathit{stack}$ $\leftarrow$ new Stack$\langle$Pair$\langle$Stmt, Stmt$\rangle$$\rangle$
\STATE $\mathit{stack}$.push($\mathit{cfg_{old}}$.$\mathit{entry}$, $\mathit{cfg_{new}}$.$\mathit{entry}$)
\WHILE{$\mathit{stack}$ is not empty}
\STATE $\langle$$\mathit{stmt_{old}}$, $\mathit{stmt_{new}}$$\rangle$ $\leftarrow$ $\mathit{stack}$.pop()
%\IF{$\mathit{stmt_{old}}$ or $\mathit{stmt_{new}}$ has already been matched}
\IF{$\mathit{matchedStmts}$.keys().contains($\mathit{stmt_{old}}$) \\ \quad || $\mathit{matchedStmts}$.values().contains($\mathit{stmt_{new}}$)}
\STATE \textbf{continue}
\ENDIF
\IF{sameStatement($\mathit{stmt_{old}}$, $\mathit{stmt_{new}}$)}
\STATE $\mathit{matchedStmts}$[$\mathit{stmt_{old}}$] $\leftarrow$ $\mathit{stmt_{new}}$
\ELSE
\STATE $\mathit{stmtList_{old}}$ $\leftarrow$ BFS($\mathit{stmt_{old}}$, $\mathit{lh}$)
\STATE $\mathit{stmtList_{new}}$ $\leftarrow$ BFS($\mathit{stmt_{new}}$, $\mathit{lh}$)
\STATE $\langle$$\mathit{stmt_{old}}$, $\mathit{stmt_{new}}$$\rangle$ $\leftarrow$ firstMatch($\mathit{stmtList_{old}}$, $\mathit{stmtList_{new}}$)
\IF{$\langle$$\mathit{stmt_{old}}$, $\mathit{stmt_{new}}$$\rangle$ $\neq$ null}
\STATE $\mathit{stack}$.push($\langle$$\mathit{stmt_{old}}$, $\mathit{stmt_{new}}$$\rangle$)
\ENDIF
\ENDIF
\ENDWHILE
\RETURN $\mathit{matchedStmts}$
\end{algorithmic}
\vspace{-4mm}
\caption{Algorithm for matching statements from two methods.
\label{fig:matching}
\todo{This algorithm is suspicious to me.  For one thing, the size of the
  stack is always $\le 1$ --- why use a stack in that case?  For another
  thing, as soon as \emph{any} statement satisfies sameStatement(), then
  the algorithm terminates.  Is there a bug in the algorithm?}
}
\end{figure}


In this step, \ourtool compares two execution traces
produced by two program versions, and identifies the
control flow differences. To do so, \ourtool
first matches each predicate recorded in the old
execution trace in the new version (Section~\ref{sec:match_predicate}),
and then identify all predicates recorded in both execution traces that
behave differently (Section~\ref{sec:identify_diff}).


\subsubsection{Matching Predicates across Versions}
\label{sec:match_predicate}

For each predicate recorded in the old execution trace,
\ourtool matches it in the new program version.
To match a predicate, \ourtool first matches its declaring method,
by using two strategies. \ourtool uses
the first strategy that succeeds.

%, and then
%identifies its runtime behaviors in the undesired execution profile.


\begin{enumerate}
\item \textbf{Identical method name.} Return a method with the identical
fully-qualified name in the new version.
\item \textbf{Similar method content.} Return the method with
the most similar content in the new version. Given
a method in the old program version, \ourtool
uses the algorithm shown in Figure~\ref{fig:matching} (discussed
in detail below) to \textit{every} instruction in it to
\textit{each} method in the new program version, and then
chooses the method with the most matched matched instructions
(in the new program version).
\end{enumerate}

The algorithm in Figure~\ref{fig:matching} is inspired by
a well-established program differencing algorithm, called
JDiff~\cite{}. \todo{describe JDiff and the difference here.}

If there is no matched method in the new program
version, \ourtool concludes that the predicate cannot be
matched. Otherwise, \ourtool uses the following two strategies
to find the corresponding predicate.

\begin{enumerate}
\item If the declaring method is matched by using the first
``Identical method name'' heuristic, \ourtool runs the algorithm
in Figure~\ref{fig:matching} to establish the mapping between
each instruction, and returns the matched instruction of the
predicate.
\item If the declaring method is matched by using the second
``similar method content'' heuristic,  \ourtool
uses the result of the algorithm in Figure~\ref{fig:matching}
by returning the matched instruction of the given predicate.
\end{enumerate}

%For both cases, if the matching algorithm fails to identified


\begin{figure}[t]
\textbf{Input}: two versions of the software: $\mathit{V_{old}}$ and $V_{new}$

\quad two execution traces on the two versions: $\mathit{T_{old}}$ and $T_{new}$\\
\textbf{Output}: a ranked list of suspicious configuration options\\
\vspace{-4mm}%
diagnoseConfigurationErrors($\mathit{V_{old}}$, $\mathit{V_{new}}$, $\mathit{T_{old}}$, $\mathit{T_{new}}$)\\
\begin{algorithmic}[1]
\STATE 
\RETURN $\mathit{weightMap}$.sortedKeys()
\end{algorithmic}
\caption{Algorithm for diagnosing configuration errors. The
procedure \CodeIn{updateWeights} is defined in Figure~\ref{fig:update}.
\label{fig:identify}
}
\end{figure}


\subsubsection{Identify Execution Differences}
\label{sec:identify_diff}

With the predicate matching information, \ourtool further
identifies predicates that behave differently
between two versions. 

The algorithm is shown in Figure~\ref{fig:identify}. \ourtool
\todo{describe the algorithm ..}

\ourtool outputs three categories of predicates:
(1) predicates that are only executed in the
old version, (2) predicates that are only executed
in the new version, and (3) predicates that
are executed in both versions but exhibit different
behaviors. For each category, \ourtool sorts the
predicates based on their deviation scores in the
descreasing order.


Differences in predicate behaviors indicate different executed paths
between two versions. Such differences provide evidence of
which part of the program might be behaving unexpectedly and why.

%is any path not taken by the actual
%program execution that starts at a conditional
%branch instruction for which the branch condition
%is affected by one or more configuration options.

