\section{Introduction}
\label{sec:introduction}

Many modern software systems are configurable. They
have a large number of configuration options for users
to customize their behaviors. This flexibility has a cost:
a small configuration error might lead to hard-to-diagnose
behaviors.
%when something goes wrong, diagnosing a configuration
%error can be both time-consuming and furstrating.

Software configuration errors are errors in which
the application code is correct, but the software is
configured incorrectly so that it does not behave
as desired. Such errors may lead software to crash,
produce erroneous output, or simply perform poorly.
In practice, software configuration
errors are \textit{prevalent}, \textit{severe}, and
\textit{hard to debug}, but they are
\textit{actionable} for users
to fix.


In deployed software systems, configuration errors have been
the dominant cause of problems~\cite{}.
A recent analysis of Yahoo's mission-critical Zookeeper service
showed that software misconfigurations accounted for
the majority of all exhibited bugs~\cite{bft}. Another
recent study analyzed reported problems of a commercial
storage company, and found configuration-related issues
caused about 31\% of all failures~\cite{Yin:2011:ESC}.
Not only are configuration errors prevalent, they
can have high, sometimes disastrous impacts. For example,
an outage in Facebook, due to
an incorrect configuration value, left the website 
inaccessible for about 2 hours~\cite{fbout}. 
The entire .se domain of country Sweden was unavailable
for about 1 hour, due to a DNS misconfiguration problem~\cite{sedown}.
Such incidents affected millions of users. Furthermore, configuration
errors are difficult to diagnose.
As reflected in a recent article about system deployment experience
in Google, the vast majority of production failures (in Google)
arise not due to bugs in the software, but bugs in the
configuration settings (i.e., configuration errors)
that control the software. Debugging
configuration errors has become a difficult problem; 
techniques helping escape from the ``configuration hell''
are highly demanded~\cite{googleconf}.

However, on the other hand, different than software bugs
which can only be fixed by experienced software developers, fixing a software
configuration error is more \textit{actionable} for software end-users
or system administrators. These users are not the software developers,
and do not have the right expertise to understand (or even access to)
the source code;  but they can simplify fix a configuration error by changing
values of certain configuration options.


\subsection{Configuration Errors Caused by Software Evolution}

Continual change is a fact of life for software systems.
Among software changes, configuration changes are prevalent.
As described in our study of 9 configurable software systems,
configuration changes happen in \textit{every} studied
version of \textit{each} system. In many
cases, reusing existing configuration \textit{without} appriopriate
\textit{reconfiguration} to the new
version can lead the software exhibit \textit{undesired} behaviors,
even the software is working exactly as \textit{intended}.
In practice, such configuration errors can lead to serious results.
For example, on July 26 2012, after software updating,
a small configuration change caused a system error, and resulted in
Microsoft's public cloud hosting and development platform, Azure,
being unavailable for about two and a half hours~\cite{msdown}.

%After upgrading to a new version,
%users often need to carefully exam the existing configuration, and may
%configure the software properly. \todo{xx}
%Otherwise, the software is working exactly as intended, but the
%wrong configuration is leading it to exhibit the undesired behavior.
%\todo{not the right place}
%Inappropriate reconfiguration can lead to serious results.

Different than software regression bugs, configuration errors
can still happen even after a unrealistically comprehensive
regression test suite (with 100\% coverage ) passes. This is primarily
because configuration errors are mostly user driven.
They usually do not indicate problems in the software itself, and
often occur when the use of software is unexpected
situations, in which it does not behave as \textit{desired} but
exactly as \textit{intended}. For software developers,
it is impossible for them to test software in every possible
situation in which it might be misconfigured; in fact, it is usually
impossible even to foresee every such situation. 


Take the popular JMeter performance testing tool as an example, 
in version 2.8, the testing report is saved as an XML
file when using a command\footnote{\CodeIn{jmeter -n -t ../threadgroup.jmx -l ../output.jtl -j ../testplan.log}} described in the user manual.
However, when upgrading to the new 2.9 version, 
using the same command, the testing report is saved
in a CSV file. Further, all regression tests between
these two versions passed; the new JMeter version
just behaves as \textit{intended} but \textit{differently}
than a user would expect.


%To understand why a configuration error happens and
%troubleshoot it to obtain the desired behavior,
%users often need to seek information
%from online help forums, software manuals, or
%ask experts. This process can be tedious, laborious, and frustrating.

Our technique (and its tool implementation \ourtool) can help
diagnose configuration errors. For the JMeter example,
users first demonstrate the desired and undesired
behaviors on two \ourtool-instrumented
JMeter versions, respectively. Then, \ourtool analyzes the
recorded execution profiles produced by two versions, and produces
a ranked list of root cause configuration options. At the
\todo{top, need to confirm} of the list is the
\CodeIn{output\_format} option whose 
value has been changed from \CodeIn{XML} to \CodeIn{CSV}
between two versions. To restore JMeter's desired
behavior as storing results in a CSV file, users
need to reconfigure its value to \CodeIn{XML}.

\subsection{Diagnosing Configuration Errors}

Broadly speaking, diagnosing a configuration
error can be divided into three separate tasks:
reproducing the error, identifying which specific
configuration option is responsible for the unexpected
behavior, and determining a better value for the
configuration option to fix the error. \ourtool addresses
the second task: identifying the root cause of
a configuration error.

\ourtool aims to help two types of users: software end-users
who may have problems with software installed on their
personal computers; and system administrators who are
responsible for maintaining production systems.
They can use \ourtool for error diagnosis.
%to troubleshoot an
%error they encounter but do not know how to fix it. 

%when they counter
%an error that they do not know how to fix, to troubleshoot a
%configuration error 
%\ourtool's output can help such users resolve the problems
%they encounter.

%\ourtool focuses on diagnosing configuration errors
%caused by software evolution. 
The key idea of \ourtool is to identify differences between
two execution traces (produced by desired and
undesired behaviors on two versions, respectively),
and then use static dependence analysis to reason
about which configuration options
might cause such differences. It uses
three steps, as illustrated in Figure~\ref{fig:overview},  to link the undesired
behavior to specific root cause configuration options:

\begin{itemize}

\item \textbf{Instrumentation and Profiling.} \ourtool
instruments two program versions by monitoring the
execution of every predicate, and then asks users to
demonstrate the different behaviors on two instrumented
program versions. User demonstration produces two execution
traces resulting from the desired and undesired program
behaviors, respectively.

\item \textbf{Execution Trace Comparison.}
\ourtool analyzes two execution traces and identifies 
predicates where the desired execution and the undesired
execution diverged. The behavioral differences in
the recorded predicates provide evidence for what
predicates in a program might be behaving abnormally
and why.

\item \textbf{Root Cause Analysis.} For each behaviorally-different
predicate between two executions, \ourtool analyzes its
causes. It uses a lightweight static dependence analysis
technique, called thin slicing~\cite{}, to attribute
differences to specific configuration options and
reports a ranked list of options to the users.


\end{itemize}

Compared to existing error diagnosis
techniques~\cite{}, \ourtool differs in three
key aspects:

\begin{itemize}
\item \textbf{It diagnoses configuration errors across versions}.
Most existing configuration error diagnosis techniques
exclusively focus on identifying errors in a single program
version~\cite{}. By contrast, \ourtool focuses on
configuration errors caused by software evolution, and
targets two different versions of the same program. 
It uses the desired behavior on the new software version
as a baseline with which to compare program behavior against, and only
reasons about the behavioral differences.


\item \textbf{It requires no testing oracle}.
Some existing work~\cite{} require users to answer the difficult
questions like ``why is the software not working'', or
``is the software currently working'' by writing a testing
oracle to check the software behavior. By contrast,
\ourtool eliminates such assumption and only requires users to
demonstrate the different behaviors on two versions.
\ourtool uses the execution trace produced in the old
version as an approximate oracle to
reason about the error in the new version.

\item \textbf{It determines likely root cause options}.
Many error diagnosis and debugging techniques primarily focus on
determining \textit{what} changes are introduced between
two versions -- they leave the more challenging
question of \textit{how} to eliminate the undesired effects
by those changes unanswered. Users must manually infer
the root cause, e.g., a misconfiguration,
of the unexpected behaviors from the technique output
based on their expertise and knowledge of the software.
By contrast, \ourtool explicitly guides users to specific
configuration options which may help fix the error.

\end{itemize}

\subsection{Evaluation}

We implemented \ourtool for Java software, and empirically evaluated
its effectiveness using \subjnum open-source configurable Java
software. We employ \ourtool to identify behavioral differences between
two versions of these software.
\todo{describe experiment results above}

We compared \ourtool to two existing techniques~\cite{}, which uses XXX.
The existing technique only
\todo{will compare it to ConfDiagnoser and ConfAnalyzer, show results
above} 

Overall, we find that \ourtool is highly effective at helping
identify the root cause of cross-version behavioral differences. 

\subsection{Contributions}

This paper makes the following main contributions:

\begin{itemize}
\vspace{-3mm}
\item \textbf{Study.} We describe an empirical
study of 9 configurable software systems.
Our study indicates that software configuration changes
are frequent and persistent during its evolution (Section~\ref{sec:study}).

\item \textbf{Technique.} We present a technique to diagnosis
configuration errors for evolving software. Our technique
uses dynamic profiling, execution trace comparison, and
static analysis to link certain undesired behaviors to a
specific responsible configuration option (Section~\ref{sec:technique}).

\item \textbf{Implementation.} We implemented our technique
in a tool, called \ourtool, for Java software (Section~\ref{sec:implementation}).
It is publicly available at \url{http://config-errors.googlecode.com}.

\item \textbf{Evaluation.} We applied \ourtool to \errornum configuration
errors from \subjnum configurable software systems,
and compared it with existing techniques.
The results show the accuracy and efficiency of \ourtool (Section~\ref{sec:evaluation}).
\end{itemize}
