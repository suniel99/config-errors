
\section{Evaluation}
\label{sec:evaluation}

We evaluated 4 aspects of \ourtool's effectiveness, answering
the following research questions:

\begin{enumerate}
\item How accurate is \ourtool in diagnosing configuration errors
caused by software evolution? That is, what is the rank of the
actual root cause configuration option in \ourtool's output (Section~\ref{sec:accuracy})?

\item How long does it take for \ourtool to diagnose
a configuration error (Section~\ref{sec:timecost})?

\item How does \ourtool's effectiveness compare to
existing approaches (Section~\ref{sec:existing})?

\item How does \ourtool's effectiveness compare to
an alternative approach based on \todo{describe it} (Section~\ref{sec:alternative})?

\end{enumerate}

\subsection{Subject Programs}

We evaluated \ourtool on the \subjnum Java programs
described in our empirical study (Table~\ref{tab:subjects}).

\subsubsection{Configuration Errors Across Versions}


\begin{figure*}[t]
\vspace{1mm}
\centering
\small{
\setlength{\tabcolsep}{.26\tabcolsep}
\begin{tabular}{|l|c|c|c|c|c|c|c|}
\hline
 Error ID.& Error Description & Root-Cause  & \#Options & \multicolumn{3}{|c|}{Rank of the Root-Cause Configuration Option}\\
 \cline{5-7}
 Program& & Configuration Option& & \ourtool & \prevtool~\cite{Zhang:2013:ADS}  & \conftool~\cite{Rabkin:2011:PPC} \\
 %&  Options &  Options&  Options \\
 \hline
 \hline
 1. Randoop& Poor performance in test generation & \CodeIn{usethreads} & \randoopoptnum & \randooprank & \n & \x \\
 2. Weka & A different error message when Weka crashes & \CodeIn{m\_numFolds} & \wekaoptnum &  \wekarank & 9 & 1 \\
 3. Synoptic&  Initial model not saved& \CodeIn{dumpInitialGraphDotFile} & \synopticoptnum & \synopticrankfirst & \n & \x \\
 4. Synoptic& Generated model not saved as JPEG file& \CodeIn{dumpInitialGraphPngFile} &\synopticoptnum & \synopticranksecond & \n & \x \\
 5. JChord& Bytecode parsed incorrectly & \CodeIn{chord.ssa} & \jchordoptnum & \jchordrankfirst & 3& \x \\
 6. JChord& Method names not printed in the console& \CodeIn{chord.print.methods} & \jchordoptnum & \jchordranksecond & \n & \x \\
 7. JMeter& Results saved to a file with a different format & \CodeIn{output\_format} & \jmeteroptnum & \jmeterrank & 1 & \x \\
 8. Javalanche& No mutants generated & \CodeIn{project.tests} &  \javalancheoptnum & \javalancherank& 4 & \x \\
\hline
\hline
 Average &  & & 49.1 & \averagerank &15.3 & 47.5\\
\hline
\end{tabular}
}
\vspace{-2mm}
\Caption{{\label{tab:errors} 
All configuration errors used in the evaluation and
the experimental results. Only the 2-nd error is
a crashing error, and all the other errors are non-crashing
errors. Column ``Root-Cause Configuration Option''
shows the actual root-cause configuration option.
Column ``\#Options'' shows the number
of configuration options supported in the new program version, taken from
Figure~\ref{tab:experiment-sub}. 
Column ``Rank of the Root-Cause Configuration Option'' shows the
absolute rank of the actual root-cause
configuration option in each technique's output (lower is better).
%``\N'' means the technique does not identify the root casue configuration options.
``\x'' means the technique is not applicable (i.e., requiring a crashing
point), and ``\n'' means the technique does not identify the actual root cause.
When computing the average rank, each ``\x'' or ``\n'' is treated as
half of the number of configuration options, because a user would need to examine
on average half of the avaiable options to find the root cause.
Column ``\ourtool'' shows the results
of using our technique. Columns ``\prevtool'' and ``\conftool'' shows
the results of using two existing techniques as described in Section~\ref{sec:existing}.
}
}
\end{figure*}


For each subject program, we manually examinated all
configuration-related changes listed in Table~\ref{tab:options},
and wrote a test driver that revealed different
behaviors on two corresponding versions for each configuration change.

We collected \errornum configuration errors caused by
software evolution. \todo{reasons of relatively few
errors}. First, we excluded option additions, since
those options are not available in the old version.
Second, we excluded option changes that are backward
compatible. Third, for such option changes, if the
new program version already identifies the potential
configuration errors by dumping explicit error messages,
we excluded them. No tool support is even needed
for this case.

We evaluated all configuration errors that we can reproduce;
we did not select only errors that \ourtool works well.
Table~\ref{tab:errors} lists all errors.

\subsection{Evaluation Procedure}

\subsection{Results}

\subsubsection{Diagnosis Accuracy}
\label{sec:accuracy}

\subsubsection{Performance of \ourtool}
\label{sec:timecost}

We measured \ourtool's performance in two ways:
the time cost of diagnosing an error and
the overhead introduced by instrumentation
in reproducing an error.

Table~\ref{tab:performance} shows the results.

\begin{table}[t]
\vspace{1mm}
\centering
\small{
\setlength{\tabcolsep}{.80\tabcolsep}
\begin{tabular}{|l||c|c|c|c|}
\hline
 Configuration & \multicolumn{2}{|c|}{Run-time Slowdown ($\times$)} & \ourtool time\\
 \cline{2-3}
 Error ID& Old Version & New Version & (seconds)\\
 %&  Options &  Options&  Options \\
 \hline
 \hline
 1 &  &  &  \\
 2 &  &  &  \\
 3 &  &  &  \\
 4 &  &  &  \\
 5 &  &  &  \\
 6 &  &  &  \\
 7 &  &  &  \\
\hline
\hline
 Mean & & & \\
\hline
\end{tabular}
}
\vspace{-2mm}
\Caption{{\label{tab:performance} \ourtool's
performance. The ``Run-time slow down'' column
shows the cost of reproducing the error in
an \ourtool-instrumented version of the subject program.
The ``\ourtool time (seconds)'' column shows
the time took by \ourtool to diagnose configuration errors.
For both columns, the mean is the geometric mean. 
}
}
\end{table}

\subsubsection{Comparison with Existing Approaches}
\label{sec:existing}

\todo{want to compare with ConfDiagnoser and ConfAnalyzer}

\subsubsection{Comparison with Alternative Approaches}
\label{sec:alternative}

\todo{use ConfDiagnoser's ranking, full slicing, no iterative slicing}

\subsection{Discussion}

\noindent \textbf{\textit{Threats to validity.}}

\vspace{1mm}
The use case for reproducing differences may not reflect
users.

\noindent \textbf{\textit{Experimental conclusions.}}
