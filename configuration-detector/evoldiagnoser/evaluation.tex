
\section{Evaluation}
\label{sec:evaluation}

\subsection{Subject Programs}




\subsubsection{Configuration Problems}

\subsection{Evaluation Procedure}

\subsection{Results}

\subsubsection{Diagnosis Accuracy}

\subsubsection{Performance of \ourtool}

\subsubsection{Comparison with Existing Techniques}

\subsubsection{Evaluation of Two Design Choices}

\subsection{Discussion}

Software configuration lies in the gray zone between
the software developers of software end-users.
The responsibility for creating correct configurations
lies with both parties; the developer should create
intuitive configuration logic, build logic that detects
errors, and convey configuration knowledge to users
effectively; \todo{evolve} the end-user should imbibe the
knowledge and manage cross-version configurations.
This shared responsibility is non-trivial to efficiently
achieve. For example, there is no obviously ``correct'' way to
build configuration logic; also, unlike fixing a bug once,
every software user has to be informed on the right way to
configure the system. Perhaps as a result, misconfigurations
have been one of the dominant causes of system issues and
is likely to continue so.

