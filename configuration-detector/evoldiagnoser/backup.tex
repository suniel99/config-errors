
Difference summarization first attributes XXX costs
such as xxx and xxx to xxx. Then, it uses xxx to
associate each xxx with a set of configuration options.
The cost of each xxx is assigned to potential
root cause configuration options by the relative
probability that the particular root cause
led to the execution of that predicate. Finally,
the per-cause costs for xxx in the program execution are
summed together. The end result is a ranked
list of configuration options ordered by
their likelihood of being the root cause.

Difference summarization identifies branches where the
desired execution and the undesired execution diverged.
It assigns a XXX cost to each path taken from the branch,
then uses XXXX analysis to determine why the
two executions diverged at that point. It attributes
the difference in XXX costs between the
two executions to the identified root causes according to their
relative likelihood. The costs of all such divergences are
summed. The output of \ourtool is a set of reasons of why
the output of two requests differ, along with a
relative impact for each reason.


To find behavioral differences, \ourtool first compares the collected
execution profiles using xxx. \ourtool then classifies these
differences into categories that distinguish a likely
reason behind the differences: undesired behavioral changes,
indirect code modification effect,
desired behavioral changes. Finally, 


\ourtool automatically tracks behavioral differences between two
versions to provide users with feedback as to how to re-configure
the software.


The goal of this work is to reduce the burden
of diagnosing software errors caused by software evolution. 
In particular, we anayze the applicability of
\textit{dynamic trace analysis} for diagnosing configuration
errors.
Even when a software system outputs an error message
as a configuration error occurs,
it is often cryptic or misleading~\cite{}. Users may not
even think of configuration as a cause of their problem.

small configuration errors can lead to hard-to-diagnose
undesired behaviors. 

The goal of\ourtool is to improve the configuration error diagnosis
process by automating it as much as possible.

when something goes wrong,
diagnosing a configuration problem can be both
time-consuming and furstrating.

Software evolves through the introduction of new features, bug fixes,
performance optimizations, and refactorings. 

For example, software
patches can break existing functionality~\cite{}.
Seemsly unrelated library upgradation can
\todo{xxx} -- for example, by corrupting Windows
registry keys or shared configuration options.
Changes to security  \todo{xxx}

%\todo{such evolution may fail the original configuration,
%and require reconfiguration}


To help users troubleshoot configuration problems caused
by software evolution, this paper presents
a technique (and its tool implementation, called
\ourtool) to identify the root cause of a configuration
error between two software versions.  \ourtool takes as input
two versions of a configurable software,
an input, and a configuration. Using the same configuration,
the input produces \textit{desired} behavior
on the old version but exhibits \textit{undesired} behavior on the new version.
\ourtool automatically diagnoses such regression
error, and outputs a single configuration option
that can be changed to produce desired behavior on the
new software version.

\todo{xx} Given two executions and some observed differences between them,
\ourtool reasons about the causes of such differences -- a single configuration
\todo{xx}. \ourtool does so by \todo{xx}. it makes a number
of key advances: it properly handles xxx (changes); it also leverages
xx ().

\todo{xx} \ourtool detects configuration options whose values
that may contribute to the behavioral differences
between two versions.

