\section{Related Work}
\label{sec:related}

The most closely related work falls into
four categories; (1) techniques for
supporting software evolution; (2) software
configuration error diagnosis techniques;
(3) automated software debugging techniques;
and (4) configuration-aware software analysis techniques.

\subsection{Supporting Software Evolution}

As software evolves, its behavior must be validated.
Regression test selection and prioritization~\cite{}
indicate which tests need to be executed for a changed
program. When a regression test fails, developers need
to understand its root cause~\cite{}. To address this
problem, many error diagnosis techniques are developed for evolving
software. For example, Delta Debugging aims to find a minimal
subset of changes that still makes the test fail~\cite{}.
Test minimization techniques~\cite{} simplify the failed
test to ease comprehension for developers. Recently,
Qi et al.~\cite{} proposed a symbolic execution-based
error diagnosis approach to synthesize new inputs that
differ marginally from the failing input in their
control flow behavior, then compare the execution traces
of the failing input and the new inputs to obtain critical
clues to the root-cause of the failure. \todo{todo ..} 

\todo{cite golden impl~\cite{Banerjee:2010:GID}}

\todo{discuss some matching techniques here, as well
as the matching algorithm}
Program differencing techniques~\cite{} try to identify
differences between two program versions.
\todo{give some example here}
Related, change impact analysis techniques, which
are often built on top of such program differencing
techniques, identify not only the changes, but also
which code fragments are affcted by which
changes. For example, a recent work~\cite{} uses symbolic
execution to accurately capture behavioral differences
between program versions. \todo{the semantic diff}
\todo{comparison} Overall, the works on program
differencing try to identify possible software regressions,
rather than finding the root-cause of a given software regression
\todo{need revise}

Zhang et al.~\cite{} developed a technique to match entire
execution histories of two program versions running with
the same input~\cite{}. The execution history contains control
flow taken, values produced, addresses referenced and
data dependences. Significantly different than \ourtool, their
work assumes semantically equivalent versions (e.g. optimized
and unoptimized) while \ourtool compares different version of
a program that can include functional changes. 

Use execution traces for detecting couplings~\cite{Giroux:2006:DIF}


\subsection{Software Configuration Error Diagnosis}

Software configuration problems are time-consuming
and frustrating to diagnose. To reduce the time and human
effort needed to troubleshoot software misconfigurations,
several prior research efforts have applied different techniques
to the problem of configuration error diagnosis~\cite{Attariyan:2008:UCD, 
Whitaker:2004:CDS, Wang:2004:AMT,
Attariyan:2010:ACT, Rabkin:2011:PPC, keller:conferr}.
Chronus~\cite{Whitaker:2004:CDS} relies on a user-provided
testing oracle to check the behavior of the system, and uses
virtual machine checkpoint and binary search to find the
point in time where the program behavior
switched from correct to incorrect. AutoBash~\cite{Su:2007:AIC}
fixes a misconfiguration by using
OS-level speculative execution to try possible
configurations, examine their effects, and roll them back when necessary.
PeerPressure~\cite{Wang:2004:AMT} 
uses statistical methods to compare
configuration states in the Windows Registry on different machines.
When a registry entry value on a machine exhibiting erroneous behavior differs
from the value usually chosen by other machines, PeerPressure
flags the value as a potential error. More recently, ConfAid~\cite{Attariyan:2010:ACT}
uses dynamic taint analysis to diagnose configuration problems 
by monitoring causality within the program binary as it executes.
\todo{discuss X-ray here}
ConfAnalyzer~\cite{Rabkin:2011:PPC} uses dynamic information flow analysis to precompute
possible configuration error diagnoses for every possible crashing point
in a program. 

Our technique is significantly different from the other approaches.
First, most previous approaches focus exclusively on a single software
version and diagnose configuration errors that lead to a crash or
assertion failure~\cite{Attariyan:2008:UCD, Whitaker:2004:CDS, 
Attariyan:2010:ACT, Rabkin:2011:PPC}.
\todo{multiple versions}
Second, several approaches~\cite{Attariyan:2010:ACT, Whitaker:2004:CDS}
assume the existence of a testing oracle that can 
check whether the software functions correctly. However,
such oracles are often
absent in practice or may not apply to the specific configuration problem.
A typical software user should not be expected, to invest the substantial
time and effort to create an oracle.
By contrast, our technique eliminates this assumption by
\todo{comparing traces}
Third, approaches like PeerPressure~\cite{Wang:2004:AMT}
and \todo{xiong's paper} benefit from
the known schema of the Windows Registry and
\todo{xx}, but cannot detect configuration errors
that lie outside these specific domains. Our technique
of analyzing the execution traces is more general.

In our previous work~\cite{}, we proposed \prevtool, an automated
software configuration error diagnosis tool. \ourtool
differs from \prevtool in three key aspects.
First, \todo{versions ...}. Second, \todo{no assumption}.
Third, \todo{algorithm differences}

\todo{cite some re-configuration papers, see bib}


\subsection{Automated Debugging}

Over the years, researchers have proposed
increasingly sophisticated debugging techniques,
so the body of this category is broad.
For the sake of space, we focus on representative techniques
that are most related to \ourtool.

Program slicing~\cite{} is a
well-known technique to determine which statements and
inputs could affect a pariticular program variable.

\todo{cite differential slicing, dual slicing, semantic trace
analysis, tao wang}
different executions with different inputs.

\todo{cite differential static analysis from MSR ppl, change contract,
differential symbolic exe, robbilard's work of coupling changes}

\todo{discuss dual slicing, trace causality comparison}

\todo{discuss why such techniques cannot be used}

\todo{cite bugex}

Formula-based debugging techniques use static and
dynamic analyses to transform programs and program executions
into formulas and manipulate these formulas to locate suspicious
statements. BugAssist~\cite{} transforms a program, an input that
makes the program fail, and the negation of the failing assertion
for that input into a formula that consists of clauses in
conjunctive normal form. It then treats the satisfiability of
the formula as a MAX-SAT problem ~\cite{},
so as to identify a subset of clauses
that, if removed, make the formula satisfiable. Finally, it reports
the code corresponding to these clauses as a potential cause of the
failure. A similar approach, Error Invariants~\cite{},
transforms a single failing trace, rather than a complete program,
into a formula. This technique leverages interpolants to identify
the points in the
failing trace where the state is modified in a way that affects the 
outcome of the execution. It then reports the statements in these
points as potential causes of the failure. One limitation of this
technique is that it cannot handle control flow related faults because it
does not encode control flow information in the formula.
\todo{comparison for differences}
 Formula based approaches can identify potential causes precisely
 and in a principled way, but they rely on an
extremely expensive analysis that prevents these techniques from
being applicable in practice, at least in their current formulation.



Statistical algorithms have been applied to the automated
debugging domain. \todo{cite tarantula}
The CBI project~\cite{} analyzes executions
collected from deployed software to isolate software failure
causes. Significantly different than \ourtool, CBI correlates
a predicate's evaluation result (either true or false) rather
than its true ratio and execution frequency with the observed
behaviors. In addition, CBI identifies likely buggy statements as
its final output, while ConfDiagnoser identifies the
behaviorally-deviated program predicates and links the undesired behavior
to specific configuration options. \todo{needs revise}
\todo{cannot be applied to our cases}
\todo{integrate with trace synthesisi work}


\subsection{Configuration-Aware Software Analysis}

Empirical studies show that configuration errors are pervasive, costly,
and time-consuming to diagnose~\cite{Yin:2011:ESC, Hubaux:2012}.
\todo{say the empirical study results of this paper.}
To alleviate this problem, researchers have designed various
software analysis techniques to improve configuration
management~\cite{Garvin:2011} and understand and test
the behavior of a configurable software
system~\cite{Garvin:2011, Qu:2008:CRT}.
For example, Garvin et al.~\cite{Garvin:2011} show how to
use historical data to avoid system failures during reconfiguration.
However, their work aims to reduce the burden of
configuration management and prevent certain errors from happening,
while our technique is designed to diagnose an exhibited problem.
Qu et al.~\cite{Qu:2008:CRT} proposed a combinatorial interaction
testing technique to model and generate configuration samples for
use in regression testing. Compared to \ourtool, those techniques
can be used to find new errors in
a configurable software system earlier, but cannot
identify the root cause of a revealed configuration error.
\todo{differences} By contrast, our technique is designed
to diagnose an exhibited error.
\todo{none of them aim to find regression}

cite SPLat paper at FSE'13


cite papers~\cite{Huang:2013:CRL, Thummalapenta:2010:ESM, Xing:2005:UAO, r2fix}
papers~\cite{Kim:2013, Jin:2012:BRF,xray,Nguyen:2010:RBF,Dig:2006:ADR,
Apel:2009:FLA, Shang:2013:ADB, rangefix, Kamiya:2002:CMT, Staats:2011:PTO}
papers~\cite{Mostafa:2009:TPA}

cite papers~\cite{Grechanik:2009:MEG, Fischer:2005:SET, publication-8111}
production line, configuration modelling~\cite{Acher:2012:SCF}

cite evolution~\cite{Bhattacharya:2012:GAP, Nguyen:2010:GAA}

use history for bug predication~\cite{Nagappan:2006:UHI}

software product line~\cite{Bodden:2013:SLS, Kang:2005:FRL, Mende:2008:SGM, Kruger:2005:SAE}

configuration~\cite{Denaro:Self-Test:TACOS:2003, Cooray:2010:RRD, Barreiros:2009:MRC, TerBeek:2011:GCE}

study~\cite{Rabiser:2012:QSU}
cite tarantula~\cite{Jones:2002}
