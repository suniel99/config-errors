
\begin{figure}[t]
\textbf{Auxiliary methods:}

\quad getPredicates($\mathit{T}$): returns all executed predicates in the execution trace $\mathit{T}$.

\quad getAffectingOptions($\mathit{p}$, $\mathit{V}$): uses thin slicing~\cite{Sridharan:2007} to compute all configuration options that may affect predicate $\mathit{p}$ in the software version $\mathit{V}$. 

\quad getExecutedStmtNum($\mathit{p}$, $\mathit{V}$, $\mathit{T}$): return the number of executed statements (controlled by predicate $\mathit{p}$) in trace $\mathit{T}$ from software version $\mathit{V}$.

\textbf{Input}: two software versions: $\mathit{V_{old}}$ and $\mathit{V_{new}}$. 

\quad two execution traces: $\mathit{T_{old}}$ and $\mathit{T_{new}}$, on both versions.

\quad a map of matched statements between $\mathit{V_{old}}$ and $\mathit{V_{new}}$ : $\mathit{stmtMap}$.

\textbf{Output}: {a ranked list of likely root cause configuration options}

\vspace{1mm}

deviation($\mathit{p_{1}}$, $\mathit{T_1}$, $\mathit{p_{2}}$, $\mathit{T_2}$):\\
\vspace{-4mm}%
\begin{algorithmic}[1]
\STATE $\mathit{result} \leftarrow$ $|$$\phi$($\mathit{p_{1}}$, $\mathit{T_1}$) - $\phi$($\mathit{p_{2}}$, $\mathit{T_2}$)$|$\\
\COMMENT{$\delta$ is a pre-defined threshold with default value: 0.1}
\IF{$\mathit{result}$ < $\delta$}
\STATE $\mathit{result}$ = 0
\ENDIF
\RETURN $\mathit{result}$
\end{algorithmic}

recommendOptions($\mathit{T_{old}}$, $\mathit{T_{new}}$, $\mathit{V_{old}}$, $\mathit{V_{new}}$, $\mathit{predMap}$)\\
\vspace{-4mm}%
\begin{algorithmic}[1]
\STATE $\mathit{optionMap}$ $\leftarrow$ new Map$\langle$Option, Float$\rangle$\\
\COMMENT{Each entry of $\mathit{optionMap}$ is initialized to 0.}
\FOR{each $\mathit{p_{old}}$ in getPredicates($\mathit{V_{old}}$)}
\STATE $\mathit{v} \leftarrow$ deviation($\mathit{p_{old}}$, $\mathit{T_{old}}$, $\mathit{stmtMap}$[$\mathit{p_{old}}$], $\mathit{T_{new}}$)
\STATE $\mathit{options_{old}}$ $\leftarrow$ getAffectingOptions($\mathit{p_{old}}$, $\mathit{V_{old}}$)
\STATE $\mathit{v}$ $\leftarrow$ $\mathit{v}$ $\times$ getExecutedStmtNum($\mathit{p_{old}}$, $\mathit{V_{old}}$, $\mathit{T_{old}}$)
\FOR{each Option $\mathit{option}$ in $\mathit{options_{old}}$}
\STATE $\mathit{optionMap}$[$\mathit{option}$] $\leftarrow$ $\mathit{optionMap}$[$\mathit{option}$] + $\mathit{v}$
\ENDFOR
\ENDFOR
\FOR{each $\mathit{p_{new}}$ in getPredicates($V_{new}$)}
\STATE $\mathit{v} \leftarrow$ deviation($\mathit{stmtMap^{-1}}$[$\mathit{p_{new}}$], $\mathit{p_{new}}$, $\mathit{T_{old}}$, $\mathit{T_{new}}$)
\STATE $\mathit{options_{new}}$ $\leftarrow$ getAffectingOptions($\mathit{p_{new}}$, $\mathit{V_{new}}$)
\STATE $\mathit{v}$ $\leftarrow$ $\mathit{v}$ $\times$ getExecutedStmtNum($\mathit{p_{new}}$, $\mathit{V_{new}}$, $\mathit{T_{new}}$)
\FOR{each Option $\mathit{option}$ in $\mathit{options_{new}}$}
\STATE $\mathit{optionMap}$[$\mathit{option}$] $\leftarrow$ $\mathit{optionMap}$[$\mathit{option}$] + $\mathit{v}$
\ENDFOR
\ENDFOR
\RETURN $\mathit{optionMap}$.sortedKeys()
\vspace{-2mm}
\end{algorithmic}
\caption{Algorithm for recommending configuration options.
Method deviation is a helper method to compute the deviation
value between two predicates $\mathit{p_1}$ and $\mathit{p_2}$, and function $\phi$
used in deviation is defined in Section~\ref{sec:identify_diff}.
\label{fig:recommend}
}


\end{figure}

\begin{comment}
\todo{Each entry of optionMap needs to be initialized to 0.}
\todo{The algorithm of Figure~\ref{fig:recommend}, together with that of
  Figure~\ref{fig:identify}, seem too complicated to me.  For example,
  pairs are put in predMap, but the correspondence between the elements of
  those pairs is not used.  I think that both figures can be replaced by
  something much shorter (and clearer to me):
}

\begin{Verbatim}
deviation(p1, T1, p2, T2):
  result = | \phi(p, T) - \phi(p_other, T_other) |
  if result < \delta
    result = 0
  return result

recommendOptions(V_old, V_new, stmtMap)
  optionMap <- new Map<Option, Float>
  for p_old in getPredicates(T_old):
    v <- deviation(p_old, T_old, stmtMap[p_old], T_new)
    ... [lines 5-9 of current algorithm]
  for p_new in getPredicates(T_new):
    v <- deviation(stmtMap^{-1}[p_new], T_old, p_new, T_new)
    ... [lines 12-16 of current algorithm]
  return optionMap.sortedKey()
\end{Verbatim}
\end{comment}
%  LocalWords:  getAffectingOptions getExecutedStmtNum recommendOptions
