\begin{abstract}


Modern software often exposes configuration options for users to
customize its behavior. During software evolution,
developers may change how the configuration options behave.
When upgrading to a new software version,
users may need to re-configure the software
by changing values of certain configuration options.

This paper addresses a central question during the evolution
of a configurable software system: which configuration options
should a user change to maintain the software's desired behavior?
We present a technique (and its tool implementation,
called \ourtool) to troubleshoot configuration errors
caused by software evolution. 
\ourtool uses dynamic profiling, execution trace
comparison, and static analysis to link the undesired
behavior to its root cause --- a single configuration option
whose value can be changed to produce desired behavior on the new
software version.

We evaluated \ourtool on \errornum configuration errors
from \subjnum configurable software systems written in Java.
In 6 errors, the root-cause configuration option was
\ourtool's first suggestion. In 1 error, the root cause
was \ourtool's third suggestion. The root cause
of the remaining error was \ourtool's sixth suggestion.
Overall, \ourtool produced significantly better results
than two existing techniques. \ourtool runs in only a few
minutes, making it an attractive alternative to manual debugging.

\smallskip
\smallskip

\noindent
\textbf{Categories and Subject Descriptors}:  
D.2.5 [Software Engineering]: Testing and Debugging.
\\
\textbf{General Terms: }
Reliability, Experimentation.
\\
\textbf{Keywords: }
Configuration, Program analyses, Software evolution.

\end{abstract}

\label{dummy-label-for-etags}
