

As shown in Figure~\ref{tab:results},
\ourtool is highly effective in pinpointing the root cause of
misconfigurations. For all \noncrash non-crashing errors
and 5 out of the \crash crashing errors, it lists the actual root cause as one of the top 3 options. 


\ourtool is particularly effective in diagnosing non-crashing configuration errors,
which are not supported by most existing tools. The primary reason is due to
\ourtool's ability to identify the behaviorally-deviated predicates through
execution profile comparison. The top-ranked deviated predicates often provide
useful clues about what parts of a program might be abnormal and why.

We use the non-crashing error in Weka as an example to illustrate
this point. 
Weka's decision tree implementation is highly-tuned, achieving 70--90\% accuracy on
its included examples. However, its accuracy drops to 62\%
on a different dataset we experimented on. We used \ourtool to diagnose this
problem by first building a database by running Weka on its examples, and
then
obtaining the undesired execution profile by running it on our dataset. As a result,
\ourtool outputs the following report (only the top option is shown):


\begin{CodeOut}
\begin{alltt} 
  Suspicious configuration option: m\_numFolds

  It affects the behavior of predicate:
  "numFold < numInstances() \% numFolds"
  (line 1354, class: weka.core.Instances) 

  This predicate evaluates to true:
    20\% of the time in normal runs (4 observations)
    70\% of the time in the undesired run (10 observations)

\end{alltt}
\end{CodeOut}

\vspace{-3mm}

The above report reveals an important fact for the low accuracy.
The predicate \CodeIn{numFold $<$ numInstances() \% numFolds} controls
the depth of a decision tree. Its
true ratio is substantially higher in the undesired execution
than normal executions. A higher true ratio leads
to a deeper tree that is more likely to overfit the training
data, and yields low accuracy on the testing data.
To resolve this problem, we changed \CodeIn{m\_numFolds}
value from 2 to 3 to reduce the tree depth, and
gained a 5\% performance increase.

%achieved
%Changing the default value of m\_numFolds from 3 to 2 leads to
%an immediate increase of 5\% improvement

Compared to non-crashing errors, \ourtool is less effective
in diagnosing non-crashing errors.
%For 4 crashing errors, the actual causes are ranked below the top 3 options.
This is a direct result of lacking enough predicate behavior observations,
since most crashing errors occur soon
after the program is launched. Therefore, many predicates are
only executed once and their $Deviation$ scores
 (Section~\ref{sec:deviation}) turn out to be the same; and the BFS-distance-heuristic to resolve ties (Section~\ref{sec:linking})
works half the time.


%considers configs closer to the erroneous behavior to be more
%likely to lead to the root cause than those farther away

%$\blacksquare$ the configuration option has
%a long propagation chain, and seems hard for \ourtool
%to diagnose correctly. no statistical significance...

%Although \ourtool ranked the actual root course of several crashing errors lower, 
Crashing errors are generally much easier to diagnose than non-crashing errors.
This is because a crashing error often produces a stack trace with valuable diagnosis clues.
For example, in Figure~\ref{tab:results}, \ourtool ranks the root cause of
error 14  8th.
However, when JChord crashes, it throws a \CodeIn{ClassNotFoundException}
that reminds users to check the classpath setting. For the other three crashing errors (error 8, 10, and 11),
JChord even outputs the wrong configuration option value in the
error message, which
directly guides users to the root cause. 



%$\blacksquare$
%Thus, the root cause gets
%ranked lower in the list. This ordering is a direct result
%of the heuristic discussed in Section XXX that 

%== imprecise of thin slicing, particularly when dealing with containers,





