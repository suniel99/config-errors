\section{Related Work}

Here is a reasonably complete list of recent related work.

\textbf{Configuration error diagnosis approaches.} Existing approaches include
pure slicing-based technique~\cite{Rabkin:2011:PPC}, binary search-based
technique~\cite{Whitaker:2004:CDS}, dynamic information flow-based
technique~\cite{Attariyan:2010:ACT}, template-based technique~\cite{Yuan:2011:COC},
invariant-based technique~\cite{Zhou:2007:DMD}, decision-tree-based approach~\cite{Mickens:2007:SID}, and
causality-based technique~\cite{Attariyan:2008:UCD}.
problem shooting~\cite{Su:2009:AGP, Wang:2004:AMT}. 
This work is significantly different than the existing ones. First,
almost all existing approaches implicitly assume a visible \textit{crashing} error,
while our approach does not have this assumption. In fact, crashing
errors are often much easier to localize than other silent failures.
Second, almost all existing approach assume the existence of an oracle
which can safely identify the potential errors, such as the timing
threshold for a web server performance. However, such error-revealing
oracles are often absent in practice.
Third, some approaches assume the existence of a configuration model, which
helps in narrowing down possible errors.

\textbf{Better configuration management approaches.} Researchers
have designed many techniques to reduce the burden of configuration
management, and prevent possible errors from happening. Typical
approaches include easy configuration management interfaces
for non-professional end-users~\cite{Kushman:2010:ECA},
OS-supported configuration management~\cite{Su:2007:AIC},
automated configuration recommendations~\cite{Zheng:2007:ACI},
learning good configurations from historical data~\cite{Rao:2009:VRL}, and
leveraging community users' knowledge for configuration tuning~\cite{Zheng:2011:MAC}.
The above approaches focus on general configuration management issues, instead
of error localizations.


\textbf{Configuration-aware software analysis and testing.} Ways to understand
configurable software and find configuration errors by testing. Representative work along this
research line includes improving regression testing of configurable software~\cite{Qu:2008:CRT},
using symbolic execution to understand configurable software~\cite{Reisner:2010:USE},
white-box approaches~\cite{whiteboxconf},
finding high-coverage configurations for testing~\cite{itrees}. Those approaches
can serve as alternative ways to find configuration errors.

\textbf{The use of execution spectrum to localize bugs.} There is a rich body of
work in software engineering community in using execution spectrum for error diagnosis.
Typical work includes~\cite{Liblit:2005:SSBI, Santelices:2009:LFU, Reps:1997:UPP,
Yilmaz:2008:TTF}. However, our work has a different abstraction than theirs.

\textbf{Software error isolation techniques.} Many techniques have been developed
in the software engineering community. Representative work include delta debugging~\cite{Zeller:2002:ICC},
dynamic slicing~\cite{Zhang:2006:LFT},
capture and replay-based technique~\cite{Qi:2011:LFE}, approaches permitting
users to ask program behaviors~\cite{Ko:2008:DRA},
clustering-based~\cite{Dickinson:2001:FFC},
error explanation~\cite{Groce:2006:EED},
and approach combining static and dynamic information~\cite{Holmes:2011:IPT, Zhang:2008:EIF}.


\textbf{Other related work.} This category includes inferring configuration options
from source code with static analysis~\cite{Rabkin:2011:SEP}, 
reverse engineering of software configurations~\cite{Wang:2008:TAR}, automatically
fixing certain configuration errors by constraint solving~\cite{rangefix},
empirical studies of configuration errors in practice~\cite{Yin:2011:ESC, Hubaux:2012},
change impact analysis for configuration options~\cite{configimpact},
and using program profiling and statistical analysis to solve
other problems (e.g., critical program region identification)~\cite{Carbin:2010:AIC}.


In contrast, current industrial practice uses stack traces to cluster
failure reports into equivalence classes. Two crash reports showing
the same stack trace, or perphas only the same top-of-stack function,
are presumed to be two reports of the same failure. This heuristic
works to the extent that a single cause corresponds to a single point
of failure, but our experience with xxx suggests that this
assumption may not often hold.
