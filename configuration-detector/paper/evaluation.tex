\section{Evaluation}

\subsection{Research Questions}

We aim to answer the following research questions:

\begin{itemize}
\item Is our technique useful in explaining software misconfiguration errors?
\item Is the information provided by our technique more useful than the statement-level
profiling and the method-level dynamic invariant detection?
\item Which factors (how much) can affect our technique's accuracy?
\end{itemize}

\subsection{Subject Programs}

We collected a number of subject programs and their mis-configuration problems in
Table~\ref{tab:subjects}.

\begin{table}[t]
\setlength{\tabcolsep}{.94\tabcolsep}
\begin{tabular}{|c|c|}
\hline
 program & Description of misconfiguration problem \\
 \hline
 \hline
 Randoop &  No tests generated \\
\hline
 Weka &  Poor performance of the decision tree \\
\hline
 Chord &  No datarace reported \\
\hline
 Synoptic &  Produce an incorrect model \\
\hline
 Soot &  Source code line number is missing \\
\hline
\end{tabular}

\Caption{{\label{tab:subjects} Subject programs and their configuration problems.}}
\end{table}

\subsubsection{Crashing Errors}

\subsubsection{Non-Crashing Errors}

\subsubsection{Injected Errors}

Use injected errors to test the technique's reliability

\subsection{Experiment Design}



\subsubsection{Accuracy in Localizing Configuration Options}


Are the ranked configurations useful for misconfiguration error diagnosis?

\subsubsection{Sensitivy to the Inputs}

What would the technique produce when feeding it with different inputs (e.g.,
radically different inputs instead of similar ones)?

\subsubsection{Comparison with Using Classic Slicing}

How would the results change if traditional slicing~\cite{Horwitz:1988} is used
in Section~\ref{sec:prop}?

\subsubsection{Comparison with Statement-level Profiling and Method-level Invariant Detection}

Our technique is at the \textit{predicate}-level. What about using
\textit{statement}-level instrumentation and \textit{method}-level dynamic invariant detection~\cite{Ernst:1999}?

Is our \textit{predicate}-level granularity a suitable one?

\subsubsection{The Effects of Increasing Context Sensitivity}

When recording configuration profiles (Section~\ref{sec:profiling}), is it useful
to increase the length of calling context? Would that help improve our technique's accuracy?
