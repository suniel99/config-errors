\section{Evaluation}
\label{sec:evaluation}

%\subsection{Research Questions}

Our evaluation answers the following research questions:

%\todo{I would organize the research questions as:
%\begin{itemize}
%\item
%How effective is \ourtool?
%\begin{itemize}
%\item
%in absolute terms
%\item
%compared to other tools
%\end{itemize}
%This can include run time
%\item
%Discussion of internal implementation choices of \ourtool
%\end{itemize}
%}

%\todo{Give cross-references to sections that address these.}

\begin{itemize}
\item How effective is \ourtool in error diagnosis? \ourtool's effectiveness can be reflected by:
\begin{itemize}
  \item the absolute ranking of the actual root cause in \ourtool's output (Section~\ref{sec:accuracy}).
  \item the time cost in error diagnosis (Section~\ref{sec:performance}).
  \item comparison with existing configuration error diagnosis techniques (Section~\ref{sec:confanalyzer}).
  \item comparison with fault-localization-based configuration error diagnosis techniques (Section~\ref{sec:comparison}).
\end{itemize}
\item What are the effects of using full slicing rather than thin slicing to identify
the affected predicates, and the effects of varying comparison execution profiles (Section~\ref{sec:choices})?
These are two internal design choices. % in \ourtool.
\end{itemize}

%\begin{itemize}
%\item How effective is \ourtool in diagnosing the root cause of
%a configuration error (Section~\ref{sec:accuracy})?
%\item Can \ourtool provide more accurate diagnosis information than
%other approaches (Section~\ref{sec:comparison})? 
%\item How long does \ourtool take to diagnose a configuration error (Section~\ref{sec:performance})?
%\item What are the effects of varying the internal implementation of \ourtool,
%such as using a different configuration propagation analysis and
%different comparison execution profiles (Section~\ref{sec:choices})?
%\item What are the effects of varying the comparison execution profiles (Section~\ref{sec:choices})?
%\end{itemize}

%We use two metrics to evaluate \ourtool's effectiveness:
%the absolute ranking of the actual root cause in \ourtool's output,
%and the time cost used in diagnosis.

\subsection{Subject Programs}


We evaluated \ourtool on \subjectnum Java programs shown
in Figure~\ref{tab:subjects}.
Randoop~\cite{PachecoLET2007} is an automated test generator
for Java programs. Weka~\cite{weka} is a toolkit that implements
machine learning algorithms. Our evaluation
only uses its decision tree module. JChord~\cite{jchord}
is a program analysis platform that enables users to design, implement,
and evaluate static and dynamic program analyses for Java.
Synoptic~\cite{Beschastnikh:2011} mines a finite state machine
model representations of a system from logs.
Soot~\cite{soot} is a Java optimization framework for analyzing and transforming Java bytecode.


\subsubsection{Configuration Errors}

\begin{figure}[t]
\centering
\small{
\setlength{\tabcolsep}{.64\tabcolsep}
\begin{tabular}{|l|c|c|c|}
\hline
 Program (version) & LOC & \#Config Options & \#Profiles\\
 \hline
 \hline
 Randoop (1.3.2) & 18587 & 57 & 12\\
 Weka (3.6.7) & 256305 & 14 & 12\\
 JChord (2.1) & 23391 &  79 & 6 \\
 Synoptic (trunk, 04/17/2012) & 19153 & 37 & 6\\
 Soot (2.5.0) & 159273 & 49 & 16 \\
\hline
\end{tabular}
}

%\todo{Give a date for Synoptic}
\Caption{{\label{tab:subjects} Subject programs. 
Column ``LOC'' is the number of lines of code,
as counted by CLOC~\cite{cloc}. Column ``\#Config Options''
is the number of available configuration options. Column ``\#Profiles''
is the number of execution profiles in the pre-built database.}}
\end{figure}


\begin{figure}[t]
\setlength{\tabcolsep}{.94\tabcolsep}
\small{
\begin{tabular}{|l|l|l|}
\hline
 Error ID & Program & Description \\
 \hline
\hline
\multicolumn{3}{|l|}{Non-crashing errors}   \\
 \hline
 1 & \randoop & No tests generated\\
 2 & \weka & Low accuracy of the decision tree\\
 3 & \jchord & No datarace reported for a racy program\\
 4 & \synoptic & Generate an incorrect model\\
 5 & \soot & Source code line number is missing\\
\hline
\hline
\multicolumn{3}{|l|}{Crashing errors}   \\
\hline
 6 & \jchord & No main class is specified\\
 7 & \jchord& No main method in the specified class\\
 8 & \jchord & Running a nonexistent analysis\\
 9 & \jchord & Invalid context-sensitive analysis name\\
 10 & \jchord & Printing nonexistent relations\\
 11 & \jchord & Disassembling nonexistent classes\\
 12 & \jchord & Invalid scope kind\\
 13 & \jchord & Invalid reflection kind\\
 14 & \jchord & Wrong classpath\\
\hline
\end{tabular}
}
\Caption{{\label{tab:errors} A list of \errors
configuration errors used in the evaluation.
%The 9 crashing errors in the bottom table are taken from~\cite{Rabkin:2011:PPC}.
}}
\end{figure}

We searched forums, FAQ pages and the literature of
configuration error diagnosis research to find actual
configuration problems that users have experienced with our
target applications. 
We chose \errors configuration errors, in which
the misconfigured values cover various data types, such as enumerated types,
numerical ranges, regular expressions, and text entries;
as listed in Figure~\ref{tab:errors}. The \noncrash non-crashing errors
are collected from actual bug reports, mailing list posts, and our own experience.
The \crash crashing errors, taken from~\cite{Rabkin:2011:PPC},
were used to evaluate the ConfAnalyzer tool.
All \errors configuration errors have been minimized: if
any part of the configuration or input is removed, the software
either crashes or no longer exhibits the undesired behavior.
%\todo{Were any of them used in previous research?  Was that the reason we
%  chose them?}

\begin{figure*}[t]
\vspace{1mm}
\setlength{\tabcolsep}{.22\tabcolsep}
\small{
\begin{tabular}{|l|c|c||c|c||c||c|c||c|}
\hline
 Error ID.  & Root Cause & \#Options& \multicolumn{2}{|c||}{\ourtool} & ConfAnalyzer~\cite{Rabkin:2011:PPC}& Coverage Analysis& Invariant Analysis & Full Slicing \\
\cline{4-9}
 Program &  & & \#Profiles& Rank  & Rank & Rank & Rank & Rank \\
 \hline
\hline
\multicolumn{9}{|l|}{Non-crashing errors}   \\
 \hline
\phz 1. Randoop& \CodeIn{maxsize} & 57& 10 / 12 & 1 & X & 13 / 13 & N / N &46\\
\phz 2. Weka&\CodeIn{m\_numFolds}& 14 &2 / 12 &1&  X& 4 / 7 & 5 / 5 &9\\
\phz 3. JChord& \CodeIn{eqth}& 79 & 2 / 6 & 3& X & 38 / 31 &2 / 2  &73\\
\phz 4. Synoptic& \CodeIn{partitionRegExp}& 37 & 2 / 6 & 1&  X& 1 / 1 & N / N &6\\
\phz 5. Soot& \CodeIn{keep\_line\_number} &49 & 6 / 16 & 2 & X & 46 / 18 & N / N &N\\
\hline
 \multicolumn{2}{|l|}{Average} & 47.2 & 3.6 / 10.4 & 1.6 & 23.6 & 20.4 / 14.0 & 15.7 / 15.7 & 31.7 \\
\hline
\hline
\multicolumn{9}{|l|}{Crashing errors}   \\
\hline
\phz 6. JChord& \CodeIn{chord.main.class}&79 &4 / 6 & 1& 1 & 1 / 1 & 4 / 4 & 5\\
\phz 7. JChord& \CodeIn{chord.main.class}& 79 &5 / 6 & 1 &  1& 1 / 1 & 4 / 4 & 5\\
\phz 8. JChord& \CodeIn{chord.run.analyses}& 79 &5 / 6 & 17& 1 &17 / 14 & 22 / 17 & 21\\
\phz 9. JChord& \CodeIn{chord.ctxt.kind}& 79 &3 / 6 & 1 &  3& 25 / 27 & 30 / 30 & 75\\
 10. JChord& \CodeIn{chord.print.rels}& 79 & 2 / 6& 15 & 1 & 20 / 16 & 25 / 19 & 24\\
 11. JChord& \CodeIn{chord.print.classes}& 79 &4 / 6 & 16 & 1 & 13 / 15 & 17 / 18 & 22\\
 12. JChord& \CodeIn{chord.scope.kind}& 79 &5 / 6 & 1&  1& 1 / 1 & N / N& 10\\
 13. JChord& \CodeIn{chord.reflect.kind}& 79 &6 / 6 & 1& 3 & 5 / 6 & 9 / 9 & 11\\
 14. JChord& \CodeIn{chord.class.path}& 79 &4 / 6 & 8 &  N& 21 / 2 & 26 / 5 & 6\\
\hline
 \multicolumn{2}{|l|}{Average} & 79 & 4.2 / 6 & 6.7 & 5.7 & 11.5 / 9.2 & 19.5 / 16.1 & 19.8\\
\hline
\end{tabular}
}
\vspace{-3mm}
%\todo{Perhaps add a new column, after ``root cause configuration option'', giving
%  the total number of configuration options in the program.  This will
%  emphasize how good the ``rank'' numbers are.}
\Caption{{\label{tab:results} Experimental results in diagnosing software
configuration errors. Column ``Root Cause'' shows the actual
root cause configuration option. Column ``\#Options'' shows the
number of available configuration options, taken from Figure~\ref{tab:subjects}.
Column ``\ourtool'' shows the results of using our technique. 
Column ``\#Profiles'' shows the number of similar execution profiles selected
from the pre-built database for comparison, and the total size of the database.
Column ``ConfAnalyzer'' shows the results of using an existing technique~\cite{Rabkin:2011:PPC} (Section~\ref{sec:confanalyzer});
and the data in this column is taken from~\cite{Rabkin:2011:PPC}.
\todo{I added the above sentence to explain where the data comes from}
Columns ``Coverage Analysis'' and ``Invariant Analysis'' show the results of using
two fault localization techniques as described in Section~\ref{sec:comparison}.
\todo{I added the following sentences}
For these two columns, a slash ``/'' separates the results of using
the selected similar execution profiles by \ourtool (Section~\ref{sec:similar}) and the results of using all execution profiles from
the pre-built database.
Column ``Full Slicing'' shows the results of using full slicing~\cite{Horwitz:1988} to compute
the affected predicates (Section~\ref{sec:choices}).
For each technique, Column ``Rank'' shows the absolute rank of the actual root 
cause in its output (lower is better). ``X''
means the technique is not applicable (i.e., requiring a crashing point), and ``N'' means the technique
does not identify the actual root cause. When computing the average
rank, each ``X'' or ``N'' is treated as half of the number of configuration options, because
a user would need to examine on average half of the options to find the root cause.
}}
\end{figure*}

%daikon can not work on synoptic, use N to denote this


\subsection{Evaluation Procedure}

For each subject program, we constructed a profile database
by running existing (correct) examples from its user manual, discussion
mailing list, and published papers~\cite{PachecoLET2007, Beschastnikh:2011, Rabkin:2011:PPC}.
We spent 3 hours per program, on average, and obtained 6--16 execution profiles.
The average size of the profile database is 35MB, and the largest database (Randoop's)
is 72MB.

We made a simple syntactic change to JChord, which affected 24 
lines of code. This change
does not modify JChord's semantics; rather, it just encapsulates
scattered configuration option initialization statements 
as static class fields. \todo{This sentence needs a rewrite:}This is purely an implementation
choice because having a centralized initialization statement
makes our tool implementation easier to specify the seed statement
in performing slicing. Here is a sample modification, where 
\<chord.kobj.k> 
is a configuration option
passed as a system property:


\begin{CodeOut}
\begin{alltt}
public void run() \ttlcb
  ...
  int kobjK = Integer.getInteger("chord.kobj.k");
  ...
\ttrcb
\end{alltt}
\end{CodeOut}
\vspace{-4mm}
\hspace{20mm}$\Downarrow$ 
%\vspace{-2mm}
\begin{CodeOut}
\begin{alltt}
static int chord\_kobj\_k = Integer.getInteger("chord.kobj.k");
public void run() \ttlcb
  ...
  int kobjK = chord\_kobj\_k; 
  ...
\ttrcb
\end{alltt}
\end{CodeOut}


When diagnosing a configuration error, we first reproduce the
error on a \ourtool-instrumented program to obtain the
execution profile. Then, using the obtained execution profile, we use \ourtool
to identify its root causes.

%since we know the misconfigured, root-cause entry for each case,
%we use the ranking of the entry as our evaluation metric.

Our experiments were run on a
2.67GHz Intel Core PC with 4GB physical memory (2GB was allocated
for the JVM), running Windows 7.


\subsection{Results}
\label{sec:results}


\subsubsection{Accuracy in Diagnosing Configuration Errors}
\label{sec:accuracy}


Figure~\ref{tab:results} shows the experimental results.
We can see that \ourtool is highly effective in pinpointing the root cause of
misconfigurations. For all \noncrash non-crashing errors
and 5 out of the \crash crashing errors, it lists the actual root cause as one of the top 3 options. 


As shown in Figure~\ref{tab:results}, \ourtool is particularly effective
in diagnosing non-crashing configuration errors, which are not supported
by most existing tools. $\blacksquare$ why effective? observation?
statically significance? Give some examples here... Randoop \CodeIn{maxsize},
Weka behaves overfitting..

\noindent \textbf{Example.} \ourtool outputs the following error report
for Weka:

\begin{CodeOut}
\begin{alltt} 
Suspicious configuration option: m\_numFolds

It affects the behavior of predicate:
"numFold < numInstances() \% numFolds"
(line 1354, class: weka.core.Instances) 

This predicate evaluates to true:
  0\% of the time in normal runs (10 observations)
  70\% of the time in the undesired run (10 observations)

\end{alltt}
\end{CodeOut}


Compared to non-crashing errors, \ourtool is less effective
in diagnosing non-crashing errors. For 4 crashing errors,
the actual causes are ranked lower.
This is because $\blacksquare$ the configuration option has
a long propagation chain, and seems hard for \ourtool
to diagnose correctly. no statistical significance...

Although \ourtool ranked the actual root course of several
crashing errors lower, crashing errors are generally much easier to diagnose than non-crashing errors.
This is because a crashing error usually happens shortly after the program
is launched, and often produces a stack trace with valuable diagnosis clues.
For example, in Figure~\ref{tab:results}, \ourtool ranks the root cause of
error 14  8th.
However, when JChord crashes, it throws a \CodeIn{ClassNotFoundException}
that reminds users to check the classpath setting. For the other three crashing errors (error 8, 10, and 11),
JChord even outputs the wrong configuration option value in the
error message, which
directly guides users to the root cause. We speculated that $\blacksquare$

%The nature of the root-cause configuration option is only one factor.
%The ranking also depends on how the root-cause option relates to
%other options in the suspect set. A highly configurable software system 
%likely produces more noises, 


%The remaining errors are a direct result of $\blacksquare$ and seems
%hard for \ourtool to diagnose correctly.


%\ourtool also successfully diagnoses xx\% of the xxx errors. For the
%remaining errors, \ourtool ranks the root cause 9th. The configuration
%error is that the xxx. Thus, the root cause gets ranked lower
%in the list.



\subsubsection{Performance of \ourtool}
\label{sec:performance}
We measured \ourtool's performance in two ways: the time cost
in diagnosing an error and the overhead introduced
in reproducing an error in a \ourtool-instrumented program.  Figure~\ref{tab:performance}
shows the results.

\begin{figure}[t]
\setlength{\tabcolsep}{.94\tabcolsep}
\small{
\begin{tabular}{|l|c|c|c|}
\hline
 Error ID. & \multicolumn{2}{|c|}{Time (seconds)} & Slowdown ($\times$)\\
  %& \multicolumn{3}{|c|}{Different Comparison Profile Selection Strategy} \\
\cline{2-3}
 Program & Thin Slicing & Error Diagnosis &  \\
 \hline
\hline
\multicolumn{4}{|l|}{Non-crashing errors}   \\
 \hline
 1. Randoop & 50 & $<$ 1 & 1.1\\
 2. Weka & 43 & $<$ 1 & 1.2 \\
 3. JChord & 147 & 82 & 13.2\\
 4. Synoptic & 24 & $<$ 1 & 3.6 \\
 5. Soot & 95 & 21 & 3.1 \\
\hline
Average & 72 & 21 & 4.4\\
\hline
\hline
\multicolumn{4}{|l|}{Crashing errors}   \\
\hline
 6. JChord & 147 & 79 & 2.4\\
 7. JChord & 147 & 75 & 1.4\\
 8. JChord & 147 & 17 &1.5\\
 9. JChord & 147 & 30 & 28.5\\
 10. JChord & 147 & 13 &13.7\\
 11. JChord & 147 & 10 &65.1 \\
 12. JChord & 147 & 83 &1.6\\
 13. JChord & 147 & 8 &1.9\\
 14. JChord & 147 & 80 &1.4\\
\hline
Average & 147 & 44 & 13\\
\hline
\end{tabular}
}
\Caption{{\label{tab:performance} \ourtool's
performance in diagnosing configuration
errors. The time cost has been divided into
two parts: computing thin slices and diagnosing
an error.}}
\end{figure}

The performance of \ourtool is reasonable.
On average, it uses \avgtime minutes to
diagnose one configuration error. Computing
thin slices for all configuration options
is expensive. However, this step is one-time effort
per program and the computed slices can be cached
to share across diagnoses. %for future use.

The performance overhead to reproduce the buggy behavior varies
among applications. The current tool implementation
imposes an average slowdown of 13 times when reproducing
an error in a \ourtool-instrumented version.
This performance overhead, admittedly high, is still acceptable
for offline error diagnosis.
For errors 3, 9, 10, and 11, the slowdown is signficiant, but the
absolute time cost for error reproduction is still low ( $<$ 13 minutes per error).  
For the other 10 errors, the average slowdown is only 1.9.

%\todo{Maybe add one more sentence here: for error 9, 10, 11, the slowdown for
%reproducing a crashing error is very large. However, the absolute
%time cost is very low, on the original JChord version, all these three errors exhibit in less than 5 seconds. Thus, even with a 65X slowdown, the absolute time cost
%is still acceptable (around 5 mins).}

%The size of profile files in the database $\blacksquare$


%The performance of \ourtool is reasonable. The time to diagnose
%an error varies among applications.  XXX app takes less than xxx,
%while xxx takes xxx to complete.



\subsubsection{Comparison with an Existing Technique}
\label{sec:confanalyzer}
We compared \ourtool with ConfAnalyzer, a heavyweight dynamic information
flow-based technique~\cite{Rabkin:2011:PPC}.
We chose ConfAnalyzer because it is one of the most precise configuration
error diagnosis techniques in the literature and its 
experimental data is publicly available.
ConfAnalyzer computes possible configuration diagnosis for Java software,
and works remarkably well for crashing errors, though as
described above these are often easy to diagnose even without tool
support. However, ConfAnalyzer cannot diagnose non-crashing errors.

The experimental results of ConfAnalyzer is shown in Figure~\ref{tab:results} (Column ``ConfAnalyzer'').
$\blacksquare$

%Rabkin and Katz proposed a family of techniques to precompute possible
%configuration diagnosis for Java software~\cite{Rabkin:2011:PPC}. In their work,
%the most accurate technique (also probably one of the most precise techniques in the literature)
%is based on dynamic information flow analysis.

%\todo{Is ConfAnalyzer heavyweight?  If so, say so.}




\subsubsection{Comparison with Two Fault-Localization-Based Approaches}
\label{sec:comparison}

%\todo{Are variants 2 and three really variants of \ourtool, or are they
%  completely different techniques that should be presented as such?  In
%  particular, can you characterize variant 2 (here and in the table) as
%  Tarantula, possibly with some small enhancements.  I have a similar
%  question about variant 3.  In any event, make clearer what part of the
%  architecture is replaced by each variant.}

Another possible way to diagnose a configuration error is to leverage
the existing fault localization techniques, by treating the undesired
execution as a failing run and all correct executions (in the database)
as passing runs. We next compare \ourtool with two state-of-the-art
techniques: % in error diagnosis:

\begin{itemize}
\item \textbf{Statement-level Coverage Analysis}. This technique treats statements covered
by the undesired execution profile as potentially buggy, and statements
covered the correct execution profiles as correct.
Then, it leverages a well-known fault localization technique,
Tarantula~\cite{Jones:2002}, to rank the likelihood of each
statement being buggy, and queries the results of thin slice
to identify its affecting configuration options as the root causes.

\item \textbf{Method-level Invariant Analysis}. This technique stores invariants detected
by Daikon~\cite{Ernst:1999} from correct execution profiles in the database.
When a configuration error occurs, this technique detects invariants from the undesired execution profile;
and compares them with those stored in the database.
It treats a method to have suspicious behaviors if its observed invariants
from the undesired execution profile are different from the invariants stored
in the database~\cite{McCamant:2003}. Finally, this technique ranks a method's suspiciousness by
the number of different invariants, and queries the results of thin slice
to identify its affecting configuration options as the root causes. 
\end{itemize}


The experimental results are shown in Figure~\ref{tab:results} (Columns
``Coverage Analysis'' and ``Invariant Analysis'').
Both techniques produce less
accurate results; and for some errors, even fail to identify
the actual root causes.

In Coverage Analysis, the statement-level abstraction is \textit{too fine-grained}.
Many statements have the exactly same coverage by the failing/passing executions,
and thus have the same suspiciousness score as computed by Tarantula~\cite{Jones:2002}.
Furthermore, the underlying Tarantula technique only records whether a
statement has been executed or not but does not record how a statement is 
executed (e.g, how often a predicate is evaluated to true). The combination
of these two factors causes the low accuracy.


In Invariant Analysis, the method-level abstraction is \textit{too coarse-grained}.
A typical invariant detection technique like Daikon~\cite{Ernst:1999}
only checks program states at method entries and exits, resulting in likely invariants that
correspond to pre- and post-conditions. Thus, invariant detection is less
senstivie to the local control flow changes within a method (e.g., a predicate's
true ratio). In our study, Invariant Analysis failed to diagnose several errors because it
reported the same invariants for the method containing behavioral-deviated predicates.



%method-level granularity is too coarse
%is not sensitive enough on small control flow changes. for example, the invariant is the same


This experiment suggests that applying existing fault localization
techniques to diagnose configuration errors may be insufficient, due
to different abstractions and algorithms. It also indicates that
focusing on the relevant predicates' behaviors as \ourtool does
 can be a good choice.




%\subsubsection{Effects of Varying Comparison Execution Profiles}
\subsubsection{Evaluation of Two Design Choices in \ourtool}
\label{sec:choices}
\begin{figure}[t]
\setlength{\tabcolsep}{.74\tabcolsep}
\small{
\begin{tabular}{|l|c|c||c|}
\hline
 Error ID. & \multicolumn{3}{|c|}{Rank of the Actual Root Cause} \\
  %& \multicolumn{3}{|c|}{Different Comparison Profile Selection Strategy} \\
\cline{2-4}
 Program & All Profiles& Random Selection&  Similarity-Based\\
 \hline
\hline
\multicolumn{4}{|l|}{Non-crashing errors}   \\
 \hline
 1. Randoop & 1 & 2 & 1\\
 2. Weka & 7 & 6 & 1\\
 3. JChord & 16 & 19 & 3\\
 4. Synoptic & 1 & 1 & 1\\
 5. Soot & 13 & 13 & 2\\
\hline
Average & 7.6 & 8.2 & 1.6 \\
\hline
\hline
\multicolumn{4}{|l|}{Crashing errors}   \\
\hline
 6. JChord & 1 & 1 &1\\
 7. JChord & 1 & 1 &1\\
 8. JChord & 17 & 17 &17\\
 9. JChord & 1 &  1&1\\
 10. JChord & 15 & 15 &15\\
 11. JChord & 16 & 16 &16\\
 12. JChord & 1 & 1 &1\\
 13. JChord & 25 & 25 &1\\
 14. JChord & 9 & 9 &9\\
\hline
Average & 9.4 & 9.4 & 6.7\\
\hline
\end{tabular}
}
\Caption{{\label{tab:selection} Comparison with different execution profile selection
strategies (Section~\ref{sec:choices}).
The last column ``Similarity-based'' is the selection strategy
used in \ourtool, and the data in that column is taken from Figure~\ref{tab:results}.}}
\end{figure}

We investigate the effects of:

\begin{itemize}
\item using traditional full slicing~\cite{Horwitz:1988} rather
than thin slicing~\cite{Sridharan:2007} in the Configuration
Propagation Analysis step (Section~\ref{sec:prop}) to compute the affected predicates.
Figure~\ref{tab:results}  (Column ``Full Slicing'') shows the results.
\item varying the comparison execution profiles from the pre-built database.
In particular, we compare the similarity-based selection strategy used in \ourtool
 (Section~\ref{sec:similar}) with two alternatives: selecting
all available profiles in the database, and
randomly selecting the same number of profiles as \ourtool uses from the database.
Figure~\ref{tab:selection} shows the results.
\end{itemize}


As shown in Figure~\ref{tab:results}  (Column ``Full Slicing''),
\ourtool achieves significantly less accurate diagnosis when using
full slicing. The primary reason is that full slicing includes
too many irrelevant statements that are only \textit{indirectly} affected by
a configuration option value but not pertinent to the task of
error diagnosis. In many cases, monitoring the control flow
of such indirectly-affected predicates and then linking their
behaviors to configuration options lead to low accuracy. Furthermore,
performing full slicing is much more expensive than thin slicing; in
our experiments, the standard full slicing algorithm ran out of
memory for Soot.

%Furthermore, modern programs typically rely heavily on well-tested data structures
%provided by standard libraries, whose internal details rarely
%the full slice presents far too much information for the task at hand.


As shown in Figure~\ref{tab:selection}, varying the comparison
strategy yields substantially different results,
depending on the application being analyzed.
%\todo{following sentence needs a rewrite}
Using all available execution profiles or randomly
selecting execution profiles is ineffective. This is because
such strategies cause \ourtool to report many irrelevant
differences between an undesired
execution and a dramatically different execution.
%, which will not be selected by \ourtool's similary-based selection strategy (Section~\ref{sec:similar}).
We also find diagnosing a crashing error is less sensitive 
to the comparison execution profiles than diagnosing a non-crashing error.
This is because crashing profiles are often much smaller, executing
fewer predicates before reaching the crashing points; and \ourtool
chops each correct execution profile before diagnosis (Section~\ref{sec:similar}).
Thus, many irrelevant differences have already been removed.

 %Comparing
%a crashing profile with a chopped correct execution profile (even
%by random selection) 



%in selecting similar
%comparison execution profiles$\blacksquare$ than diagnosing
%a crashing error.
%For the \crash crashing errors, the only difference yielded
%from using different profile selection strategies is on
%the error 13. $\blacksquare$



\subsection{Experimental Discussions}


\noindent \textbf{\textit{Limitations.}} 
We conclude three limitations of our technique from the experiments. %is limited in three aspects.
First, we only focus on named configuration options
with a common key-value semantic, and our implementation
and experiments are restricted to Java. 
Second,  our implementation currently does not
support debugging non-deterministic errors. 
For non-deterministic errors, \ourtool could potentially leverage 
a deterministic replay system
that can capture an undesired non-deterministic
execution and faithfully reproduce it for late analysis.
Third, \ourtool's effectiveness  largely
depends on the availability of a similar but correct execution profile.
Using an arbitrary execution profile (as we demonstrated in Section~\ref{sec:choices}
by random selection) may significantly affect the results.

\vspace{1mm}

\noindent \textbf{\textit{Threats to Validity.}} 
There are two major threats to validity in our evaluation. 
First, the \subjectnum programs and the configuration errors may not be
representative. Thus, we can not claim the results can be
extended to an arbitrary program.
Another threat is that we only employed two dependence
analyses (thin slicing and full slicing) and three
abstraction granularities (at the predicate level,
statement level, and method level) in our evaluation.
 Using other dependence analyses or abstraction levels
might achieve different results.

%how easy to construct such database in practice.
%User study of usefulness of the results


\vspace{1mm}

\noindent \textbf{\textit{Experimental Conclusions.}} 
We have three chief findings: (1) \ourtool is effective
in diagnosing both crashing and non-crashing configuration errors
with a small profile database.
(2) \ourtool produces more accurate diagnosis than
approaches leveraging existing fault localization
techniques~\cite{Jones:2002, McCamant:2003}, 
suggesting the necessity of designing new configuration error
diagnosis techniques. And (3) Using thin slicing
%to identify the affected predicates
permits \ourtool to produce more accurate diagnosis than using
full slicing; and varying the execution profile selection
strategy can result in substantially different diagnosis.

%\ourtool makes configuration error diagnosis easier by suggesting
%the specific options that may lead to an unexpected behavior. 




%Compared to
%alternative approaches, \ourtool distinguishes itself by being able to
%diagnose both crashing and non-crashing errors without requiring
%a user-provided testing oracle. $\blacksquare$
