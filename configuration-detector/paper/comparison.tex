
%\todo{Are variants 2 and three really variants of \ourtool, or are they
%  completely different techniques that should be presented as such?  In
%  particular, can you characterize variant 2 (here and in the table) as
%  Tarantula, possibly with some small enhancements.  I have a similar
%  question about variant 3.  In any event, make clearer what part of the
%  architecture is replaced by each variant.}

Another possible way to diagnose a configuration error is to leverage
the existing fault localization techniques, by treating the undesired
execution as a failing run and all correct executions (in the database)
as passing runs. We next compare \ourtool with two state-of-the-art
techniques: % in error diagnosis:

\begin{itemize}
\item \textbf{Statement-level Coverage Analysis}. This technique treats statements covered
by the undesired execution profile as potentially buggy, and statements
covered the correct execution profiles as correct.
Then, it leverages a well-known fault localization technique,
Tarantula~\cite{Jones:2002}, to rank the likelihood of each
statement being buggy, and queries the results of thin slice
to identify its affecting configuration options as the root causes.

\item \textbf{Method-level Invariant Analysis}. This technique stores invariants detected
by Daikon~\cite{Ernst:1999} from correct execution profiles in the database.
When a configuration error occurs, this technique detects invariants from the undesired execution profile;
and compares them with those stored in the database.
It treats a method to have suspicious behaviors if its observed invariants
from the undesired execution profile are different from the invariants stored
in the database~\cite{McCamant:2003}. Finally, this technique ranks a method's suspiciousness by
the number of different invariants, and queries the results of thin slice
to identify its affecting configuration options as the root causes. 
\end{itemize}


The experimental results are shown in Figure~\ref{tab:results} (Columns
``Coverage Analysis'' and ``Invariant Analysis'').
Both techniques produce less
accurate results; and for some errors, even fail to identify
the actual root causes.

In Coverage Analysis, the statement-level granularity is \textit{too fine-grained}.
Many statements have exactly the same coverage by the failing/passing executions,
and thus have the same suspiciousness score as computed by Tarantula~\cite{Jones:2002}.
Furthermore, Tarantula only records whether a
statement has been executed or not but does not record how a statement is 
executed (e.g., how often a predicate is evaluated to true). The combination
of these two factors causes the low accuracy.


In Invariant Analysis, the method-level granularity is \textit{too coarse-grained}.
Invariant detection techniques like Daikon~\cite{Ernst:1999}
only check program states at method entries and exits to infer
likely pre- and post-conditions, and thus are less
senstivie to many control flow details within a method (e.g., a predicate's
true ratio). In our study, Invariant Analysis failed to diagnose 3
errors because it failed to infer invariants for 1 program (Synoptic),
and reported the same invariants for the other two programs (Soot and Randoop)
on the method containing behavioral-deviated predicates between an undesired execution and correct executions.



%method-level granularity is too coarse
%is not sensitive enough on small control flow changes. for example, the invariant is the same


This experiment suggests that applying existing fault localization
techniques~\cite{Jones:2002, McCamant:2003} to configuration error diagnosis may be insufficient, due
to different abstractions and algorithms. It also indicates that
focusing on the behaviros of relevant predicates as our tool does
 can be a good choice.
