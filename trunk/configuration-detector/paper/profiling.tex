This step instruments the tested program offline
by inserting code to monitor each affected predicate's outcome
at runtime.

The instrumentation is cheap, and incurs acceptable runtime
overhead (as demonstrated in our experiments). The instrumentation step
simply inserts 2 statements
before and after each affected program predicate. Take
the code in Figure~\ref{fig:example} as an example.
In Figure~\ref{fig:example}, the predicate at line 312 is affected by
the configuration option \CodeIn{maxsize}. Thus, \ourtool inserts
two instrumentation statements before and after line 312 to count the number of
the predicate being executed and the number being evaluated to true,
as follows.


\begin{CodeOut}
\begin{alltt}
310. private ExecutableSequence createNewUniqueSequence() \ttlcb
311.   Sequence newSequence = ...; 
       \underline{incrExecCount("maxsize", "newSequence.size() > maxsize");}
312.   if (newSequence.size() > maxsize) \ttlcb
         \underline{incrTrueCount("maxsize", "newSequence.size() > maxsize");}
314.     return null;
      ...
319. \ttrcb
\end{alltt}
\end{CodeOut}


$\blacksquare$The instrumented program expresses an execution
as a trace comprising a vector of \textit{predicate profiles}.
Each predicate profile is a pair of configuration
option and its affected predicate.
The predicate profile also keeps the recorded runtime information.
%of the affected predicate, including
%the count of the predicate being executed and the count of it
%being evaluated to true. 

%\ourtool expresses an execution as
%a predicate profile vector. 
As we
show in the experiments (Section~\ref{sec:evaluation}), such predicate profile
vectors, although by no mean complete, capture
sufficient information to permit users
to reason about the causal effects of configurations
and how they relate to a software's behavior, while
also imposing a moderate amount of performance impact
on foreground applications.


%health as the results of executing a set of predicates.

%As we will show in the experiments~\cite{}, this profile
%provides valuable information about the
%program execution and can help validate a test suite
%or indicate the usage context of a function
%or other computation.

%A software user seeking on specific piece of
%information or aiming to verify a specific invariant
%and uninterested in any other facts about the code
%may be able to use xxx to advantage, but will not
%get as much from it as a programmer open to other,
%possibly valuable information.

