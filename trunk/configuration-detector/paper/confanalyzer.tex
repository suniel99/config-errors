We compared \ourtool with ConfAnalyzer, a dynamic information
flow-based technique~\cite{Rabkin:2011:PPC}.
We chose ConfAnalyzer because it is the most recent technique and
also one of the most precise configuration error diagnosis techniques
in the literature.
ConfAnalyzer tracks the flow of labeled objects through the
program dynamically,
and treats a configuration option as a root cause if its
value may flow to a crashing point.
ConfAnalyzer works remarkably well for most of the crashing errors (all of
which are from the ConfAnalyzer paper~\cite{Rabkin:2011:PPC}), though as
described above these are often easy to diagnose even without tool
support. However, ConfAnalyzer cannot diagnose non-crashing errors.

The experimental results of ConfAnalyzer are shown in
Figure~\ref{tab:results} (column ``ConfAnalyzer'').
For the 9 crashing errors, \ourtool produced better results for 3 of them,
the same results for another 3, and worse results for the remaining 3.

ConfAnalyzer performs better on crashing errors
having shorter execution paths, when an error exhibits
almost immediately after the software is launched.
In such cases, only a small number of configuration options are initialized and
few of them can flow to the crashing point. 
\ourtool fails to produce good diagnosis for these errors, because it can not identify
 the statistically-behaviorally-deviated predicates based on the limited
observation of program behaviors.

ConfAnalyzer outputs less accurate or no results
for errors where the root cause option value
flows into containers or system calls (e.g., error 14 in Figure~\ref{tab:results}).
\ourtool can reason (to some extent) about the \textit{consequence} of
such a misconfiguration based on the observed predicate behaviors.

%%  LocalWords:  ConfAnalyzer
