We next compare \ourtool with three variants and
one existing technique~\cite{Rabkin:2011:PPC}.

\vspace{1mm}
\noindent \textbf{Variant 1. \ourtool with Full Slicing.} 
\ourtool uses thin slicing~\cite{Sridharan:2007} to compute the affected predicates
of each configuration option. Another way to do so is
using the traditional full slicing algorithm~\cite{Horwitz:1988}.
This variant replaces thin slicing with 
full slicing~\cite{Horwitz:1988} in the configuration
propagation analysis step (Section~\ref{sec:prop}).

\todo{Are variants 2 and three really variants of \ourtool, or are they
  completely different techniques that should be presented as such?  In
  particular, can you characterize variant 2 (here and in the table) as
  Tarantula, possibly with some small enhancements.  I have a similar
  question about variant 3.  In any event, make clearer what part of the
  architecture is replaced by each variant.}

\vspace{1mm}
\noindent \textbf{Variant 2. \ourtool with Coverage Analysis.}
This variant uses statement-level coverage information
to diagnose a configuration error. It treats statements covered
by the undesired execution profile as potentially buggy, and statements
covered the correct execution profiles (from the pre-built database) as correct.
Then, this variant uses a well-known fault localization technique,
Tarantula~\cite{Jones:2002}, to rank the likelihood of each
statement being buggy, and queries the results of thin slice
to identify its affecting configuration options as the root causes. 


\vspace{1mm}
\noindent \textbf{Variant 3. \ourtool with Invariant Analysis.}
This variant uses method-level invariant
to diagnose configuration errors. It stores invariants detected
by Daikon~\cite{Ernst:1999} from correct execution profiles in the database. When a configuration
error occurs, this variant detects invariants from the undesired execution profile;
and compares the detected invariants
with those stored in the database.
It treats a method to have suspicious behaviors if its observed invariants
from the undesired execution profile are different from the invariants stored in the database. Finally, this variant ranks
a method's suspiciousness by the number of different variants, and
queries the results of thin slice
to identify its affecting configuration options as the root causes. 

\vspace{1mm}
\noindent \textbf{ConfAnalyzer: A Heavyweight Dynamic Information Flow-based Approach~\cite{Rabkin:2011:PPC}.}
Rabkin and Katz proposed a family of techniques to precompute possible
configuration diagnosis for Java software~\cite{Rabkin:2011:PPC}. In their work,
the most accurate technique (also probably one of the most precise techniques in the literature)
is based on dynamic information flow analysis.
This technique works remarkably well for crashing errors, though as
described above these are often easy to diagnose even without tool
support.  ConfAnalyzer cannot diagnose non-crashing errors.

%\todo{Is ConfAnalyzer heavyweight?  If so, say so.}

\vspace{1mm}

The experimental results of comparing \ourtool with the above
four alternative approaches are shown in Figure~\ref{tab:results}.
$\blacksquare$ discuss the results here

