
%\todo{Are variants 2 and three really variants of \ourtool, or are they
%  completely different techniques that should be presented as such?  In
%  particular, can you characterize variant 2 (here and in the table) as
%  Tarantula, possibly with some small enhancements.  I have a similar
%  question about variant 3.  In any event, make clearer what part of the
%  architecture is replaced by each variant.}

One possible way to diagnose a configuration error is to leverage
the existing fault localization techniques, by treating the undesired
execution as a failing run and all correct executions (in the database)
as passing runs. We next compare \ourtool with two fault-localization-based
techniques: % in error diagnosis:

\begin{itemize}
\item \textbf{Statement-level Coverage Analysis}. This technique treats statements covered
by the undesired execution profile as potentially buggy, and statements
covered the correct execution profiles as correct.
Then, it leverages a well-known fault localization technique,
Tarantula~\cite{Jones:2002}, to rank the likelihood of each
statement being buggy, and queries the results of thin slice
to identify its affecting configuration options as the root causes.

\item \textbf{Method-level Invariant Analysis}. This technique stores invariants detected
by Daikon~\cite{Ernst:1999} from correct execution profiles in the database.
When a configuration error occurs, this technique detects invariants from the undesired execution profile;
and compares the detected invariants with those stored in the database.
It treats a method to have suspicious behaviors if its observed invariants
from the undesired execution profile are different from the invariants stored
in the database~\cite{McCamant:2003}. Finally, this technique ranks a method's suspiciousness by
the number of different invariants, and queries the results of thin slice
to identify its affecting configuration options as the root causes. 
\end{itemize}


The experimental results are shown in Figure~\ref{tab:results} (Columns
``Coverage Analysis'' and ``Invariant Analysis'').
$\blacksquare$ discuss the results here

statement-level granularity is too fine
many statement has the same coverage
Tarantuala does not distinguishes the evaluation results of a predicate, it only count whether it has been executed or not.


method-level granularity is too coarse
is not sensitive enough on small control flow changes. for example, the invariant is the same


focusing on the behaviors of the affected predicates can be a good choice.
