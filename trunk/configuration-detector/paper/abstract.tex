\begin{abstract}

The behavior of a software system often depends on how that system is
configured. 
Small configuration errors can lead to
hard-to-diagnose undesired behaviors.
We present a technique (and its tool implementation,
called \ourtool) to identify the root cause of a configuration error ---
a single configuration option that can be changed to produce desired behavior.
Our technique uses static analysis,
dynamic profiling, and statistical analysis to link the
undesired behavior to specific configuration options.
It differs from existing approaches in
two key aspects: it does not
require users to provide a testing oracle (to check whether
the software functions correctly) and thus is fully-automated; and it can
diagnose both crashing and non-crashing
errors.

%, and uses statistical inference to link the observed
%undesired behaviors to specific configuration options.

%$\blacksquare$ say a few of our tool's technique


%We have built a tool called \ourtool for Java software.
We evaluated \ourtool on \noncrash non-crashing configuration errors
and \crash crashing configuration errors from
\subjectnum configurable software systems written in Java.
On average, the root cause was \ourtool's seventh-ranked suggestion; in
\topnum out of \errors errors, the root cause was the top 3 suggestions;
and more
than half of the time, the root cause was the first suggestion.

% \ourtool identifies the
% source of the misconfiguration
% as the top 3 root cause for \topnum out of all \errors errors,
% making it a promising solution. %in automated debugging. %error diagnosis. %alternative to manual debugging.

%that xxx -- \ourtool uses these
%xxx to link the undesired behavior to specific configuration options. Our
%results using \ourtool to solve misconfigurations in xxx, xxx, xxx show
%that \ourtool identifies the soruce of the misconfiguration xxxx.

%\todo{Add this somewhere in the abstract:  Our technique uses static analysis,
%dynamic profiling, and statistical analysis to link the
%undesired behavior to specific configuration options.}

\end{abstract}

\label{fake-label-for-etags}
