\begin{figure*}[t]
\vspace{1mm}
\setlength{\tabcolsep}{.22\tabcolsep}
\small{
\begin{tabular}{|l|c|c||c|c||c||c|c||c|}
\hline
 Error ID.  & Root Cause & \#Options& \multicolumn{2}{|c||}{\ourtool} & ConfAnalyzer~\cite{Rabkin:2011:PPC}& Coverage Analysis& Invariant Analysis & Full Slicing \\
\cline{4-9}
 Program &  & & \#Profiles& Rank  & Rank & Rank & Rank & Rank \\
 \hline
\hline
\multicolumn{9}{|l|}{Non-crashing errors}   \\
 \hline
\phz 1. Randoop& \CodeIn{maxsize} & 57& 10 / 12 & 1 & X & 13 / 13 & N / N &46\\
\phz 2. Weka&\CodeIn{m\_numFolds}& 14 &2 / 12 &1&  X& 4 / 7 & 5 / 5 &9\\
\phz 3. JChord& \CodeIn{eqth}& 79 & 2 / 6 & 3& X & 38 / 31 &2 / 2  &73\\
\phz 4. Synoptic& \CodeIn{partitionRegExp}& 37 & 2 / 6 & 1&  X& 1 / 1 & N / N &6\\
\phz 5. Soot& \CodeIn{keep\_line\_number} &49 & 6 / 16 & 2 & X & 46 / 18 & N / N &N\\
\hline
 \multicolumn{2}{|l|}{Average} & 47.2 & 3.6 / 10.4 & 1.6 & 23.6 & 20.4 / 14.0 & 15.7 / 15.7 & 31.7 \\
\hline
\hline
\multicolumn{9}{|l|}{Crashing errors}   \\
\hline
\phz 6. JChord& \CodeIn{chord.main.class}&79 &4 / 6 & 1& 1 & 1 / 1 & 4 / 4 & 5\\
\phz 7. JChord& \CodeIn{chord.main.class}& 79 &5 / 6 & 1 &  1& 1 / 1 & 4 / 4 & 5\\
\phz 8. JChord& \CodeIn{chord.run.analyses}& 79 &5 / 6 & 17& 1 &17 / 14 & 22 / 17 & 21\\
\phz 9. JChord& \CodeIn{chord.ctxt.kind}& 79 &3 / 6 & 1 &  3& 25 / 27 & 30 / 30 & 75\\
 10. JChord& \CodeIn{chord.print.rels}& 79 & 2 / 6& 15 & 1 & 20 / 16 & 25 / 19 & 24\\
 11. JChord& \CodeIn{chord.print.classes}& 79 &4 / 6 & 16 & 1 & 13 / 15 & 17 / 18 & 22\\
 12. JChord& \CodeIn{chord.scope.kind}& 79 &5 / 6 & 1&  1& 1 / 1 & N / N& 10\\
 13. JChord& \CodeIn{chord.reflect.kind}& 79 &6 / 6 & 1& 3 & 5 / 6 & 9 / 9 & 11\\
 14. JChord& \CodeIn{chord.class.path}& 79 &4 / 6 & 8 &  N& 21 / 2 & 26 / 5 & 6\\
\hline
 \multicolumn{2}{|l|}{Average} & 79 & 4.2 / 6 & 6.7 & 5.7 & 11.5 / 9.2 & 19.5 / 16.1 & 19.8\\
\hline
\end{tabular}
}
\vspace{-3mm}
%\todo{Perhaps add a new column, after ``root cause configuration option'', giving
%  the total number of configuration options in the program.  This will
%  emphasize how good the ``rank'' numbers are.}
\Caption{{\label{tab:results} Experimental results in diagnosing software
configuration errors. Column ``Root Cause'' shows the actual
root cause configuration option. Column ``\#Options'' shows the
number of available configuration options, taken from Figure~\ref{tab:subjects}.
Column ``\ourtool'' shows the results of using our technique. 
Column ``\#Profiles'' shows the number of similar execution profiles selected
from the pre-built database for comparison, and the total size of the database.
Column ``ConfAnalyzer'' shows the results of using an existing technique~\cite{Rabkin:2011:PPC} (Section~\ref{sec:confanalyzer});
and the data in this column is taken from~\cite{Rabkin:2011:PPC}.
\todo{I added the above sentence to explain where the data comes from}
Columns ``Coverage Analysis'' and ``Invariant Analysis'' show the results of using
two fault localization techniques as described in Section~\ref{sec:comparison}.
\todo{I added the following sentences}
For these two columns, a slash ``/'' separates the results of using
the selected similar execution profiles by \ourtool (Section~\ref{sec:similar}) and the results of using all execution profiles from
the pre-built database.
Column ``Full Slicing'' shows the results of using full slicing~\cite{Horwitz:1988} to compute
the affected predicates (Section~\ref{sec:choices}).
For each technique, Column ``Rank'' shows the absolute rank of the actual root 
cause in its output (lower is better). ``X''
means the technique is not applicable (i.e., requiring a crashing point), and ``N'' means the technique
does not identify the actual root cause. When computing the average
rank, each ``X'' or ``N'' is treated as half of the number of configuration options, because
a user would need to examine on average half of the options to find the root cause.
}}
\end{figure*}

%daikon can not work on synoptic, use N to denote this
