
\section{Introduction}
\label{sec:introduction}

Modern software is extraordinarily complex. Many applications have a large
number of configuration options that offer users great flexibility to
customize their behaviors. This flexibility has a cost: when something
goes wrong, fixing a configuration problem can be both time-consuming
and frustrating. 
Technical support contributes 17\% of the total cost of ownership of
today's desktop computers~\cite{confevidence}, and troubleshooting misconfigurations
is a large part of technical support.

Misconfigurations are often exhibited by an application unexpectedly terminating
or producing undesirable output. While an ideal application would always
output a helpful error message when such events occur, it is unfortunately
the case that such messages are often cryptic, misleading, or even non-existent.
Thus, users must search manuals, FAQs, and online forums to find potential
solutions to the problem. $\blacksquare$ this process is frustrating..

$\blacksquare$ also lead to spurious bug reports.

The process of explaining configuration errors can be divided into two
separate tasks: diagnosing which specific configuration option is
responsible for the unexpected behavior, and determining how to fix that
configuration option. In this paper, we address the former task: finding
the root cause of a configuration error.

We are focusing specifically on configuration errors, in which the application
code is correct, but the program is inappropriately configured so that
it does not behave as desired. For instance, a mistake in a configuration
option passed in command line may lead software to crash, or simply produce
incorrect output.


\vspace{1mm}
\noindent \textbf{\textit{Our technique.}} 

$\blacksquare$ We assume that the error is know, i.e., the error has
been previously encountered ... Thus, the problem of explaining a
configuration error on a different environment can be reduced to
identifying that the xxx is in a state similar to a xxx good state
on the reference computer for which a solution is known.

\ourtool helps mitigate such problems.


Most prior work on configuration debugging has relied
on large user communities or on modifying the program�s execution
environment $\blacksquare$ focusing on crashing errors.

The core of our approach is to xxx. 
We envision this being done by the developers
at release time. When a user encounters an error, they can
use the error message to query this database, perhaps via a web
service.

The developers provide a profile database, which users can use
to query. The users can also enrich the database, providing their
own examples. even a single run

$\blacksquare$ our technique is lightweighted.
Our technique requires no alterations to
the JVM or standard library. This distinguishes our work from
competing techniques such as dynamic taint tracking.

\vspace{1mm}
\noindent \textbf{\textit{Evaluations.}} 

\vspace{1mm}
\noindent \textbf{\textit{Contributions.}} This paper makes
the following contributions:


