\section{Technique}
\label{sec:technique}

We model a configuration as a set of key-value pairs, where
the keys are strings and the values have arbitrary type. This
abstraction is offered
by the POSIX system environment, the Java Properties API,
and the Windows registry.


\subsection{Overview}

Figure~\ref{fig:workflow} sketches the high-level workflow of our technique.
Our technique takes as input a Java program and its configuration options.
It first performs a propagation analysis to identify
the affected predicates for each configuration option (Section~\ref{sec:prop}).
After that, our technique \textit{selectively} instruments
the program at the affected predicates. 
To diagnose an error, a user runs the instrumented program
with the error-revealing input and configuration
to obtain an execution profile (Section~\ref{sec:profiling}).
Then, our technique analyzes the obtained execution profile
to identify the behaviorally-deviated predicates and their
root causes, and reports these to the user (Section~\ref{sec:analysis}).

%The output error report is a ranked list
%of suspicious configuration options that may cause the exhibited problem.

%, linking each configuration
%to its affected predicates

\tinysqueeze

\subsection{Configuration Propagation Analysis}
\label{sec:prop}
\tinysqueeze
For each configuration option, this step statically determines
its affected \textit{predicates}. In our context, a \textit{predicate}
is a boolean expression in a conditional or loop statement, whose evaluation result
decides whether to execute the followed statement or not.
A predicate's rutnime outcome affects the program control flow.
In \ourtool, we focus on identifying and monitoring 
configuration option-affected control flow in a program
rather than the value in an arbitrary data point based on the
following two observations. First, control flow 
often propagates the majority of configuration-related affects
and is essential to a program's execution path, while
the value of a specific data flow point is largely input-dependent.
And second, the outcome of a program predicate can only be
either true or false; thus, the number of recorded states in monitoring
affected control flow is far less than monitoring arbitrary
data values.  Nevertheless, program predicate is not the only
abstraction our technique can use. In our experiments (Section~\ref{sec:evaluation}),
we empirically demonstrate that choosing other abstractions
such as monitoring statement-level coverage
or method-level invariant yields less accurate results.


To identify the predicates affected by a configuration option, a straightforward
way is using program slicing~\cite{Horwitz:1988} to compute
a forward slice from the initialization statement of a
configuration option. Unfortunately, traditional slicing has
several limitations that prevent it from being used.
First, traditional slicing does not distinguish flows along
pointers from flows along values; thus, a resulting slice includes all statements that
\textit{may} affect a point of interest and often grows too large. Second,
many statements included in the resulting slice are indirectly
affected by a configuration option value but may not be pertinent
to the task of diagnosing a configuration error.
Monitoring the control flow of such indirectly-affected statements 
and then linking their behaviors to specific configuration option values
may lead to less accurate diagnosis.

Take the code in Figure~\ref{fig:example} as an example to illustrate
this problem.  Traditional slicing concludes that the predicates
in lines 104, 312, and 315 are affected by the configuration option \CodeIn{maxsize}.
However, the predicates in lines 104 and 315, though possibly
affected by the \CodeIn{maxsize}, are actually irrelevant
to \CodeIn{maxsize}'s value. That is, the value of \CodeIn{maxsize}
controls the length of a generated a sequence rather
than deciding whether a sequence has an active flag (line 104) or
a sequence has been executed before (line 315).

%such slightly-related predicates (computed by full slicing) and linking their behaviors with a
%configuration option may decrease the diagnosis accuracy.
%This has also been confirmed by our experiments.

%However, if the provided input changes the workflow,
%instead of all data flow into it. 

To address this limitation, our technique uses thin
slicing~\cite{Sridharan:2007} as a manner to include
\textit{only} statements that are \textit{directly} affected by a configuration option.
Differing from traditional slicing, thin slicing
focuses on statements that flow values to the seed (here, a
seed is the initialization statement of a configuration option), ignoring the 
control flow dependencies as well as the uses of
base pointers. By doing this, thin slicing improves the relevance
of the slice by only including the statements that compute
and copy a value to a configuration option.
This property separates
pointer computations from the flow of configuration option value,
naturally connects a configuration option with its
directly affected statements, and makes thin slicing
especially attractive.
For example, in the code excerpt of Figure~\ref{fig:example},
a forward thin slice computed for \CodeIn{maxsize}
only includes the predicate in line 312.
In Section~\ref{sec:evaluation}, 
we also empirically compare traditional slicing with
thin slicing, and demonstrate that thin slicing is a better choice
over traditional slicing.

%When Randoop is used to generate tests for different inputs (here,
%input mean programs under test), the created tests (method-call
%sequence at line 12) would be dramatically different.
%However, for similar inputs, the program execution flow should
%be similar.


%In fact, there is another configuration option $\blacksquare$
%that affect line 6.

% 


\tinysqueeze
\subsection{Configuration Behavior Profiling}
\label{sec:profiling}
\tinysqueeze
This step instruments the tested program offline
by inserting code to monitor each affected predicate's outcome
at runtime.

The instrumentation step simple. It inserts 2 statements
before and after each affected program predicate to monitor
a predicate's behavior before and after being evaluated, respectively. Take
the code in Figure~\ref{fig:example} as an example.
The predicate at line 312 is affected by
the configuration option \CodeIn{maxsize}. \ourtool inserts
one instrumentation statement before line 312 to
count the number of this predicate being executed, and
inserts another statement after line 312 to count the number
of this predicate being evaluated to true, as follows.


\begin{CodeOut}
\begin{alltt}
310. private ExecutableSequence createNewUniqueSequence() \ttlcb
311.   Sequence newSequence = ...; 
       \underline{incrExecCount("maxsize", "newSequence.size() > maxsize");}
312.   if (newSequence.size() > maxsize) \ttlcb
         \underline{incrTrueCount("maxsize", "newSequence.size() > maxsize");}
314.     return null;
      ...
319. \ttrcb
\end{alltt}
\end{CodeOut}


Executing the instrumented program with inputs and configurations
produces a trace, consists of a set of \textit{predicate profiles}.
Each predicate profile corresponds to a pair of configuration option
and one of its affected predicates. A predicate profile also includes
the predicate's execution count and evaluation result. For example,
suppose the predicate on line 312 has been executed 100 times, in which
30 times has been evaluated to true. \ourtool creates a predicate
profile to represent such observation as follows:

$\langle$ \CodeIn{``maxsize''}, \CodeIn{``newSequence.size() > maxsize''}, 100, 30$\rangle$ 



%\ourtool expresses an execution as
%a predicate profile vector. 
As we show in the experiments (Section~\ref{sec:evaluation}),
such predicate profiles, although by no mean complete in
recording the whole trace, capture
sufficient information to reason about the causal effects of configurations
and how a configuration option relates to a software's behavior, while
also imposing a moderate amount of performance impact
on foreground applications.


%health as the results of executing a set of predicates.

%As we will show in the experiments~\cite{}, this profile
%provides valuable information about the
%program execution and can help validate a test suite
%or indicate the usage context of a function
%or other computation.

%A software user seeking on specific piece of
%information or aiming to verify a specific invariant
%and uninterested in any other facts about the code
%may be able to use xxx to advantage, but will not
%get as much from it as a programmer open to other,
%possibly valuable information.




\tinysqueeze

\subsection{Configuration Deviation Analysis}
\label{sec:analysis}
\tinysqueeze
\ourtool starts error diagnosis after obtaining the execution profile from
an undesired execution. It selects similar
profiles from known correct executions (Section~\ref{sec:similar}), compares
each selected profile with
the undesired one to identify the most behavioral-deviated predicates
(Section~\ref{sec:deviation}), and then determines
the likely root cause options (Section~\ref{sec:linking}).


\subsubsection{Selecting Similar Execution Profiles for Comparison}
\label{sec:similar}

\ourtool's database contains a number of
profiles from known correct executions.  These execution profiles 
can be dramatically different from another.  To avoid reporting irrelevant
differences when 
determining how and why the observed execution profile behaves
differently from the correct ones, \ourtool first
compares the undesired profile with the existing
correct profiles, then selects a set of similar ones
as the basis of diagnosis.

Given an execution profile $e$, \ourtool first aggregates
the collected predicate profiles into a $n$-dimensional
vector $v_e$ =$\langle r_{e1}, r_{e2}, ..., r_{en}\rangle$, where $n$
is the number of all possible predicate profiles in the program
and each $r_{ei}$ is a ratio representing how often the $i$-th predicate
profile evaluated to true at runtime.
If a predicate has never been executed in an execution,
\ourtool uses 0 as its true ratio. %in its corresponding predicate profiles.

\ourtool computes the similarity of two executions profiles $e$ and $f$
by computing the standard cosine similarity from information retrieval~\cite{Witten96managinggigabytes}
of $v_{e}$ and $v_{f}$.

\vspace{-2mm}

{\small{
\[
\|Similarity|(e, f) = \|cos\_sim|(v_{e}, v_{f}) = \frac{\sum_{i = 1}^{n}r_{ei} \times r_{fi}}
{\sqrt{\sum_{i = 1}^{n}r_{ei}^2} \times \sqrt{\sum_{i = 1}^{n}r_{fi}^2}}
\]
}}

\vspace{-2mm}
\todo{Weird enough that why two sqrt symbols above are not the same size?}

This similarity metric basically compares two execution profiles based on
 control flow taken (approximated by how often a predicate is evaluated to
true). Its resulting value ranges from 0 meaning completely different, to 1 meaning almost the same, 
and in-between values indicating intermediate similarity.


%$\blacksquare$ put the intuition here. dot production? why ignore
%numbers? similarity? control flow.

%I am re-inventing wheels below, just use cosine similarity above

%\ourtool computes the distance between two execution profiles $t_1$ and $t_2$ using
%the following equations:

%{\small{
%\[
%\|Distance|(t_1, t_2) = 1 - \frac{\sum_{i = 1}^{n}Delta(r_{1i}, r_{2i})^2}{n}
%\]

%\[
%\|Delta|(r_1, r_2) = 
%\left\{
%\begin{array}{l l l l}
%  0 & \ \mbox{if $r_1$ = \CodeIn{N/A} and $r_2$ $\neq$ \CodeIn{N/A}}\\
%  0 & \ \mbox{if $r_1$ $\neq$ \CodeIn{N/A} and $r_2$ = \CodeIn{N/A}}\\
%  1 & \ \mbox{if $r_1$ = \CodeIn{N/A} and $r_2$ = \CodeIn{N/A}}\\
%  \CodeIn{min}\{r_2/r_1, r_1/r_2\} & \; \mbox{otherwise}\\ \end{array} \right.
%\]
%}
%}


%$\blacksquare$ need to illustrate why use the distance metric.
%\todo{Also explain/justify.  Give intuition for the metric.}


A crashing error may happen shorty after the program is launched.
Therefore, the resulting execution profile can be much smaller than
%a correct execution profile, 
most correct execution profiles,
since many predicates in the program are not executed.
Using the $Similarity$ metric to compare a correct execution profile from the database
with a crashing profile, it is unlikely for \ourtool to find similar ones.
To facilitate comparison, %avoid comparing those un-executed predicates,
when diagnosing a crashing error, \ourtool first
chops each correct execution profile by only remaining
the common predicates executed by the crashing profile, and then
uses the chopped profile for comparison. 

Given an undesired execution profile,
\ourtool selects all execution profiles (or the chopped profiles for
a crashing error) from the database
with a $Similarity$ value above a threshold (default value: 0.9, as used in our
experiments).

%In addition,
%when diagnosing a crashing error, such chopped execution profiles will replace the original profiles
%(in the database) in the following steps.

%(also used
%the chopped profile for next steps). Doing so, \ourtool
%avoids comparing irrelevant differences in a correct execution
%profile which a crashing profile even has not reached yet.

%An execution profile from a crashing error

%\todo{Mike does not understand this paragraph.}
%For the execution profile produced in a crashing error, \ourtool 
%chops a correct execution profile from the database by only remaining the predicate
%profiles covered by the undesired profile. \ourtool performs
%this simple preprocessing because a crashing error $\blacksquare$
%often produces an incomplete execution profile, and it is unlikely
%to find a similar one from the database.


\subsubsection{Identifying Deviated Predicates}
\label{sec:deviation}


Our
automated error diagnosis approach compares an undesired execution profile with a set
of \textit{similar} and \textit{correct} execution profiles. 
%\todo{I like the following sentence.  This intuition should appear in the
%  introduction as well.  The introduction should say what we do (it does
%  this) and also give a hint as to the approach.}
The behavioral differences in the recorded predicates provide evidence for what parts of a program might be
incorrect and why. %This helps to further reason about its root cause.

For a predicate in an execution profile,
there are two primary factors to
characterize its dynamic behavior: how often it is
evaluated (i.e., the number of the
observed executions), and how often it is evaluated to true (i.e., the true ratio).
Ideally, we would like to have a metric
to capture a predicate's desired behavior
from the known correct execution profiles, but is robust enough
to tolerate small noises so that it does
not overfit a specific correct execution profile.
Looking more closely, we found that although
the general execution control flows (approximated by
a predicate's true ratio) may be similar for many 
execution profiles, the absolute execution
number of a predicate may largely depend on the given input, % and
varying greatly across executions. On ther other hand, for
some predicates, they may have
only been executed very few times but the true ratios
are dramatically different across executions.

%that in two execution profiles, a predicate's is only executed
%in very few times but the true ratios are 
%As another example, some program has preprocessing... $\blacksquare$


%Thus, if we can generate a signature that
%captures the execution path of a predicate, we should be
%able to more precisely identify a configuration error

Thus, we need to a metric that can characterize
a predicate's behavior with high sensitivity, meaning the predicate's
behavior (i.e., true ratio) is correlated with
%ratio accounts for
the execution results (i.e., correct or undesired).
But we also want
high specificity, meaning a predicate's behavior
is more representative if it is observed
in more executions.
%behavior should not be
%mis-characterized only based on a small number of
%observed executions.
To do so, we define the following
$\phi$ metric by using a standard way to
combine sensitivity and specificity to compute their
harmonic mean; this metric prefers high scores in both dimensions. 

\vspace{-3mm}

{\small{
\[
\|\phi|(e, p) = \frac{2}{{1}/{\|trueRatio|(e, p)} + {1}/{totalExecNum(e, p)}}
\]
}}

\vspace{-3mm}

In $\phi(e, p)$, $trueRatio(e, p)$ is a function that returns the ratio of the predicate $p$ being
evaluated to true in $e$, and $totalExecNum(e, p)$ is a function
that returns the total number of predicate $p$ being executed in $e$.
To smooth corner cases, if a predicate $p$ is not executed in $e$, i.e., 
$totalExecNum(e, p) = 0$, then $\phi(e, p)$ returns 0; and if a predicate $p$'s true ratio is 0, i.e., $trueRatio(e, p) = 0$, then $\phi(e, p)$ returns
$1/totalExecNum(e, p)$.


Using metric $\phi$, we define the following $Deviation$ metric
to characterize a predicate $p$'s deviation degree across two execution
profiles $e$ and $f$. A larger $Deviation$ value indicates that
predicate $p$ has a more deviated behavior. % between $e$ and $f$.

\vspace{-2mm}

{\small{
\[
\|Deviation|(p, e, f) = |\phi(e, p) - \phi(f, p)|
\]
}}
\vspace{-4mm}

%It balances a predicate's evaluation result and the total number of executions.
\todo{the basic idea is here, but this paragraph needs re-write}
The definitions of metrics $\phi$ and $Deviation$ have several desirable properties
in characterizing a predicate $p$'s behavior. Clearly, $trueRatio(e,p)$
is a value between 0 and 1 and $totalExecNum(e, p)$ is a value greater than 1; increasing
either value while keeping the other one unchanged increases $\phi(e, p)$.
\ourtool focuses on monitoring a program's control flow, which is also
reflected in the definition of $\phi$. In $\phi(e, p)$,
the value of $trueRatio(e, p)$ is more important than the value
of $totalExecNum(e, p)$
in deciding $\phi(e, p)$; since in theory: $1/trueRatio(e, p) \geq 1/totalExecNum(e, p)$, but
in practice: often $1/trueRatio(e, p) \gg 1/totalExecNum(e, p)$.
Thus, the value of $Deviation(p, e, f)$ is much more sensitive to $p$'s
true ratio change between execution profiles $e$ and $f$. 
\todo{A few more properties I do not known how (and whether need) to write down:
(1) when totalExecNum(e, p) increases to some extent, the change Deviation(p)
becomes ignorable, and the change of trueRatio dominates the Deviation value. (2)
}

%and $Deviation(p)$'s value is more sensitive to the change of $trueRatio(e, p)$.
%When computing $Deviation(p)$ across two executions,
%if $p$'s execution number is unchanged, a small change of $p$'s true ratio
%would lead to a non-trivial $Deviation$ value. On the other hand,
%if $p$'s true ratio is unchanged, the change in $p$'s execution number
%leads to a less noticeable $Deviation$ value.

%Intuitively, for two predicates $p_1$ and $p_2$, if they have the same
%he true ratio but $p_1$ has been observed in more executions, we
%should have more confidence in its statistical significance. $\blacksquare$
%(the wording here is bad)


\ourtool computes the $Deviation$ value for each predicate $p$
appearing in two execution profiles $e$ or $f$, and
ranks them in a decreasing order. 

%After comparing the undesired execution profile with one selected correct execution profile,
%\ourtool ranks all observed predicate profiles in
%a decreasing order based on the computed $Deviation$ value.




%\subsubsection{Filtering Execution Noises}
%remove some off-by-one


\subsubsection{Linking Predicates to Root Causes}
\label{sec:linking}

% and outputs a ranked list of suspicious
%configuration options.

\ourtool links the behavioral-deviated
predicate to their root cause configuration options
by using the results of thin slicing (computed by the Configuration Propagation
Analysis step in Section~\ref{sec:prop}).
\ourtool identifies 
the affecting configuration options for each deviated predicate,
and treats the configuration option
affecting a higher ranked deviated predicate as the more likely
root cause. If a predicate is affected by multiple
configuration options, \ourtool prefers options whose initialization
statements are \textit{closer} to the
deviated predicate (in terms of the breath-first search
distance in the dependence graph of thin slicing).
This heuristic is based on the intuition that statements closer to the
predicate seem more likely to be relevant to its behavior.
Hence, we assume the user gradually explores statements of
increasing distance from the suspicious predicate until
the desired statements %(where a configuration option is initialized)
are found; a breadth-first
search of the dependence graph simulates this strategy.


When multiple correct execution profiles are selected for comparison,
\ourtool first produces a ranked list of root cause
configuration options for each comparison pair, and then outputs
a final list by using majority voting over all ranking lists.
In the final ranking list, one configuration option ranks higher
than another if it ranks higher in more than half of the ranking lists.
However, according to Arrow's impossibility theorem~\cite{Fishburn1970103},
no rank order voting system can produce a non-cyclic ranking while also
meeting a specific set of criteria (e.g., getting more than half the voters).
Our implementation breaks possible cycles by arbitrarily ranking the
involved options, but our experiments did not use this capability.

\todo{Should add 1 more sentence to say how to generate the explanation?
In particular, the true ratio in the normal runs are averaged from all
good runs. need to mention that?}

%\todo{Such a ranking can have cycles.  Does the implementation suffer this
%  problem?}




\subsection{Discussion}

In this paper, we focus specifically on configuration errors,
assuming the application code is correct, but the software		
is inappropriately configured so that it does not		
behave as desired. We next discuss some design issues in \ourtool.

%\vspace{1mm}
\vspace{0.5mm}
\noindent \textbf{Differences between program inputs and configuration options.}
We used our judgment to distinguish configuration options 
from other program inputs. A configuration option is
often used to control a program's control flow rather
than produce results, and is often supplied via a command-line
flag or configuration file.

%A configurable software system exposes a range of configuration
%options that permit users to
%customize its behaviors. %Broadly speaking,
%A configuration option can be seen as a special program input
%(or \textit{meta-}input), which needs to be set before the
%software is used. Unlike program inputs, a configuration option is often
%used to control a program's execution rather than
%produce results for a certain task, and thus
%is often independent of the concrete input values that a user might provide.


%\vspace{1mm}
\vspace{0.5mm}
\noindent \textbf{Why not use profiles from unit test executions?}
\ourtool's database stores correct profiles from complete 
executions that start at the main method.
\ourtool does not use profiles from unit test executions, which check the
correctness
of a single program component and produce
an incomplete execution profile that is not representative of
the whole program workflow. 



%\vspace{1mm}
\vspace{0.5mm}
\noindent \textbf{Why not store profiles from failing executions in the database?}
We envision the profile database is built by developers at release time.
It is more natural and easier for a developer to provide correct execution
profiles, instead of anticipating and enumerating the possible
errors a user may encounter.
\looseness=-1

\vspace{0.5mm}
\noindent \textbf{What if a similar execution profile is not available?}
\ourtool's effectiveness largely depends on the availability of
similar execution profiles from the database. For a given undesired execution profile, lacking a similar
profile in \ourtool's database may lead \ourtool to produce
less useful results.  It also indicates inadequacy of the tests from
which the database was constructed.
Future work should remedy this problem. One
possible approach is to synthesize a new execution, either by
generating a new input for the program~\cite{palus} or by mutating an
existing execution~\cite{sumnerICSE2011}.


%Why dynamic slicing is not usable? No seed statement, and great overhead. Using JSlicer incurs
%a great overhead. It needs to track every instruction and
%perform synchronization when dependence graph is updated.

%Our technique can be seen as a way to reduce overhead,
%including selective profiling, and static pre-processing
%techniques.

