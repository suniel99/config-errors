\ourtool starts error diagnosis 
after obtaining the execution trace from
an erroneous execution. It first compares it with
existing traces from known correct executions, selects
similar traces for comparison (Section~\ref{sec:similar}),
identifies the most behavioral-deviated predicates
(Section~\ref{sec:deviation}), and then determines
its most likely responsible options (Section~\ref{sec:linking}).


\subsubsection{Selecting Similar Traces for Comparison}
\label{sec:similar}

\ourtool's pre-built database contains a number of
traces from known correct executions, in which one trace
can be dramatically different from another. To
determine how and why the observed trace behaves
differently from the correct ones, \ourtool first
compares and contrasts the 
the erroneous execution with other correct ones, and
then selects a set of similar ones. $\blacksquare$

Given a trace $t$, \ourtool first aggregates
the observed predicate profiles into a $n$-dimensional
vector $v_{t}$ =$\langle r_1, r_2, ..., r_n\rangle$ , where $n$
is the number of predicates in the program and each $r_i$ is
the ratio of the $i$-th predicate profile being evaluated
to true at runtime. If a predicate is not executed in
a particular execution, \ourtool assigns \CodeIn{null} to its true ratio.


When comparing two traces $t_1$ and $t_2$, \ourtool computes
their inner-product distance of the normalized vectors, and selects
those with distance below a threshold (default value: 0.3,
as we used in experiments). When computing the inner-product distance,
if the value of some dimensional is \CodeIn{null}. $\blacksquare$

For crashing errors, \ourtool does not select similar traces.
Instead, it uses all available traces in the database
for comparison, because all traces there are from correct
executions, and can
be served as reference to identify the crashing root causes. $\blacksquare$

\subsubsection{Identifying Deviated Predicates}
\label{sec:deviation}


The comparison between an erroneous trace with a set
of \textit{similar} traces forms a basis for our
automated error diagnosis approach. Given an erroneous trace and a set of similar but correct trace,
the differences in predicate profiles provide evidence for what parts of a program might be
incorrect and why. This helps to further reason about its root cause.


For each observed predicate $p$, \ourtool uses the following metric
to characterize its deviation degree in two traces $t_1$ and $t_2$:

\[
\|Deviation|(p) = |\phi(t_1, p) - \phi(t_2, p)|
\]

\[
\|\phi|(t, p) = \frac{2}{\frac{1}{\|trueRatio|(t, p)} + \frac{1}{totalNum(t, p)}}
\]

$trueRatio(t, p)$ is a function that returns the ratio of predicate $p$ being
evaluated to true in trace $t$, and $totalNum(p)$ is a function
that returns the total number of observations of predicate $p$ in trace $t$..

The metric $\phi$ has some good properties in characterizing the $\blacksquare$

For non-crashing errors, since the obtained trace is a complete one,
\ourtool computes the $Deviation$ metric for each observed predicate in
both traces, and ranks them in a decreasing order based on the computed $Deviation$ value.

For crashing erros, since the obtained trace is incomplete,
\ourtool computes the $Deviation$ metric for the predicates only appearing
in the crashing trace, and ranks them decreasingly based on the computed value.


%\subsubsection{Filtering Execution Noises}
%remove some off-by-one


\subsubsection{Linking Predicates to Root Causes}
\label{sec:linking}

$\blacksquare$

The last step in \ourtool is to link those behavioral-deviated
predicates to its root causes, and rank the suspicious
configuration options as the output.

The intuition behind is that \ourtool determines that
altering a configuration option may change the application�s
control flow such that it deviates from the correct trace,
 it reports that option as a possible root cause.

When comparing an erroneous trace with a set of correc traces,
\ourtool first ranks suspicious configuration options between
each trace pair, and then averages the ranking across traces.

It first identifies the most deviated predicate profile, and
its affecting predicate.
It identifies the configuration option
affecting the highest ranked predicate profile as the most likely
root cause; in the case of ties, it ranks all tied options
based on its distance in a thin slicing SDG.
%as being equally likely to be the root cause.

Finally, needs to average the results
