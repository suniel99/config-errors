\ourtool starts error diagnosis 
after obtaining the trace from
an erroneous execution. It first compares it with
existing traces from known correct executions, selects
similar traces for comparison (Section~\ref{sec:similar}),
identifies the most behavioral-deviated predicates
(Section~\ref{sec:deviation}), and then determines
the most likely responsible options (Section~\ref{sec:linking}).


\subsubsection{Selecting Similar Traces for Comparison}
\label{sec:similar}

\ourtool's database contains a number of
traces from known correct executions, in which one trace
can be dramatically different from another. To
determine how and why the observed trace behaves
differently from the correct ones, \ourtool first
compares the erroneous trace with the existing
correct traces, and selects a set of similar ones
as the basis of diagnosis.

Given a trace $t$, \ourtool first aggregates
the observed predicate profiles into a $n$-dimensional
vector $v_{t}$ =$\langle r_{t1}, r_{t2}, ..., r_{tn}\rangle$ , where $n$
is the number of all possible predicate profiles in the program
and each $r_{ti}$ is the ratio of the $i$-th predicate profile being evaluated
to true at runtime in trace $t$. For a particular trace, if a predicate is not executed,
\ourtool uses \CodeIn{N/A} as its true ratio.


For two traces $t_1$ and $t_2$, \ourtool computes their similarity
by computing the distance of their corresponding vectors, as follows:

\[
\|Distance|(t_1, t_2) = 1 - \frac{\sum_{i = 1}^{n}Delta(r_{1i}, r_{2i})^2}{n}
\]

\[
\|Delta|(r_1, r_2) = 
\left\{
\begin{array}{l l l l}
  0 & \ \mbox{if $r_1$ = \CodeIn{N/A} and $r_2$ $\neq$ \CodeIn{N/A}}\\
  0 & \ \mbox{if $r_1$ $\neq$ \CodeIn{N/A} and $r_2$ = \CodeIn{N/A}}\\
  1 & \ \mbox{if $r_1$ = \CodeIn{N/A} and $r_2$ = \CodeIn{N/A}}\\
  \CodeIn{min}\{r_2/r_1, r_1/r_2\} & \; \mbox{otherwise}\\ \end{array} \right.
\]


$\blacksquare$ need to illustrate why use the distance metric.

Given an erroneous trace, \ourtool selects all traces from the database
with a distance below a threshold (default value: 0.3 as used in our
experiments).

For the trace produced in a crashing error, \ourtool 
chops a correct trace from the database by only remaining the predicate
profiles covered by the erroneous trace. \ourtool performs
this simple preprocessing because a crashing error $\blacksquare$
often produces an incomplete trace, and it is unlikely
to find a similar one from the database.


\subsubsection{Identifying Deviated Predicates}
\label{sec:deviation}


The comparison between an erroneous trace with a set
of \textit{similar} traces forms a basis for our
automated error diagnosis approach. Given an erroneous trace and a set of similar but correct trace,
the behavioral differences in predicate profiles provide evidence for what parts of a program might be
incorrect and why. This helps to further reason about its root cause.


For each observed predicate profile $p$, \ourtool uses the following metric
to characterize its deviation degree in two traces $t_1$ and $t_2$:

\[
\|Deviation|(p) = |\phi(t_1, p) - \phi(t_2, p)|
\]

\[
\|\phi|(t, p) = \frac{2}{{1}/{\|trueRatio|(t, p)} + {1}/{totalExecNum(t, p)}}
\]

where $trueRatio(t, p)$ is a function that returns the ratio of predicate $p$ being
evaluated to true in trace $t$, and $totalExecNum(p)$ is a function
that returns the total number of predicate $p$ being executed in trace $t$.

The metric $\phi$ has some good properties in characterizing the
runtime behaviors of a predicate $p$.$\blacksquare$.
%The intuition behind is that \ourtool determines that
%altering a configuration option may change the application�s
%control flow such that it deviates from the correct trace,
% it reports that option as a possible root cause.
It balances
a predicate's evaluation result and the total number of executions.
Intuitively, for two predicates $p_1$ and $p_2$, if they have the same
the true ratio but $p_1$ has been observed in more executions, we
should have more confidence in its statistical significance. $\blacksquare$
(the wording here is bad)


After comparing the erroneous trace with one selected correct trace,
\ourtool ranks all observed predicate profiles in
a decreasing order based on the computed $Deviation$ value.




%\subsubsection{Filtering Execution Noises}
%remove some off-by-one


\subsubsection{Linking Predicates to Root Causes}
\label{sec:linking}


\ourtool finally links those behavioral-deviated
predicate profiles to their root causes, and ranks the suspicious
configuration options in the output.

\ourtool consultes the results of thin slicing to identify
the affecting configuration options of each deviated predicate.
It identifies the configuration option
affecting the highest ranked predicate profile as the most likely
root cause; in the case of ties, it ranks all tied options
based on the length of the shortest path from the initialization
statement of a configuration option to the crashing point
in the system dependence graph constructed by thin slicing. 
The intuition behind is that when multiple options are equally
responsible for one predicate, the closest option is more likely
and should be inspected first.

When multiple correct traces are selected to compare against
the erroneous one, \ourtool produces a ranked list of suspicious
configuration options for each comparison pair, and then outputs
a final list by using majority voting on all produced ranking lists.
In the final output list, one configuration option ranks higher
than another one if it ranks higher in more than half of the ranking lists.


%as being equally likely to be the root cause.

