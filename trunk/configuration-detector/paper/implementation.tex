\section{Implementation}
\label{sec:implementation}

We implemented a tool, called \ourtool, on top of the WALA
framework~\cite{wala}. Our tool works at the bytecode level.
It first statically computes the affected predicates
for each configuration option, and then performs offline instrumentation.
The behaviors of all affected predicates are recorded at runtime, and
kept in a trace file. Using the trace file,
\ourtool works in a fully automatic and
push-button way, and scales to realistic programs.

It is worth noting that after compiled into bytecode,
a predicate in Java source code can be translated into multiple bytecode
instructions. Thus, a thin slice computed for a configuration
option may miss predicates that should be included.
Consider the following simplified code snippet from JChord~\cite{jchord}
and its translated WALA bytecode:


\begin{CodeOut}
\begin{alltt}
   // eqth is a configuration option initialized somewhere else
1. if (eqth.equals("run")) \ttlcb
2.   add();
3. \ttrcb
\end{alltt}
\end{CodeOut}
\vspace{-2mm}
\hspace{20mm}$\Downarrow$ 
%\vspace{-2mm}
\begin{CodeOut}
\begin{alltt}
1. r0 = getstatic eqth
2. r1 = ldc "run"
3. r2 = r0.equals(r1)
4. r4 = true
5. conditional branch(eq) r2, r4
7. invokevirtual add() 
\end{alltt}
\end{CodeOut}

In the translated bytecode, lines 1 -- 5 correspond to line 1 in the
source code. However, a thin slice computed
for the configuration option \CodeIn{eqth} only includes lines 1 and 3
in the translated bytecode, missing the actual branching
instruction (line 5 in the bytecode).
To overcome this limitation, our implementation performs an extra check
for each bytecode instruction in the resulting slice:
if an instruction 
is translated from a predicate in the source code, then the
branching instruction (e.g., line 5 above) translated
from that predicate should also be included.


For a Java program, \ourtool does not analyze the standard Java
library and all its dependent libraries. We believe such approximation
is reasonable, since it is
unlikely for a configuration option set on client software
to affect the behaviors of the dependent libraries.

