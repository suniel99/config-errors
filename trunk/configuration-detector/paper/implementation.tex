\section{Implementation}
\label{sec:implementation}

We implemented a tool, called \ourtool, on top of the WALA
framework~\cite{wala}. Our tool analyzes Java bytecode.
It statically computes the affected predicates
for each configuration option, assigns a unique
ID for each affected predicate, and then performs offline instrumentation.
The runtime behavior of all affected predicates are recorded
in a file. 

%Using the profile file,
%\ourtool works in a fully automatic and
%push-button way, and scales to realistic programs.

\todo{the following 3 paragraphs (up to "For a Java program") should be cut first, if we
are running out of space.}

After being compiled into bytecode,
a predicate in Java source code may be translated into multiple bytecode
instructions. Thus, a computed thin slice may
miss predicates that should be included.
Consider the following simplified code snippet from JChord~\cite{jchord}
and its translated WALA bytecode:


\begin{CodeOut}
\begin{alltt}
   // eqth is a configuration option 
1. if (eqth.equals("run")) \ttlcb
2.   add();
3. \ttrcb
\end{alltt}
\end{CodeOut}
\vspace{-2mm}
\hspace{20mm}$\Downarrow$ 
%\vspace{-2mm}
\begin{CodeOut}
\begin{alltt}
1. r0 = getstatic eqth
2. r1 = ldc "run"
3. r2 = r0.equals(r1)
4. r4 = true
5. conditional branch(eq) r2, r4
6. invokevirtual add() 
\end{alltt}
\end{CodeOut}

In the translated bytecode, lines 1 -- 5 correspond to line 1 in the
source code. However, a thin slice computed
for the configuration option \CodeIn{eqth} only includes lines 1 and 3
in the translated bytecode, missing the actual branching
instruction (line 5 in the bytecode).
To overcome this limitation, our implementation performs an extra check
for each bytecode instruction in the resulting slice:
if an instruction 
is translated from a predicate in the source code, then the
branching instruction (e.g., line 5 above) translated
from the same predicate should also be included.


For a Java program, \ourtool does not analyze the standard JDK
library and all the dependent libraries. We believe this approximation
is reasonable, since a configuration option set on client software
usually does not affect the behaviors of its dependent libraries.

