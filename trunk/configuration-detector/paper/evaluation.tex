\section{Evaluation}
\label{sec:evaluation}

%\subsection{Research Questions}

Our evaluation answers the following research questions:

\begin{itemize}
\item How effective is \ourtool in diagnosing the root cause of
configuration errors?
\item Can \ourtool provide more accurate diagnosis information than
the existing approaches? 
\item What are the effects of varying the comparison profiles from
the pre-built database?
\item How long does \ourtool take to diagnose a configuration error?
\end{itemize}

%how well can \ourtool identify solutions to configuration problems?
%How effective is predicate-level ..
%Can \ourtool identify slutions to configuration problems involving xx

\subsection{Subject Programs}

We evaluated \ourtool on \subjectnum Java applications shown
in Table~\ref{tab:subjects}.
Randoop~\cite{randoop} is a robust automated test generator
for Java programs. Weka~\cite{weka} is a useful toolkit implementing
a broad variety of machine learning algorithms. Our evaluation
uses its decision tree module component. JChord~\cite{jchord}
is a program analysis platform that enables users to design, implement,
and evaluate static and dynamic program analyses for Java bytecode.
Synoptic~\cite{synoptic} is a tool to mine a finite state machine
model representations of a system from logs.
Soot~\cite{soot} is a Java optimization framework, which provides four
intermediate representations for analyzing and transforming Java bytecode:

As shown in Table~\ref{tab:subjects}, each subject program has a non-trivial
codebase, and exposes a number of configuration options for users
to customize its behaviors.

\subsubsection{Configuration Errors}



\begin{table}[t]
%\setlength{\tabcolsep}{.84\tabcolsep}
\begin{tabular}{|l|c|c|c|}
\hline
 Program (version) & LOC & \#Conf Options & \#Profiles\\
 \hline
 \hline
 Randoop (1.3.2) & 18587 & 57 & 12\\
 Weka (3.6.7) & 256305 & 14 & 12\\
 JChord (2.1) & 23391 &  79 & 6 \\
 Synoptic (trunk) & 19153 & 37 & 6\\
 Soot (2.5.0) & 159273 & 49 & 16 \\
\hline
\end{tabular}


\Caption{{\label{tab:subjects} Subject programs. 
Column ``LOC'' is the number of lines of code,
as counted by CLOC~\cite{cloc}. Column ``\#Conf Options''
is the number of available configuration options. Column ``\#Profiles''
is the number of profiles in the pre-built database.}}
\end{table}


\begin{table}[t]
\setlength{\tabcolsep}{.24\tabcolsep}
\begin{tabular}{|l|l|l|}
\hline
 Error ID & Program & Description \\
 \hline
\hline
\multicolumn{3}{|l|}{Non-crashing errors}   \\
 \hline
 1 & \randoop & No tests generated for NanoXML~\cite{nanoxml}\\
 2 & \weka & Low accuracy of the decision tree\\
 3 & \jchord & No datarace reported for a racy program\\
 4 & \synoptic & Generate an incorrect model\\
 5 & \soot & Source code line number is missing\\
\hline
\hline
\multicolumn{3}{|l|}{Crashing errors}   \\
\hline
 6 & \jchord & No main class is specified\\
 7 & \jchord& No main method in the specified class\\
 8 & \jchord & Running a nonexistent analysis\\
 9 & \jchord & Invalid context-sensitive analysis name\\
 10 & \jchord & Printing nonexistent relations\\
 11 & \jchord & Disassembling nonexistent classes\\
 12 & \jchord & Invalid scope kind\\
 13 & \jchord & Invalid reflection kind\\
 14 & \jchord & Wrong classpath\\
\hline
\end{tabular}

\Caption{{\label{tab:errors} A list of \errors
configuration errors used in the evaluation.
The 9 crashing errors in the bottom table are taken from~\cite{Rabkin:2011:PPC}.
}}
\end{table}

We searched forums, FAQ pages and the literature of
configuration error diagnosis research to find actual
configuration problems that users have experienced with our
target applications. Combining with our own experience,
we chose \errors representative configuration errors.
Table~\ref{tab:errors} lists the configuration errors for each application.
The configuration error list includes \noncrash non-crashing errors and \crash
crashing errors, covering various data types, such as enumerated types,
numerical ranges, regular expressions, and text entries.

\subsection{Evaluation Procedural}

For each Java application, we constructed a profile database
by running existing (correct) examples from its user manual, discussion
mailiing list, and the published papers. The number of obtained
 profiles is shown in Table~\ref{tab:subjects}. We
found constructing such a database is quite easy. As shown
in Section~\ref{sec:results}, even a database containing
a small number of profiles is sufficient to
diagnose many configuration errors.

%Run examples from user manual to form the database. The effort
%can be amortized in development time.

We made a 30-line change to the JChord code. This change
does not modify JChord's semantic; rather, it just encapsulates
all scattered configuration option initialization statements 
as static class fields. This is purely an implementation
choice because having a separate initialization statement
makes us relatively easier to specify the seed statement
in slicing. Here is a sample modification:



\begin{CodeOut}
\begin{alltt}
   // chord.kobj.k is a configuration option
   // passed as a system property
   public void run() \ttlcb
     ...
     int kobjK = Integer.getInteger("chord.kobj.k");
     ...
   \ttrcb
\end{alltt}
\end{CodeOut}
\vspace{-4mm}
\hspace{20mm}$\Downarrow$ 
%\vspace{-2mm}
\begin{CodeOut}
\begin{alltt}
   static int chord\_kobj\_k = Integer.getInteger("chord.kobj.k");
   public void run() \ttlcb
     ...
     int kobjK = chord\_kobj\_k; 
     ...
   \ttrcb
\end{alltt}
\end{CodeOut}



%Use 1-CFA to construct call graph, achieve much better precision
%than 0-CFA. 

%Use extraction~\cite{Rabkin:2011:SEP}

%To evaluate \ourtool, 

When diagnosing a configuration error, we first reproduce the
error on a \ourtool-instrumented version to obtain the
trace file. Then, using the obtained trace file, we use \ourtool
to identify its root cause.

We use two metrics to evaluate \ourtool's effectiveness:
the absolute ranking of the actual root cause in \ourtool's output,
and the time cost used in diagnosis.


%This step
%could be automated by combining with some existing  work on
%automated configuration option extraction~\cite{}.
All of our experiments were run on a
2.67GHz Intel Core PC with 4GB physical memory (1GB is allocated
for the JVM), running Windows 7.


\subsection{Results}
\label{sec:results}

\begin{table*}[t]
\setlength{\tabcolsep}{.24\tabcolsep}
\begin{tabular}{|l||c||c|c||c|c|c||c|}
\hline
 Error ID.  & & \multicolumn{2}{|c||}{\ourtool} & Full Slicing & Coverage Analysis& Invariant Analysis & ConfAnalyzer~\cite{Rabkin:2011:PPC}\\
\cline{3-8}
 Program & Reponsible Option & \#Cmp Profile & Rank  & Rank & Rank & Rank & Rank \\
 \hline
\hline
\multicolumn{8}{|l|}{Non-crashing errors}   \\
 \hline
 1. Randoop& \CodeIn{maxsize} & 10 & 1 & & N & N &X \\
 2. Weka&\CodeIn{m\_numFolds}&2&1& & 4 & 5 &X\\
 3. JChord& \CodeIn{chord.kobj.k} & 3 & 2& & N &2  &X\\
 4. Synoptic& \CodeIn{partitionRegExp}& 2 & 1& & 1 & &X\\
 5. Soot& \CodeIn{keep\_line\_number} & 1 & 3 &  & N& &X\\
\hline
\hline
\multicolumn{8}{|l|}{Crashing errors}   \\
\hline
 6. JChord& \CodeIn{chord.main.class}&3& 1& & & &1\\
 7. JChord& \CodeIn{chord.main.class}&3 & 3& & & &1\\
 8. JChord& \CodeIn{chord.run.analyses}&3 & 17& & & &1\\
 9. JChord& \CodeIn{chord.ctxt.kind}&3 & 1 & & & &3\\
 10. JChord& \CodeIn{chord.print.rels}& 3& 15 & & & &1\\
 11. JChord& \CodeIn{chord.print.classes}&3 & 15 & & & &1\\
 12. JChord& \CodeIn{chord.scope.kind}&3 & 1& & & &1\\
 13. JChord& \CodeIn{chord.reflect.kind} &3 & 1& & & &3\\
 14. JChord& \CodeIn{chord.class.path}&3 & 8 & & & &N\\
\hline
\end{tabular}

\Caption{{\label{tab:results} Experimental results in diagnosing software
configuration errors. Column ``Responsible Option'' shows the actual
configuration option for the error. Column ``\ourtool'' shows the results of using
our technique. Columns ``Full Slicing'', ``Coverage Analysis'',
and ``Invariant Analysis'' shows the results of three variants of \ourtool~\ref{sec:comparison}.
Column ``ConfAnalyzer'' shows the results of an existing
technique~\cite{Rabkin:2011:PPC}.
Column ``\#Cmp Profiles'' shows the number of similar profiles selected
from the pre-built database for comparison.
For each technique, Column ``Rank'' shows the rank of the actual responsible
option in its output (lower is better). ``X''
means the technique is not applicable, and ``N'' means the technique
does not output the actual responsible option.}}
\end{table*}


\subsubsection{Accuracy in Diagnosing Configuration Errors}

Table~\ref{tab:results} shows the experimental results.
We can see that \ourtool is highly effective in pinpointing the root cause of
misconfigurations. For all \noncrash non-crashing errors
and 5 out of the \crash crashing errors, it lists the actual root cause as one of the
top 3 options. For the rest
4 crashing errors, the actual causes are ranked lower.
This is because $\blacksquare$


We next briefly explain the root cause of each non-crashing configuration
errors as shown Table~\ref{tab:results}, and discuss why
\ourtool could identy them. 

\begin{itemize}
\item \textbf{Randoop}. reason
\item \textbf{Weka}. reason
\item \textbf{JChord}. reason
\item \textbf{Synoptic}. reason
\item \textbf{Soot}. reason
\end{itemize}

Compared to non-crashing errors, \ourtool is less effective
in localizing non-crashing errors. They are several reasons
for this.

The nature of the root-cause configuration option is only one factor.
The ranking also depends on how the root-cause option relates to
other options in the suspect set. A highly configurable software system 
likely produces more noises, 

%The detailed illustration
%for those crashing errors can be found in~\cite{Rabkin:2011:PPC}.

The remaining errors are a direct result of $\blacksquare$ and seems
hard for \ourtool to diagnose correctly.


%\ourtool also successfully diagnoses xx\% of the xxx errors. For the
%remaining errors, \ourtool ranks the root cause 9th. The configuration
%error is that the xxx. Thus, the root cause gets ranked lower
%in the list.

Crashing errors are often much easier to diagnose than non-crashing errors,
because crashing errors usually happen shortly after the program
is launched, and a crash often comes up with a stack trace that provides
useful diagnosis clues. For example, \ourtool ranks the root cause of
Error 14 (Table~\ref{tab:results}) 8th and ConfAnalyzer even fails to diagnose it.
However, when JChord crashes, it dumps an \CodeIn{ClassNotFoundException}
that explicitly leads users to check its classpath setting (via the
\CodeIn{chord.class.path} option).


\vspace{1mm}
\noindent \textbf{\textit{Summary.}} \ourtool is effectively
in diagnosing both crashing and non-crashing configuration errors. $\blacksquare$


\subsubsection{Comparison with Alternative Approaches}
\label{sec:comparison}

We next compare \ourtool with four alternative approaches
in diagnosing configuration errors.
The first three approaches are variants of \ourtool,
and the fourth one is an existing technique~\cite{Rabkin:2011:PPC}

\vspace{1mm}
\noindent \textbf{\ourtool with Full Slicing.} 
In \ourtool, we choose thin slicing to compute the affected predicates for
each configuration option. However, such affected predicates
can also be computed by using the traditional full slicing algorithm~\cite{Horwitz:1988}.
To evaluate the trade-offs, this alternative replaces
thin slicing with the traditional
full slicing~\cite{Horwitz:1988} in configuration
propagation analysis (Section~\ref{sec:prop}).

\vspace{1mm}
\noindent \textbf{Coverage Analysis.}
In this variant, we use statement coverage rather than
predicate behaviors to diagnose configuration errors. Specifically,
we treat statements covered by the observed erroneous execution as
potentially buggy, and statements covered the correct executions (from
the pre-built database) as correct. Then, this variant uses
a well-known fault localization technique, Tarantula~\cite{Jones:2002},
to rank the likelihood of each statement being buggy.
For each ranked statement, this variant uses backward thin slicing
to compute the affecting configuration options. $\blacksquare$ clear in
how to rank?


\vspace{1mm}
\noindent \textbf{Invariant Analysis.}
In this variant, we use method-level invariant as the abstraction level
to diagnose configuration errors. This variant first uses
Daikon~\cite{Ernst:1999} to dynamically
detect invariants from the erroneous execution  and invariants
from correct executions (in the pre-built database). Then, it
treats a method potentially buggy if its invariants detected
from the erroneous execution differ from invariants detected
by a correct execution. Finally, it ranks each suspicious
method by the number of different invariants. $\blacksquare$



\vspace{1mm}
\noindent \textbf{Dynamic information flow.}
Rabkin and Katz proposed a family of techniques (and its tool implementation called ConfAnalyzer)
to precompute possible
configuration diagnosis for Java software~\cite{Rabkin:2011:PPC}. In their work,
the most accurate technique (also the most precise technique in the literature,
to the best of our knowledge) is based on dynamic information flow analysis, which precisely
tracks the flow of each configuration option value during program execution at the bit level.
Their technique works remarkably well for crashing errors, but can
not diagnose non-crashing errors.

\vspace{1mm}
%Compare with statistical debugging?
$\blacksquare$ put the results here

\vspace{1mm}
\noindent \textbf{\textit{Summary.}} how effective
Is our \textit{predicate}-level granularity a suitable one?

\subsubsection{Effects of Varying Comparison Profiles}
\label{sec:ranking}


\begin{table}[t]
\setlength{\tabcolsep}{.24\tabcolsep}
\begin{tabular}{|l|c|c||c|}
\hline
 Error ID. & \multicolumn{3}{|c|}{Rank of the Actual Responsible Option } \\
  %& \multicolumn{3}{|c|}{Different Comparison Profile Selection Strategy} \\
\cline{2-4}
 Program & All Profiles & Random Selection&  Similarity-Based\\
 \hline
\hline
\multicolumn{4}{|l|}{Non-crashing errors}   \\
 \hline
 1. Randoop & 1 & 2 & 1\\
 2. Weka & 7 & 6 & 1\\
 3. JChord & 16 & 19 & 2\\
 4. Synoptic & 1 & 1 & 1\\
 5. Soot & 13 & 13 & 3\\
\hline
\hline
\multicolumn{4}{|l|}{Crashing errors}   \\
\hline
 6. JChord & & &1\\
 7. JChord & & &3\\
 8. JChord & & &17\\
 9. JChord & & &1\\
 10. JChord & & &15\\
 11. JChord & & &15\\
 12. JChord & & &1\\
 13. JChord & & &1\\
 14. JChord & & &8\\
\hline
\end{tabular}

\Caption{{\label{tab:selection} Comparison with different profile selection
strategies (Section~\ref{sec:ranking}).
The last column ``Similarity-based'' is the selection strategy
used in \ourtool, and the data in that column is taken from Table~\ref{tab:results}.}}
\end{table}


\ourtool compares the predicate behaviors in the erroneous execution against
correct and similar execution in the pre-built database.
We next investigate the effect of using different comparison
executions. We compare the current similarity-based strategy
 (Section~\ref{sec:similar}) with two alternatives: randomly selecting the same number of
profiles as the similarity-based strategy does, and using
all profiles in the database. Table~\ref{tab:selection} shows the experimental results.

We can see that varying the comparison strategy has result in
substantially different effects on the diagnosis results,
depending on the application being analyzed.

$\blacksquare$

Lacking a similar trace may indicate the inadequacy of
integration testing.

The result difference derives from the nature of the applications.
We found....  . On the other hand,

It does not make sense to compare different profiles...


%What would the technique produce when feeding it with different inputs (e.g.,
%radically different inputs instead of similar ones)?

%In our approach, we first select a set of similar profiles from the  database,
%and then to do comparison. What about just using a single trace, i.e., the
%most similar trace? the most dissimilar traces? or what about just using a set
%of random selected trace.

%Comparison of different distance metrics to find similar statements.

\vspace{1mm}
\noindent \textbf{\textit{Summary.}} how effective

\subsubsection{Performance of \ourtool}

\ourtool performs very well on these errors. The average time
to diagnose xxx (with max, min xx).

The cost of slicing

The cost of instrumentation, permit the use in fielded software?


%The performance of \ourtool is reasonable. The time to diagnose
%an error varies among applications.  XXX app takes less than xxx,
%while xxx takes xxx to complete.


\vspace{1mm}
\noindent \textbf{\textit{Summary.}} how effective

\vspace{1mm}

\subsection{Experimental Discussions}


\noindent \textbf{\textit{Limitations.}} 
Our technique is limited in four aspects.
First, we only focus on named configuration options
with a common key-value semantic, and our tool implementation
and experiments are
restricted to Java. 
Second,  our tool implementation currently does not
support debugging non-deterministic errors. 
For non-deterministic errors, \ourtool could potentially leverage one of
several deterministic replay systems~\cite{Huang:2010:LLD}
that can capture a buggy non-deterministic
execution and faithfully reproduce it for late analysis.
Third, our analysis does not track configuration
options that are passed between processes via the command line.
Fourth, the effectiveness of \ourtool largely
depends on the availability of a similar but correct trace.
Using an abitrary trace (as we demonstrated in Section~\ref{sec:ranking}
by random selection) may significantly affect the results.

%similar inputs, if no input is available, test adequacy.



%Our current \ourtool prototype assists in solving configuration errors that are confined
%to a single computer system, such as a home computer, personal workstation, or stand-alone server. (no process communication...)

%similar inputs, if no inputs is available.

%how easy to construct such database in practice.

%User study of usefulness of the results
\vspace{1mm}

\noindent \textbf{\textit{Threats to Validity.}} 
There are two major threats to validity in our evaluation. 
First, the \subjectnum programs and the diagnosed errors may not be
representative. Thus, we can not claim the results can be
extended to an arbitrary program.
Another threat is that we only employed two dependence
analyses (thin slicing and full slicing) and three
abstraction granularities (predicate level,
statement level, and method level) in our evaluation.
 Using other dependence analyses or abstraction levels
might achieve different results.



%\vspace{1mm}

%\noindent \textbf{\textit{Experimental Conclusions.}} 
%\ourtool makes configuration error diagnosis easier by suggesting
%the specific options that may lead to an unexpected behavior. Compared to
%alternative approaches, \ourtool distinguishes itself by being able to
%diagnose both crashing and non-crashing errors without requiring
%a user-provided testing oracle. $\blacksquare$
