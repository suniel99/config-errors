\section{Evaluation}
\label{sec:evaluation}

%\subsection{Research Questions}

Our evaluation answers the following research questions:

%\todo{I would organize the research questions as:
%\begin{itemize}
%\item
%How effective is \ourtool?
%\begin{itemize}
%\item
%in absolute terms
%\item
%compared to other tools
%\end{itemize}
%This can include run time
%\item
%Discussion of internal implementation choices of \ourtool
%\end{itemize}
%}

%\todo{Give cross-references to sections that address these.}

\begin{itemize}
\item How effective is \ourtool in error diagnosis?% including:
\begin{itemize}
  \item the absolute ranking of the actual root cause in \ourtool's output (Section~\ref{sec:results}).
  \item the time cost in error diagnosis (Section~\ref{sec:performance}).
  \item comparison with existing configuration error diagnosis techniques (Section~\ref{sec:confanalyzer}).
  \item comparison with fault-localization-based configuration error diagnosis techniques (Section~\ref{sec:comparison}).
\end{itemize}
\item What are the effects of using full slicing rather than thin slicing to identify
the affected predicates, and the effects of varying comparison execution profiles (Section~\ref{sec:choices})?
These are two internal design choices. % in \ourtool.
\end{itemize}

%\begin{itemize}
%\item How effective is \ourtool in diagnosing the root cause of
%a configuration error (Section~\ref{sec:accuracy})?
%\item Can \ourtool provide more accurate diagnosis information than
%other approaches (Section~\ref{sec:comparison})? 
%\item How long does \ourtool take to diagnose a configuration error (Section~\ref{sec:performance})?
%\item What are the effects of varying the internal implementation of \ourtool,
%such as using a different configuration propagation analysis and
%different comparison execution profiles (Section~\ref{sec:choices})?
%\item What are the effects of varying the comparison execution profiles (Section~\ref{sec:choices})?
%\end{itemize}

%We use two metrics to evaluate \ourtool's effectiveness:
%the absolute ranking of the actual root cause in \ourtool's output,
%and the time cost used in diagnosis.

\subsection{Subject Programs}


We evaluated \ourtool on \subjectnum Java programs shown
in Figure~\ref{tab:subjects}.
Randoop~\cite{PachecoLET2007} is an automated test generator
for Java programs. Weka~\cite{wekaarticle} is a toolkit that implements
machine learning algorithms. Our evaluation
only uses its decision tree module. JChord~\cite{chordtutorial}
is a program analysis platform that enables users to design, implement,
and evaluate static and dynamic program analyses for Java.
Synoptic~\cite{Beschastnikh:2011} mines a finite state machine
model representations of a system from logs.
Soot~\cite{Vallee-Rai-1999} is a Java optimization framework for analyzing and transforming Java bytecode.


\subsubsection{Configuration Errors}

\begin{figure}[t]
\centering
\small{
\setlength{\tabcolsep}{.64\tabcolsep}
\begin{tabular}{|l|c|c|c|}
\hline
 Program (version) & LOC & \#Config Options & \#Profiles\\
 \hline
 \hline
 Randoop (1.3.2) & 18587 & 57 & 12\\
 Weka (3.6.7) & 256305 & 14 & 12\\
 JChord (2.1) & 23391 &  79 & 6 \\
 Synoptic (trunk, 04/17/2012) & 19153 & 37 & 6\\
 Soot (2.5.0) & 159273 & 49 & 16 \\
\hline
\end{tabular}
}

%\todo{Give a date for Synoptic}
\Caption{{\label{tab:subjects} Subject programs. 
Column ``LOC'' is the number of lines of code,
as counted by CLOC~\cite{cloc}. Column ``\#Config Options''
is the number of available configuration options. Column ``\#Profiles''
is the number of execution profiles in the pre-built database.}}
\end{figure}


\begin{figure}[t]
\setlength{\tabcolsep}{.94\tabcolsep}
\small{
\begin{tabular}{|l|l|l|}
\hline
 Error ID & Program & Description \\
 \hline
\hline
\multicolumn{3}{|l|}{Non-crashing errors}   \\
 \hline
 1 & \randoop & No tests generated\\
 2 & \weka & Low accuracy of the decision tree\\
 3 & \jchord & No datarace reported for a racy program\\
 4 & \synoptic & Generate an incorrect model\\
 5 & \soot & Source code line number is missing\\
\hline
\hline
\multicolumn{3}{|l|}{Crashing errors}   \\
\hline
 6 & \jchord & No main class is specified\\
 7 & \jchord& No main method in the specified class\\
 8 & \jchord & Running a nonexistent analysis\\
 9 & \jchord & Invalid context-sensitive analysis name\\
 10 & \jchord & Printing nonexistent relations\\
 11 & \jchord & Disassembling nonexistent classes\\
 12 & \jchord & Invalid scope kind\\
 13 & \jchord & Invalid reflection kind\\
 14 & \jchord & Wrong classpath\\
\hline
\end{tabular}
}
\Caption{{\label{tab:errors} A list of \errors
configuration errors used in the evaluation.
%The 9 crashing errors in the bottom table are taken from~\cite{Rabkin:2011:PPC}.
}}
\end{figure}

We searched forums, FAQ pages and the literature of
configuration error diagnosis research to find actual
configuration problems that users have experienced with our
target applications. 
We chose \errors configuration errors, in which
the misconfigured values cover various data types, such as enumerated types,
numerical ranges, regular expressions, and text entries;
as listed in Figure~\ref{tab:errors}. The \noncrash non-crashing errors
are collected from actual bug reports, mailing list posts, and our own experience.
The \crash crashing errors, taken from~\cite{Rabkin:2011:PPC},
were used to evaluate the ConfAnalyzer tool.
All \errors configuration errors have been minimized: if
any part of the configuration or input is removed, the software
either crashes or no longer exhibits the undesired behavior.
%\todo{Were any of them used in previous research?  Was that the reason we
%  chose them?}

\begin{figure*}[t]
\setlength{\tabcolsep}{.19\tabcolsep}
\small{
\begin{tabular}{|l|c|c||c|c||c||c|c||c|}
\hline
 Error ID.  & Root Cause & \#Options& \multicolumn{2}{|c||}{\ourtool} & ConfAnalyzer~\cite{Rabkin:2011:PPC}& Coverage Analysis& Invariant Analysis & Full Slicing \\
\cline{4-9}
 Program &  & & \#Profiles& Rank  & Rank & Rank & Rank & Rank \\
 \hline
\hline
\multicolumn{9}{|l|}{Non-crashing errors}   \\
 \hline
\phz 1. Randoop& \CodeIn{maxsize} & 57& 10 / 12 & 1 & X & 16 & N &46\\
\phz 2. Weka&\CodeIn{m\_numFolds}& 14 &2 / 12 &1&  X& 7 & 5 &9\\
\phz 3. JChord& \CodeIn{chord.kobj.k}& 79 & 2 / 6 & 3& X & 31 &2  &73\\
\phz 4. Synoptic& \CodeIn{partitionRegExp}& 37 & 2 / 6 & 1&  X& 1 & N &6\\
\phz 5. Soot& \CodeIn{keep\_line\_number} &49 & 6 / 16 & 2 & X & 18 & N &N\\
\hline
 \multicolumn{2}{|l|}{Average} & 47.2 & 3.6 / 10.4 & 1.6 & 23.6 & 12.6 & 15.7 & 31.7 \\
\hline
\hline
\multicolumn{9}{|l|}{Crashing errors}   \\
\hline
\phz 6. JChord& \CodeIn{chord.main.class}&79 &4 / 6 & 1& 1 & 1 & 4 & 5\\
\phz 7. JChord& \CodeIn{chord.main.class}& 79 &5 / 6 & 1 &  1& 1 & 4 & 5\\
\phz 8. JChord& \CodeIn{chord.run.analyses}& 79 &5 / 6 & 17& 1 &14 & 17 & 21\\
\phz 9. JChord& \CodeIn{chord.ctxt.kind}& 79 &3 / 6 & 1 &  3& 27 & 30 & 75\\
 10. JChord& \CodeIn{chord.print.rels}& 79 & 2 / 6& 15 & 1 & 16 & 19 & 24\\
 11. JChord& \CodeIn{chord.print.classes}& 79 &4 / 6 & 16 & 1 & 15 & 18 & 22\\
 12. JChord& \CodeIn{chord.scope.kind}& 79 &5 / 6 & 1&  1& 1 & N & 10\\
 13. JChord& \CodeIn{chord.reflect.kind}& 79 &6 / 6 & 1& 3 & 6 & 9 & 11\\
 14. JChord& \CodeIn{chord.class.path}& 79 &4 / 6 & 8 &  N& 2 & 5 & 6\\
\hline
 \multicolumn{2}{|l|}{Average} & 79 & 4.2 / 6 & 6.7 & 5.7 & 9.2 & 16.1 & 19.8\\
\hline
\end{tabular}
}
\vspace{-2mm}
%\todo{Perhaps add a new column, after ``root cause configuration option'', giving
%  the total number of configuration options in the program.  This will
%  emphasize how good the ``rank'' numbers are.}
\Caption{{\label{tab:results} Experimental results in diagnosing software
configuration errors. Column ``Root Cause'' shows the actual
root cause configuration option. Column ``\#Options'' shows the
number of available configuration options, taken from Figure~\ref{tab:subjects}.
Column ``\ourtool'' shows the results of using our technique. 
Column ``\#Profiles'' shows the number of similar execution profiles selected
from the pre-built database for comparison, and the total size of the database.
Column ``ConfAnalyzer'' shows the results of using an existing technique~\cite{Rabkin:2011:PPC} (Section~\ref{sec:confanalyzer}).
Columns ``Coverage Analysis'' and ``Invariant Analysis'' show the results of using two fault localization techniques as described in Section~\ref{sec:comparison}.
Column ``Full Slicing'' shows the results of using full slicing~\cite{Horwitz:1988} to compute
the affected predicates (Section~\ref{sec:choices}).
For each technique, Column ``Rank'' shows the absolute rank of the actual root 
cause in its output (lower is better). ``X''
means the technique is not applicable (i.e., requiring a crashing point), and ``N'' means the technique
does not identify the actual root cause. When computing the average
rank, each ``X'' or ``N'' is treated as half of the number of configuration options, because
a user would need to examine on average half of the options to find the root cause.
}}
\end{figure*}

%daikon can not work on synoptic, use N to denote this


\subsection{Evaluation Procedure}

For each subject program, we constructed a profile database
by running existing (correct) examples from its user manual, discussion
mailing list, and published papers~\cite{PachecoLET2007, Beschastnikh:2011, Rabkin:2011:PPC}.
We spent 3 hours per program, on average, and obtained 6--16 execution profiles.
The average size of the profile database is 35MB, and the largest database (Randoop's)
is 72MB.

We made a simple syntactic change to JChord, which affected 24 
lines of code. This change
does not modify JChord's semantics; rather, it just encapsulates
scattered configuration option initialization statements 
as static class fields. \todo{This sentence needs a rewrite:}This is purely an implementation
choice because having a centralized initialization statement
makes our tool implementation easier to specify the seed statement
in performing slicing. Here is a sample modification, where 
\<chord.kobj.k> 
is a configuration option
passed as a system property:


\begin{CodeOut}
\begin{alltt}
public void run() \ttlcb
  ...
  int kobjK = Integer.getInteger("chord.kobj.k");
  ...
\ttrcb
\end{alltt}
\end{CodeOut}
\vspace{-4mm}
\hspace{20mm}$\Downarrow$ 
%\vspace{-2mm}
\begin{CodeOut}
\begin{alltt}
static int chord\_kobj\_k = Integer.getInteger("chord.kobj.k");
public void run() \ttlcb
  ...
  int kobjK = chord\_kobj\_k; 
  ...
\ttrcb
\end{alltt}
\end{CodeOut}


When diagnosing a configuration error, we first reproduce the
error on a \ourtool-instrumented program to obtain the
execution profile. Then, using the obtained execution profile, we use \ourtool
to identify its root causes.

%since we know the misconfigured, root-cause entry for each case,
%we use the ranking of the entry as our evaluation metric.

Our experiments were run on a
2.67GHz Intel Core PC with 4GB physical memory (2GB was allocated
for the JVM), running Windows 7.


\subsection{Results}
\label{sec:results}


\subsubsection{Accuracy in Diagnosing Configuration Errors}
\label{sec:accuracy}


As shown in Figure~\ref{tab:results},
\ourtool is highly effective in pinpointing the root cause of
misconfigurations. For all \noncrash non-crashing errors
and 5 out of the \crash crashing errors, it lists the actual root cause as one of the top 3 options. 


\ourtool is particularly effective in diagnosing non-crashing configuration errors,
which are not supported by most existing tools. The primary reason is due to
\ourtool's ability to identify the behaviorally-deviated predicates through
execution profile comparison, and the top-ranked deviated predicates often provide
useful clues of what parts of a program might be abnormal and why.

Besides the Randoop example in Section~\ref{sec:mot}, we further illustrate
this point by using the non-crashing error in Weka.
Weka's decision tree implementation is highly-tuned, achieving 70--90\% accuracy on
its attached examples. However, we found its accuracy drops to 62\%
on a different dataset we experimented on. We used \ourtool to diagnose this
problem by first building a database by running Weka on its attached examples, and
then
obtaining the undesired execution profile by running it on our dataset. As a result,
\ourtool outputs the following report (only the top option is shown):


\begin{CodeOut}
\begin{alltt} 
Suspicious configuration option: m\_numFolds

It affects the behavior of predicate:
"numFold < numInstances() \% numFolds"
(line 1354, class: weka.core.Instances) 

This predicate evaluates to true:
  20\% of the time in normal runs (4 observations)
  70\% of the time in the undesired run (10 observations)

\end{alltt}
\end{CodeOut}

\vspace{-3mm}

The above report reveals an important fact for the low accuracy.
The predicate \CodeIn{numFold $<$ numInstances() \% numFolds} controls
the depth of a decision tree; and its
true ratio is substantially higher in the undesired execution
than normal executions. Such behavior leads
to a deeper tree that is more likely to overfit the training
data. We changed \CodeIn{m\_numFolds}
value from 2 to 3 to reduce the tree depth, and
gained an 5\% performance increase immediately.

%achieved
%Changing the default value of m\_numFolds from 3 to 2 leads to
%an immediate increase of 5\% improvement

Compared to non-crashing errors, \ourtool is less effective
in diagnosing non-crashing errors. For 4 crashing errors,
the actual causes are ranked below the top 3 options.
This is a direct result of lacking enough predicate behavior observations,
since most crashing errors become apparent shortly
after the program is launched. Therefore, many predicates' $Deviation$ scores
 (Section~\ref{sec:deviation}) are the same; and the heuristic to resolve ties (Section~\ref{sec:linking})
works half the time.


%considers configs closer to the erroneous behavior to be more
%likely to lead to the root cause than those farther away

%$\blacksquare$ the configuration option has
%a long propagation chain, and seems hard for \ourtool
%to diagnose correctly. no statistical significance...

Although \ourtool ranked the actual root course of several
crashing errors lower, crashing errors are generally much easier to diagnose than non-crashing errors.
This is because a crashing error often produces a stack trace with valuable diagnosis clues.
For example, in Figure~\ref{tab:results}, \ourtool ranks the root cause of
error 14  9th.
However, when JChord crashes, it throws a \CodeIn{ClassNotFoundException}
that reminds users to check the classpath setting. For the other three crashing errors (error 8, 10, and 11),
JChord even outputs the wrong configuration option value in the
error message, which
directly guides users to the root cause. 



%$\blacksquare$
%Thus, the root cause gets
%ranked lower in the list. This ordering is a direct result
%of the heuristic discussed in Section XXX that 

%== imprecise of thin slicing, particularly when dealing with containers,







\subsubsection{Performance of \ourtool}
\label{sec:performance}
We measure \ourtool's performance in two ways: the time cost
in diagnosing an error; and the overhead introduced
in reproducing an error in a \ourtool-instrumented program.  Figure~\ref{tab:performance}
shows the results.

\begin{figure}[t]
\setlength{\tabcolsep}{.94\tabcolsep}
\small{
\begin{tabular}{|l|c|c|c|}
\hline
 Error ID. & \multicolumn{2}{|c|}{Time Cost (seconds)} & Slowdown ($\times$)\\
  %& \multicolumn{3}{|c|}{Different Comparison Profile Selection Strategy} \\
\cline{2-3}
 Program & Thin Slicing & Error Diagnosis &  \\
 \hline
\hline
\multicolumn{4}{|l|}{Non-crashing errors}   \\
 \hline
 1. Randoop & 50 & $<$ 1 & 1.1\\
 2. Weka & 43 & $<$ 1 & 1.2 \\
 3. JChord & 147 & 82 & 13.2\\
 4. Synoptic & 24 & $<$ 1 & 3.6 \\
 5. Soot & 95 & 21 & 3.1 \\
\hline
Average & 72 & 21 & 4.4\\
\hline
\hline
\multicolumn{4}{|l|}{Crashing errors}   \\
\hline
 6. JChord & 147 & 79 & 2.4\\
 7. JChord & 147 & 75 & 1.4\\
 8. JChord & 147 & 17 &1.5\\
 9. JChord & 147 & 30 & 28.5\\
 10. JChord & 147 & 13 &13.7\\
 11. JChord & 147 & 10 &65.1 \\
 12. JChord & 147 & 83 &1.6\\
 13. JChord & 147 & 8 &1.9\\
 14. JChord & 147 & 80 &1.4\\
\hline
Average & 147 & 44 & 13\\
\hline
\end{tabular}
}
\Caption{{\label{tab:performance} \ourtool's
performance in diagnosing configuration
errors. The time cost has been divided into
two parts: computing thin slices and diagnosing
an error.}}
\end{figure}

The performance of \ourtool is reasonable.
On average, it uses \avgtime minutes to
diagnose one configuration error. Computing
a thin slice from each configuration option
is expensive. However, this step is one-time effort
per program and the computed slices can be cached
to share across diagnosis. %for future use.

The performance overhead to reproduce the buggy behavior varies
among applications. The current tool implementation
imposes an average slowdown of 13 times when reproducing
an error in a \ourtool-instrumented version.
Such performance overhead, admittedly high, has nonetheless proved acceptable
for offline error diagnoses.

%The size of profile files in the database $\blacksquare$


%The performance of \ourtool is reasonable. The time to diagnose
%an error varies among applications.  XXX app takes less than xxx,
%while xxx takes xxx to complete.



\subsubsection{Comparison with an Existing Technique}
\label{sec:confanalyzer}
We compared \ourtool with ConfAnalyzer, a heavyweight dynamic information
flow-based technique~\cite{Rabkin:2011:PPC}.
We chose ConfAnalyzer because it is the most recent technique and
also one of the most precise configuration error diagnosis techniques
in the literature.
ConfAnalyzer tracks the flow of labelled objects through the
program dynamically,
and treats a configuration option as a root cause if its
value may flow to a crashing point.
ConfAnalyzer works remarkably well for most of the crashing errors (all of
which are selected from the ConfAnalyzer paper~\cite{Rabkin:2011:PPC}), though as
described above these are often easy to diagnose even without tool
support. However, ConfAnalyzer cannot diagnose non-crashing errors.

The experimental results of ConfAnalyzer are shown in Figure~\ref{tab:results} (Column ``ConfAnalyzer'').
For the 9 crashing errors, \ourtool produced better results for 3 of them,
the same results for another 3, and worse results for the remaining 3.

ConfAnalyzer performs better on crashing errors
having shorter execution paths, when an error exhibits
almost immediately after the software is launched.
In such cases, only a small number of configuration options are initialized and
few of them can flow to the crashing point. 
\ourtool fails to produce good diagnosis for these errors, because it can not identify
 the statistically-behavioral-deviated predicates based on the limited
observation of program behaviors.

ConfAnalyzer outputs less accurate or no results
for errors where the root cause option value
flows into containers or system calls (e.g., error 14 in Figure~\ref{tab:results}).
\ourtool can reason (to some extent) about the \textit{consequence} of
such a misconfiguration based on the observed predicate behaviors.



%Many cases only one configuration value flows into.
%For the classpath case, it misses. pollution?

%tracks them as
%they propagate during execution, and (3) identifies, for an
%observed failure, the subset of inputs that are potentially
%relevant for debugging that failure.

%\todo{Is ConfAnalyzer heavyweight?  If so, say so.}




\subsubsection{Comparison with Two Fault-Localization-Based Approaches}
\label{sec:comparison}
\begin{figure}[t]
\textbf{Axuiliary methods}:

sameStatement($\mathit{s}$, $\mathit{s'}$): return whether two statements
$\mathit{s}$ and $\mathit{s'}$ are the same statements. \todo{What does
  ``same statement'' mean?  The same source code?  The same subexpressions
  too? (For example, does this return true for if statement only if the
  predicate, then clause, and else clause are all identical?)}

BFS($\mathit{s}$, $\mathit{cfg}$): return an ordered list of reachable successive statements from statement $\mathit{s}$ in $\mathit{cfg}$ by Breath-First Search (BFS).

firstMatch($\mathit{stmtList_1}$, $\mathit{stmtList_2}$): return the first matched statement pair between $\mathit{stmtList_1}$ and $\mathit{stmtList_2}$ (iterating $\mathit{stmtList_1}$ first).
\todo{What is a ``matched statement pair''?}


\textbf{Input}: two methods from two software versions: $\mathit{m_{old}}$ and $m_{new}$,

\quad a maximum lookahead value $\mathit{lh}$ (Our experiment uses $\mathit{lh}=5$.).\\
\textbf{Output}: matched statements between old and new versions.
\vspace{-4mm}%
matchStatements($\mathit{m_{old}}$, $\mathit{m_{new}}$, $\mathit{lh}$)\\
\begin{algorithmic}[1]
\STATE $\mathit{matchedStmts}$ $\leftarrow$ new Map$\langle$Stmt, Stmt$\rangle$
\STATE $\mathit{cfg_{old}}$ $\leftarrow$ constructControlFlowGraph($\mathit{m_{old}}$)
\STATE $\mathit{cfg_{new}}$ $\leftarrow$ constructControlFlowGraph($\mathit{m_{new}}$)
\STATE $\mathit{stack}$ $\leftarrow$ new Stack$\langle$Pair$\langle$Stmt, Stmt$\rangle$$\rangle$
\STATE $\mathit{stack}$.push($\mathit{cfg_{old}}$.$\mathit{entry}$, $\mathit{cfg_{new}}$.$\mathit{entry}$)
\WHILE{$\mathit{stack}$ is not empty}
\STATE $\langle$$\mathit{stmt_{old}}$, $\mathit{stmt_{new}}$$\rangle$ $\leftarrow$ $\mathit{stack}$.pop()
%\IF{$\mathit{stmt_{old}}$ or $\mathit{stmt_{new}}$ has already been matched}
\IF{$\mathit{matchedStmts}$.keys().contains($\mathit{stmt_{old}}$) \\ \quad || $\mathit{matchedStmts}$.values().contains($\mathit{stmt_{new}}$)}
\STATE \textbf{continue}
\ENDIF
\IF{sameStatement($\mathit{stmt_{old}}$, $\mathit{stmt_{new}}$)}
\STATE $\mathit{matchedStmts}$[$\mathit{stmt_{old}}$] $\leftarrow$ $\mathit{stmt_{new}}$
\ELSE
\STATE $\mathit{stmtList_{old}}$ $\leftarrow$ BFS($\mathit{stmt_{old}}$, $\mathit{lh}$)
\STATE $\mathit{stmtList_{new}}$ $\leftarrow$ BFS($\mathit{stmt_{new}}$, $\mathit{lh}$)
\STATE $\langle$$\mathit{stmt_{old}}$, $\mathit{stmt_{new}}$$\rangle$ $\leftarrow$ firstMatch($\mathit{stmtList_{old}}$, $\mathit{stmtList_{new}}$)
\IF{$\langle$$\mathit{stmt_{old}}$, $\mathit{stmt_{new}}$$\rangle$ $\neq$ null}
\STATE $\mathit{stack}$.push($\langle$$\mathit{stmt_{old}}$, $\mathit{stmt_{new}}$$\rangle$)
\ENDIF
\ENDIF
\ENDWHILE
\RETURN $\mathit{matchedStmts}$
\end{algorithmic}
\vspace{-4mm}
\caption{Algorithm for matching statements from two methods.
\label{fig:matching}
\todo{This algorithm is suspicious to me.  For one thing, the size of the
  stack is always $\le 1$ --- why use a stack in that case?  For another
  thing, as soon as \emph{any} statement satisfies sameStatement(), then
  the algorithm terminates.  Is there a bug in the algorithm?}
}
\end{figure}


In this step, \ourtool compares two execution traces
from two program versions, and identifies the
control flow differences. \ourtool focuses on the
behavior of each recorded predicate. It first 
matches each predicate recorded in the old
execution trace to the new version (Section~\ref{sec:match_predicate}),
and then identify all predicates recorded in both execution traces that
behave differently (Section~\ref{sec:identify_diff}).


\subsubsection{Matching Predicates across Versions}
\label{sec:match_predicate}

For each predicate recorded in the old execution trace,
\ourtool matches it in the new program version.
To match a predicate, \ourtool first matches its declaring method,
by using two strategies. \ourtool uses
the first strategy that succeeds.

%, and then
%identifies its runtime behaviors in the undesired execution profile.


\begin{enumerate}
\item \textbf{Identical method name.} Return a method with the identical
fully-qualified name in the new version.
\item \textbf{Similar method content.} Return the method with
the most similar content in the new version. Given
a method in the old program version, \ourtool
uses the algorithm shown in Figure~\ref{fig:matching} (discussed
in detail below) to \textit{every} instruction in it to
\textit{each} method in the new program version, and then
chooses the method with the most matched matched instructions
(in the new program version).
\end{enumerate}

The algorithm in Figure~\ref{fig:matching} is inspired by
a well-established program differencing algorithm, called
JDiff~\cite{}. \todo{describe JDiff and the difference here.}

If there is no matched method in the new program
version, \ourtool concludes that the predicate cannot be
matched. Otherwise, \ourtool uses the following two strategies
to find the corresponding predicate.

\begin{enumerate}
\item If the declaring method is matched by using the first
``Identical method name'' heuristic, \ourtool runs the algorithm
in Figure~\ref{fig:matching} to establish the mapping between
each instruction, and returns the matched instruction of the
predicate.
\item If the declaring method is matched by using the second
``similar method content'' heuristic,  \ourtool
uses the result of the algorithm in Figure~\ref{fig:matching}
by returning the matched instruction of the given predicate.
\end{enumerate}

%For both cases, if the matching algorithm fails to identified

\begin{figure}[t]

\textbf{Auxiliary methods}: 

\quad getPredicates($\mathit{T}$): return all predicates in the execution trace $\mathit{T}$.

\textbf{Input}: two execution traces: $\mathit{T_{old}}$ and $T_{new}$

\quad the instruction map: $\mathit{stmtMap}$ produced by Figure~\ref{fig:matching}

\quad a threshold: $\delta$ (default value: 0.1, used in our experiments)\\
\textbf{Output}: all pairs of behaviroally-deviated predicates between two versions, and their deviation degree\\
\vspace{-4mm}%
identifyDeviatedPredicates($\mathit{T_{old}}$, $\mathit{T_{new}}$, $\mathit{stmtMap}$)\\
\begin{algorithmic}[1]
\STATE $\mathit{predMap}$ $\leftarrow$ new Map$\langle$$\langle$Predicate, Predicate$\rangle$, Float$\rangle$
\FOR{each predicate $\mathit{p_{old}}$ in getPredicates($\mathit{T_{old}}$)}
\STATE $\mathit{p_{new}}$ $\leftarrow$ $\mathit{stmtMap}$[$\mathit{p_{old}}$]
\STATE $\mathit{v}$ $\leftarrow$ $\phi$($\mathit{p_{old}}$, $\mathit{T_{old}}$)
\IF{$\mathit{p_{new}}$ $\neq$ $\mathit{null}$}
\STATE $\mathit{v}$ $\leftarrow$ $|\phi$($\mathit{p_{old}}$, $\mathit{T_{old}}$) - $\phi$($\mathit{p_{new}}$, $\mathit{T_{new}}$)$|$
\ENDIF
\IF{$\mathit{v}$ $\geq$ $\delta$}
\STATE $\mathit{predMap}$.put($\langle$$\mathit{p_{old}}$, $p_{new}$$\rangle$, $\mathit{v}$)
\ENDIF
\ENDFOR
\FOR{each predicate $\mathit{p_{new}}$ in getPredicates($\mathit{T_{new}}$)}
\IF{$\neg$$\mathit{stmtMap}$.values().contain($\mathit{p_{new}}$)}
\STATE $\mathit{v}$ $\leftarrow$ $\phi$($\mathit{p_{new}}$, $\mathit{T_{new}}$)
\IF{$\mathit{v}$ $\geq$ $\delta$}
\STATE $\mathit{predMap}$.put($\langle$$\mathit{null}$, $p_{new}$$\rangle$, $\mathit{v}$)
\ENDIF
\ENDIF
\ENDFOR
\RETURN $\mathit{predMap}$
\end{algorithmic}
\caption{Algorithm for identifying behavioral-deviated
predicates. The $\phi$ function is defined in Section~\ref{sec:identify_diff}.
\label{fig:identify}
}
\end{figure}


\subsubsection{Identify Control Flow Differences}
\label{sec:identify_diff}

With the predicate matching information, \ourtool further
identifies predicates that behave differently
between two versions. 

The algorithm is shown in Figure~\ref{fig:identify}. \ourtool
\todo{describe the algorithm ..}

\ourtool outputs three categories of predicates:
(1) predicates that are only executed in the
old version, (2) predicates that are only executed
in the new version, and (3) predicates that
are executed in both versions but exhibit different
behaviors. For each category, \ourtool sorts the
predicates based on their deviation scores in the
descreasing order.


Differences in predicate behaviors indicate different executed paths
between two versions. Such differences provide evidence of
which part of the program might be behaving unexpectedly and why.

%is any path not taken by the actual
%program execution that starts at a conditional
%branch instruction for which the branch condition
%is affected by one or more configuration options.





%\subsubsection{Effects of Varying Comparison Execution Profiles}
\subsubsection{Evaluation of Two Design Choices in \ourtool}
\label{sec:choices}
\begin{figure}[t]
\setlength{\tabcolsep}{.74\tabcolsep}
\small{
\begin{tabular}{|l|c|c||c|}
\hline
 Error ID. & \multicolumn{3}{|c|}{Rank of the Actual Root Cause} \\
  %& \multicolumn{3}{|c|}{Different Comparison Profile Selection Strategy} \\
\cline{2-4}
 Program & All Profiles& Random Selection&  Similarity-Based\\
 \hline
\hline
\multicolumn{4}{|l|}{Non-crashing errors}   \\
 \hline
 1. Randoop & 1 & 2 & 1\\
 2. Weka & 7 & 6 & 1\\
 3. JChord & 16 & 19 & 3\\
 4. Synoptic & 1 & 1 & 1\\
 5. Soot & 13 & 13 & 2\\
\hline
Average & 7.6 & 8.2 & 1.6 \\
\hline
\hline
\multicolumn{4}{|l|}{Crashing errors}   \\
\hline
 6. JChord & 1 & 1 &1\\
 7. JChord & 1 & 1 &1\\
 8. JChord & 17 & 17 &17\\
 9. JChord & 1 &  1&1\\
 10. JChord & 15 & 15 &15\\
 11. JChord & 16 & 16 &16\\
 12. JChord & 25 & 25 &1\\
 13. JChord & 1 & 1 &1\\
 14. JChord & 9 & 9 &8\\
\hline
Average & 9.4 & 9.4 & 6.7\\
\hline
\end{tabular}
}
\Caption{{\label{tab:selection} Comparison with different execution profile selection
strategies (Section~\ref{sec:choices}).
The last column ``Similarity-based'' is the selection strategy
used in \ourtool, and the data in that column is taken from Figure~\ref{tab:results}.}}
\end{figure}

We investigate the effects of:

\begin{itemize}
\item using traditional full slicing~\cite{Horwitz:1988} rather
than thin slicing~\cite{Sridharan:2007} in the Configuration
Propagation Analysis step (Section~\ref{sec:prop}) to compute the affected predicates.
Figure~\ref{tab:results}  (Column ``Full Slicing'') shows the results.
\item varying the comparison execution profiles from the pre-built database.
In particular, we compare the similarity-based selection strategy used in \ourtool
 (Section~\ref{sec:similar}) with two alternatives: selecting
all available profiles in the database, and
randomly selecting the same number of profiles as \ourtool uses from the database.
Figure~\ref{tab:selection} shows the results.
\end{itemize}


As shown in Figure~\ref{tab:results}  (Column ``Full Slicing''),
\ourtool achieves significantly less accurate diagnosis when using
full slicing. The primary reason is that full slicing includes
too many irrelevant statements that are only \textit{indirectly} affected by
a configuration option value but not pertinent to the task of
error diagnosis. In many cases, monitoring the control flow
of such indirectly-affected predicates and then linking their
behaviors to configuration options lead to low accuracy. Furthermore,
performing full slicing is much more expensive than thin slicing; in
our experiments, the standard full slicing algorithm ran out of
memory on Soot.

%Furthermore, modern programs typically rely heavily on well-tested data structures
%provided by standard libraries, whose internal details rarely
%the full slice presents far too much information for the task at hand.


As shown in Figure~\ref{tab:selection}, varying the comparison
strategy yields substantially different results,
depending on the application being analyzed.
%\todo{following sentence needs a rewrite}
Using all available execution profiles or randomly
selecting execution profiles is ineffective, because
they make \ourtool report many irrelevant
differences between an undesired
execution and a dramatically different execution.
%, which will not be selected by \ourtool's similary-based selection strategy (Section~\ref{sec:similar}).
We also find diagnosing a crashing error is less sensitive 
to the comparison execution profiles than diagnosing a non-crashing error.
This is because crashing profiles are often much smaller, executing
fewer predicates before reaching the crashing points; and \ourtool
chops each correct execution profile before diagnosis (Section~\ref{sec:similar}).
Thus, many irrelevant differences have already been removed.

 %Comparing
%a crashing profile with a chopped correct execution profile (even
%by random selection) 



%in selecting similar
%comparison execution profiles$\blacksquare$ than diagnosing
%a crashing error.
%For the \crash crashing errors, the only difference yielded
%from using different profile selection strategies is on
%the error 13. $\blacksquare$



\subsection{Experimental Discussions}


\noindent \textbf{\textit{Limitations.}} 
Our technique is limited in three aspects.
First, we only focus on named configuration options
with a common key-value semantic, and our tool implementation
and experiments are
restricted to Java. 
Second,  our tool implementation currently does not
support debugging non-deterministic errors. 
For non-deterministic errors, \ourtool could potentially leverage 
a deterministic replay system
that can capture an undesired non-deterministic
execution and faithfully reproduce it for late analysis.
Third, the effectiveness of \ourtool largely
depends on the availability of a similar but correct execution profile.
Using an arbitrary execution profile (as we demonstrated in Section~\ref{sec:choices}
by random selection) may significantly affect the results.

\vspace{1mm}

\noindent \textbf{\textit{Threats to Validity.}} 
There are two major threats to validity in our evaluation. 
First, the \subjectnum programs and the configuration errors may not be
representative. Thus, we can not claim the results can be
extended to an arbitrary program.
Another threat is that we only employed two dependence
analyses (thin slicing and full slicing) and three
abstraction granularities (at the predicate level,
statement level, and method level) in our evaluation.
 Using other dependence analyses or abstraction levels
might achieve different results.

%how easy to construct such database in practice.
%User study of usefulness of the results


\vspace{1mm}

\noindent \textbf{\textit{Experimental Conclusions.}} 
We have three chief findings: (1) \ourtool is effective
in diagnosing both crashing and non-crashing configuration errors
with a small profile database.
(2) \ourtool produces more accurate diagnosis than
approaches leveraging existing fault localization
techniques~\cite{Jones:2002, McCamant:2003}, 
suggesting the necessity of designing new configuration error
diagnosis techniques. And (3) Using thin slicing
%to identify the affected predicates
permits \ourtool to produce more accurate diagnosis than using
full slicing; and varying the execution profile selection
strategy can result in substantially different diagnosis.

%\ourtool makes configuration error diagnosis easier by suggesting
%the specific options that may lead to an unexpected behavior. 




%Compared to
%alternative approaches, \ourtool distinguishes itself by being able to
%diagnose both crashing and non-crashing errors without requiring
%a user-provided testing oracle. $\blacksquare$
