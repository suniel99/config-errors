\section{Evaluation}
\label{sec:evaluation}

%\subsection{Research Questions}

Our evaluation answers the following research questions:

\todo{I would organize the research questions as:
\begin{itemize}
\item
How effective is \ourtool?
\begin{itemize}
\item
in absolute terms
\item
compared to other tools
\end{itemize}
This can include run time
\item
Discussion of internal implementation choices of \ourtool
\end{itemize}
}

\todo{Give cross-references to sections that address these.}

\begin{itemize}
\item How effective is \ourtool in diagnosing the root cause of
a configuration error?
\item Can \ourtool provide more accurate diagnosis information than
the existing approaches? 
\item What are the effects of varying the comparison traces?
\item How long does \ourtool take to diagnose a configuration error?
\end{itemize}

%how well can \ourtool identify solutions to configuration problems?
%How effective is predicate-level ..
%Can \ourtool identify slutions to configuration problems involving xx

\subsection{Subject Programs}

\todo{The citations are websites.  Add academic citations as well, if
  possible.  (If we don't have enough space, so be it.)}

We evaluated \ourtool on \subjectnum Java programs shown
in Table~\ref{tab:subjects}.
Randoop~\cite{PachecoLET2007} is an automated test generator
for Java programs. Weka~\cite{wekaarticle} is a toolkit that implements
machine learning algorithms. Our evaluation
only uses its decision tree module. JChord~\cite{chordtutorial}
is a program analysis platform that enables users to design, implement,
and evaluate static and dynamic program analyses for Java.
Synoptic~\cite{Beschastnikh:2011} mines a finite state machine
model representations of a system from logs.
Soot~\cite{Vallee-Rai-1999} is a Java optimization framework for analyzing and transforming Java bytecode.


\subsubsection{Configuration Errors}



\begin{table}[t]
%\setlength{\tabcolsep}{.84\tabcolsep}
\begin{tabular}{|l|c|c|c|}
\hline
 Program (version) & LOC & \#Conf Options & \#Traces\\
 \hline
 \hline
 Randoop (1.3.2) & 18587 & 57 & 12\\
 Weka (3.6.7) & 256305 & 14 & 12\\
 JChord (2.1) & 23391 &  79 & 6 \\
 Synoptic (trunk) & 19153 & 37 & 6\\
 Soot (2.5.0) & 159273 & 49 & 16 \\
\hline
\end{tabular}

\todo{Give a date for Synoptic}

\Caption{{\label{tab:subjects} Subject programs. 
Column ``LOC'' is the number of lines of code,
as counted by CLOC~\cite{cloc}. Column ``\#Conf Options''
is the number of available configuration options. Column ``\#Traces''
is the number of traces in the pre-built database.}}
\end{table}


\begin{table}[t]
\setlength{\tabcolsep}{.24\tabcolsep}
\begin{tabular}{|l|l|l|}
\hline
 Error ID & Program & Description \\
 \hline
\hline
\multicolumn{3}{|l|}{Non-crashing errors}   \\
 \hline
 1 & \randoop & No tests generated\\
 2 & \weka & Low accuracy of the decision tree\\
 3 & \jchord & No datarace reported for a racy program\\
 4 & \synoptic & Generate an incorrect model\\
 5 & \soot & Source code line number is missing\\
\hline
\hline
\multicolumn{3}{|l|}{Crashing errors}   \\
\hline
 6 & \jchord & No main class is specified\\
 7 & \jchord& No main method in the specified class\\
 8 & \jchord & Running a nonexistent analysis\\
 9 & \jchord & Invalid context-sensitive analysis name\\
 10 & \jchord & Printing nonexistent relations\\
 11 & \jchord & Disassembling nonexistent classes\\
 12 & \jchord & Invalid scope kind\\
 13 & \jchord & Invalid reflection kind\\
 14 & \jchord & Wrong classpath\\
\hline
\end{tabular}

\Caption{{\label{tab:errors} A list of \errors
configuration errors used in the evaluation.
The 9 crashing errors in the bottom table are taken from~\cite{Rabkin:2011:PPC}.
}}
\end{table}

We searched forums, FAQ pages and the literature of
configuration error diagnosis research to find actual
configuration problems that users have experienced with our
target applications. 
We chose \errors configuration errors, in which
the misconfigured values cover various data types, such as enumerated types,
numerical ranges, regular expressions, and text entries;
as listed in Table~\ref{tab:errors}.
\todo{Were any of them used in previous research?  Was that the reason we
  chose them?}


\subsection{Evaluation Procedure}

For each subject program, we constructed a trace database
by running existing (correct) examples from its user manual, discussion
mailiing list, and published papers\todo{any such papers should have been
  cited above or here}.
We spent 3 hours per program, on average, and obtained 6--16 traces.
The average size of the trace database is 35MB, and the largest one (Randoop's
database) is 72MB.

%The size of trace files in the database $\blacksquare$

%Run examples from user manual to form the database. The effort
%can be amortized in development time.

We made a simple syntactic change to JChord, which affected 88
lines of code. This change
does not modify JChord's semantics; rather, it just encapsulates
scattered configuration option initialization statements 
as static class fields. \todo{This sentence needs a rewrite:}This is purely an implementation
choice because having a centralized initialization statement
makes our tool implementation easier to specify the seed statement
in performing slicing. Here is a sample modification, where 
\<chord.kobj.k> 
is a configuration option
passed as a system property:


\begin{CodeOut}
\begin{alltt}
   public void run() \ttlcb
     ...
     int kobjK = Integer.getInteger("chord.kobj.k");
     ...
   \ttrcb
\end{alltt}
\end{CodeOut}
\vspace{-4mm}
\hspace{20mm}$\Downarrow$ 
%\vspace{-2mm}
\begin{CodeOut}
\begin{alltt}
   static int chord\_kobj\_k = Integer.getInteger("chord.kobj.k");
   public void run() \ttlcb
     ...
     int kobjK = chord\_kobj\_k; 
     ...
   \ttrcb
\end{alltt}
\end{CodeOut}



%Use 1-CFA to construct call graph, achieve much better precision
%than 0-CFA. 

%Use extraction~\cite{Rabkin:2011:SEP}

%To evaluate \ourtool, 

When diagnosing a configuration error, we first reproduce the
error on a \ourtool-instrumented program to obtain the
trace file. Then, using the obtained trace file, we use \ourtool
to identify suspicious configuration options.

%since we know the misconfigured, root-cause entry for each case,
%we use the ranking of the entry as our evaluation metric.

We use two metrics to evaluate \ourtool's effectiveness:
the absolute ranking of the actual root cause in \ourtool's output,
and the time cost used in diagnosis.
Our experiments were run on a
2.67GHz Intel Core PC with 4GB physical memory (2GB is allocated
for the JVM), running Windows 7.


\subsection{Results}
\label{sec:results}

\begin{table*}[t]
\setlength{\tabcolsep}{.09\tabcolsep}
\begin{tabular}{|l||c|c||c|c||c|c|c||c|}
\hline
 Error ID.  & Root cause & \#Options& \multicolumn{2}{|c||}{\ourtool} & Full Slicing & Coverage Analysis& Invariant Analysis & ConfAnalyzer~\cite{Rabkin:2011:PPC}\\
\cline{4-9}
 Program & configuration option & & \#Traces& Rank  & Rank & Rank & Rank & Rank \\
 \hline
\hline
\multicolumn{9}{|l|}{Non-crashing errors}   \\
 \hline
\phz 1. Randoop& \CodeIn{maxsize} & 57& 10 / 12 & 1 & 46 & N & N &X \\
\phz 2. Weka&\CodeIn{m\_numFolds}& 14 &2 / 12 &1& 9 & 4 & 5 &X\\
\phz 3. JChord& \CodeIn{chord.kobj.k}& 79 & 3 / 6 & 2& 73 & N &2  &X\\
\phz 4. Synoptic& \CodeIn{partitionRegExp}& 37 & 2 / 6 & 1& 6 & 1 & \todo{missing?} &X\\
\phz 5. Soot& \CodeIn{keep\_line\_number} &49 & 1 / 16 & 3 & N & N& N &X\\
\hline
 \multicolumn{2}{|l|}{Average} & 47.2 & 3.6 & 1.6 & 33.5 & 2.5 &  & X \\
\hline
\hline
\multicolumn{9}{|l|}{Crashing errors}   \\
\hline
\phz 6. JChord& \CodeIn{chord.main.class}&79 &5 / 6 & 1& 5 & 1 & 4 &1\\
\phz 7. JChord& \CodeIn{chord.main.class}& 79 &5 / 6 & 1 & 5 & 1 & 4 &1\\
\phz 8. JChord& \CodeIn{chord.run.analyses}& 79 &5 / 6 & 17& 21 &14 & 17 &1\\
\phz 9. JChord& \CodeIn{chord.ctxt.kind}& 79 &4 / 6 & 1 & 75 & 27 & 30 &3\\
 10. JChord& \CodeIn{chord.print.rels}& 79 & 4 / 6& 15 & 24 & 16 & 19 &1\\
 11. JChord& \CodeIn{chord.print.classes}& 79 &4 / 6 & 16 & 22 & 15 & 18 &1\\
 12. JChord& \CodeIn{chord.scope.kind}& 79 &5 / 6 & 1& 10 & 1 & N &1\\
 13. JChord& \CodeIn{chord.reflect.kind}& 79 &6 / 6 & 1& 11 & 6 & 9 &3\\
 14. JChord& \CodeIn{chord.class.path}& 79 &5 / 6 & 8 & 6 & 2 & 5 &N\\
\hline
 \multicolumn{2}{|l|}{Average} & 79 & 4.7 & 6.7 & 19.8 & 9.2 & 13.3 &1.5\\
\hline
\end{tabular}
%\todo{Perhaps add a new column, after ``root cause configuration option'', giving
%  the total number of configuration options in the program.  This will
%  emphasize how good the ``rank'' numbers are.}
\Caption{{\label{tab:results} Experimental results in diagnosing software
configuration errors. Column ``Responsible Option'' shows the actual
configuration option for the error. Column ``\#Options'' shows the
number of available configuration options, taken from Table~\ref{tab:subjects}.
Column ``\ourtool'' shows the results of using
our technique. Columns ``Full Slicing'', ``Coverage Analysis'',
and ``Invariant Analysis'' show the results of three variants of \ourtool
as described in Section~\ref{sec:comparison}.
Column ``ConfAnalyzer'' shows the results of an existing
technique~\cite{Rabkin:2011:PPC}.
Column ``\#Traces'' shows the number of similar traces selected
from the pre-built database for comparison, and the total size of the database.
For each technique, Column ``Rank'' shows the rank of the actual responsible
option in its output (lower is better). ``X''
means the technique is not applicable (i.e., requiring a crashing point), and ``N'' means the technique
does not identify the actual responsible option. When computing the average
rank, we only include the case when a technique can output the root cause.}}
\end{table*}


\subsubsection{Accuracy in Diagnosing Configuration Errors}

Table~\ref{tab:results} shows the experimental results.
We can see that \ourtool is highly effective in pinpointing the root cause of
misconfigurations. For all \noncrash non-crashing errors
and 5 out of the \crash crashing errors, it lists the actual root cause as one of the
top 3 options. 


As shown in Table~\ref{tab:results}, \ourtool is particularly effective
in diagnosing non-crashing configuration errors, which are not supported
by most existing tools. $\blacksquare$ why effective? observation?
statically significance? Give some examples here... Randoop \CodeIn{maxsize},
Weka behaves overfitting..


Compared to non-crashing errors, \ourtool is less effective
in diagnosing non-crashing errors. For 4 crashing errors,
the actual causes are ranked lower.
This is because $\blacksquare$ the configuration option has
a long propagation chain, and seems hard for \ourtool
to diagnose correctly. no statistical significance...

Although \ourtool ranked the actual root course of several
crashing errors lower, crashing errors are generally much easier to diagnose than non-crashing errors.
This is because a crashing error usually happens shortly after the program
is launched, and often produces a stack trace with valuable diagnosis clues.
For example, in Table~\ref{tab:results}, \ourtool ranks the root cause of
error 14  8th.
However, when JChord crashes, it dumps a \CodeIn{ClassNotFoundException}
that reminds users to check the classpath setting. For the other three crashing errors (error 8, 10, and 11),
JChord even outputs the wrong configuration option value in the
error message, which
directly guides users to the root cause. We speculated that $\blacksquare$

%The nature of the root-cause configuration option is only one factor.
%The ranking also depends on how the root-cause option relates to
%other options in the suspect set. A highly configurable software system 
%likely produces more noises, 

%The detailed illustration
%for those crashing errors can be found in~\cite{Rabkin:2011:PPC}.

%The remaining errors are a direct result of $\blacksquare$ and seems
%hard for \ourtool to diagnose correctly.


%\ourtool also successfully diagnoses xx\% of the xxx errors. For the
%remaining errors, \ourtool ranks the root cause 9th. The configuration
%error is that the xxx. Thus, the root cause gets ranked lower
%in the list.



\vspace{1mm}
\noindent \textbf{\textit{Summary.}} \ourtool is effective
in diagnosing both crashing and non-crashing configuration errors. $\blacksquare$
As shown
in Section~\ref{sec:results}, even a database containing
a small number of profiles is sufficient to
diagnose many configuration errors.

\subsubsection{Comparison with Alternative Approaches}
\label{sec:comparison}

We next compare \ourtool with three variants and
one existing technique~\cite{Rabkin:2011:PPC}.

\vspace{1mm}
\noindent \textbf{Variant 1. \ourtool with Full Slicing.} 
\ourtool uses thin slicing~\cite{Sridharan:2007} to compute the affected predicates
of each configuration option. Another way to do so is
using the traditional full slicing algorithm~\cite{Horwitz:1988}.
This variant replaces thin slicing with 
full slicing~\cite{Horwitz:1988} in the configuration
propagation analysis step (Section~\ref{sec:prop}).

\todo{Are variants 2 and three really variants of \ourtool, or are they
  completely different techniques that should be presented as such?  In
  particular, can you characterize variant 2 (here and in the table) as
  Tarantula, possibly with some small enhancements.  I have a similar
  question about variant 3.  In any event, make clearer what part of the
  architecture is replaced by each variant.}

\vspace{1mm}
\noindent \textbf{Variant 2. \ourtool with Coverage Analysis.}
This variant uses statement-level coverage information
to diagnose a configuration error. It treats statements covered
by the erroneous trace as potentially buggy, and statements
covered the correct traces (from the pre-built database) as correct.
Then, this variant uses a well-known fault localization technique,
Tarantula~\cite{Jones:2002}, to rank the likelihood of each
statement being buggy, and queries the results of thin slice
to identify its affecting configuration options as the root causes. 


\vspace{1mm}
\noindent \textbf{Variant 3. \ourtool with Invariant Analysis.}
This variant uses method-level invariant
to diagnose configuration errors. It stores invariants detected
by Daikon~\cite{Ernst:1999} from correct traces in the database. When a configuration
error occurs, this variant detects invariants from the erroneous trace;
and compares the detected invariants
with those stored in the database.
It treats a method to have suspicious behaviors if its observed invariants
from the erroneous trace are different from the invariants stored in the database. Finally, this variant ranks
a method's suspiciousness by the number of different variants, and
queries the results of thin slice
to identify its affecting configuration options as the root causes. 

\vspace{1mm}
\noindent \textbf{ConfAnalyzer: A Dynamic Information Flow-based Approach~\cite{Rabkin:2011:PPC}.}
Rabkin and Katz proposed a family of techniques to precompute possible
configuration diagnosis for Java software~\cite{Rabkin:2011:PPC}. In their work,
the most accurate technique (also probably one of the most precise techniques in the literature)
is based on dynamic information flow analysis.
This technique works remarkably well for crashing errors, though as
described above these are often easy to diagnose even without tool
support.  ConfAnalyzer cannot diagnose non-crashing errors.

\todo{Is ConfAnalyzer heavyweight?  If so, say so.}

\vspace{1mm}

The experimental results of comparing \ourtool with the above
four alternative approaches are shown in Table~\ref{tab:results}.
$\blacksquare$ discuss the results here

\vspace{1mm}
\noindent \textbf{\textit{Summary.}} Using thin slicing to identify
the affected predicates leads to more accurate diagnosis than using
full slicing. Monitoring the runtime behaviors of the affected
predicates is more effective in diagnosing a configuration error
than using statement-level profiling and method-level invariant detection.
Dynamic information flow-based approach is slightly better in diagnosing
crashing errors than \ourtool, but fails to diagnose non-crashing errors.

\subsubsection{Effects of Varying Comparison Traces}
\label{sec:ranking}


\begin{table}[t]
\setlength{\tabcolsep}{.24\tabcolsep}
\begin{tabular}{|l|c|c||c|}
\hline
 Error ID. & \multicolumn{3}{|c|}{Rank of the Actual Responsible Option } \\
  %& \multicolumn{3}{|c|}{Different Comparison Profile Selection Strategy} \\
\cline{2-4}
 Program & All Traces & Random Selection&  Similarity-Based\\
 \hline
\hline
\multicolumn{4}{|l|}{Non-crashing errors}   \\
 \hline
 1. Randoop & 1 & 2 & 1\\
 2. Weka & 7 & 6 & 1\\
 3. JChord & 16 & 19 & 2\\
 4. Synoptic & 1 & 1 & 1\\
 5. Soot & 13 & 13 & 3\\
\hline
Average & 7.6 & 8.2 & 1.6 \\
\hline
\hline
\multicolumn{4}{|l|}{Crashing errors}   \\
\hline
 6. JChord & 1 & 1 &1\\
 7. JChord & 1 & 1 &1\\
 8. JChord & 17 & 17 &17\\
 9. JChord & 1 &  1&1\\
 10. JChord & 15 & 15 &15\\
 11. JChord & 16 & 16 &16\\
 12. JChord & 1 & 1 &1\\
 13. JChord & 25 & 25 &1\\
 14. JChord & 8 & 8 &8\\
\hline
Average & 9.4 & 9.4 & 6.7\\
\hline
\end{tabular}

\Caption{{\label{tab:selection} Comparison with different trace selection
strategies (Section~\ref{sec:ranking}).
The last column ``Similarity-based'' is the selection strategy
used in \ourtool, and the data in that column is taken from Table~\ref{tab:results}.}}
\end{table}


\ourtool compares the predicate behaviors in the erroneous trace against
similar traces from the pre-built database.
We next investigate the effect of using different trace selection strategies.
In particular, we compare the similarity-based selection strategy used in \ourtool
 (Section~\ref{sec:similar}) with two alternatives: selecting
all available traces in the database, and
randomly selecting a number\todo{Be specific about exactly how many.  Ideally it
  would be exactly the same number, in each case, as \ourtool used.} of traces from the database.
Table~\ref{tab:selection} shows the experimental results.

We can see that varying the comparison strategy can result in
substantially different effects on the diagnosis results,
depending on the application being analyzed. Diagnosing
a non-crashing error is more sensitive in selecting similar
comparison traces $\blacksquare$ than diagnosing
a crashing error.
For the \crash crashing errors, the only difference yielded
from using different trace selection strategies is on
the error 13. $\blacksquare$


\vspace{1mm}
\noindent \textbf{\textit{Summary.}} Diagnosing a non-crashing
error is more sensitive to the comparison traces than a crashing one.

\subsubsection{Performance of \ourtool}

We measure \ourtool's performance in two ways: the time cost
in diagnosing an error; and the overhead introduced
in reproducing an error before diagnosis.  Table~\ref{tab:performance}
shows the results.

\begin{table}[t]
\setlength{\tabcolsep}{.44\tabcolsep}
\begin{tabular}{|l|c|c|c|}
\hline
 Error ID. & \multicolumn{2}{|c|}{Time Cost (seconds)} & Slowdown ($\times$)\\
  %& \multicolumn{3}{|c|}{Different Comparison Profile Selection Strategy} \\
\cline{2-3}
 Program & Thin Slicing & Error Diagnosis &  \\
 \hline
\hline
\multicolumn{4}{|l|}{Non-crashing errors}   \\
 \hline
 1. Randoop & 50 & $<$ 1 & 1.1\\
 2. Weka & 43 & $<$ 1 & 1.2 \\
 3. JChord & 147 & 82 & 13.2\\
 4. Synoptic & 24 & $<$ 1 & 3.6 \\
 5. Soot & 95 & 21 & 3.1 \\
\hline
Average & 72 & 21 & 4.4\\
\hline
\hline
\multicolumn{4}{|l|}{Crashing errors}   \\
\hline
 6. JChord & 147 & 79 & 2.4\\
 7. JChord & 147 & 75 & 1.4\\
 8. JChord & 147 & 17 &1.5\\
 9. JChord & 147 & 30 & 28.5\\
 10. JChord & 147 & 13 &13.7\\
 11. JChord & 147 & 10 &65.1 \\
 12. JChord & 147 & 83 &1.6\\
 13. JChord & 147 & 8 &1.9\\
 14. JChord & 147 & 80 &1.4\\
\hline
Average & 147 & 44 & 13\\
\hline
\end{tabular}

\Caption{{\label{tab:performance} \ourtool's
performance in diagnosing configuration
errors. The time cost has been divided into
two parts: computing thin slices and diagnosing
an error.}}
\end{table}

The performance of \ourtool is reasonable.
On average, it uses \avgtime minutes to
diagnose one configuration error. The time cost for
computing a thin slice from each configuration option
is high. However, this step is one-time effort
for each program and the computed results can be cached
to share across diagnosis. %for future use.

The performance overhead to reproduce the buggy behavior varies
among applications. The current tool implementation
imposes an average slowdown of 13 times when reproducing
an error in a \ourtool-enabled environment.
Such performance overhead, admittedly high, has nonetheless proved acceptable
for offline error diagnoses.

%The size of trace files in the database $\blacksquare$


%The performance of \ourtool is reasonable. The time to diagnose
%an error varies among applications.  XXX app takes less than xxx,
%while xxx takes xxx to complete.


\vspace{1mm}
\noindent \textbf{\textit{Summary.}} say about \ourtool's performance

\vspace{1mm}

\subsection{Experimental Discussions}


\noindent \textbf{\textit{Limitations.}} 
Our technique is limited in three aspects.
First, we only focus on named configuration options
with a common key-value semantic, and our tool implementation
and experiments are
restricted to Java. 
Second,  our tool implementation currently does not
support debugging non-deterministic errors. 
For non-deterministic errors, \ourtool could potentially leverage one of
several deterministic replay systems~\cite{Huang:2010:LLD}
that can capture a buggy non-deterministic
execution and faithfully reproduce it for late analysis.
Third, the effectiveness of \ourtool largely
depends on the availability of a similar but correct trace.
Using an abitrary trace (as we demonstrated in Section~\ref{sec:ranking}
by random selection) may significantly affect the results.

%similar inputs, if no input is available, test adequacy.



%Our current \ourtool prototype assists in solving configuration errors that are confined
%to a single computer system, such as a home computer, personal workstation, or stand-alone server. (no process communication...)

%similar inputs, if no inputs is available.

%how easy to construct such database in practice.

%User study of usefulness of the results
\vspace{1mm}

\noindent \textbf{\textit{Threats to Validity.}} 
There are two major threats to validity in our evaluation. 
First, the \subjectnum programs and the configuration errors may not be
representative. Thus, we can not claim the results can be
extended to an arbitrary program.
Another threat is that we only employed two dependence
analyses (thin slicing and full slicing) and three
abstraction granularities (at the predicate level,
statement level, and method level) in our evaluation.
 Using other dependence analyses or abstraction levels
might achieve different results.



%\vspace{1mm}

%\noindent \textbf{\textit{Experimental Conclusions.}} 
%\ourtool makes configuration error diagnosis easier by suggesting
%the specific options that may lead to an unexpected behavior. Compared to
%alternative approaches, \ourtool distinguishes itself by being able to
%diagnose both crashing and non-crashing errors without requiring
%a user-provided testing oracle. $\blacksquare$
