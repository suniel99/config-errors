\section{Conclusion and Future Work}
\label{sec:conclusion}

Software evolution often causes software systems to
exhibit undesired behaviors.
When they do, identifying the root causes
can be challenging, especially when software's
behavior depends on a wide range of configuration options.

This paper describes \ourtool, a technique to help software
users to troubleshoot configuration errors. \ourtool
focuses on errors caused by software evolution, and
recommends configuration options whose values can be changed to
produce the desired behavior on the new software version.
Our experimental results show that \ourtool
can accurately identify the root causes of
\errornum configuration errors in \subjnum real-world software systems.
The source code of \ourtool is publicly available
at: \url{http://config-errors.googlecode.com}.

We consider \ourtool as one step towards improving
the diagnosability of configurable software systems.
Future work should focus on the following directions:

\vspace{-2mm}

\begin{itemize}
\item \textbf{User study.} We plan a user study to evaluate
\ourtool's usefulness to end-users. A challenge
will be finding study participants who are familiar
with given subject programs.

\item \textbf{Formulating guidance of error diagnosis.}
An important question that is not answered in this paper
is how to automatically distinguish software bugs from 
configuration errors, when a software system exhibits
some undesired behavior. We plan to formulate guidance
regarding when the user should give up on \ourtool
and assume the error is not related to a configuration option.

\item \textbf{Improving configuration error reporting and fixing.}
A good bug report for a configuration error should contain both the
failing software state and its use-case scenario.
%When fixing a configuration error, there is no obviously ``correct'' way to
%build configuration logic. 
Unlike fixing a bug once,
every software user has to be informed of the right way to
configure the system. Perhaps as a result, misconfigurations
have been one of the dominant causes of system issues and
are likely to continue so~\cite{Yin:2011:ESC}. To address these related problems,
we plan to further study existing software configuration errors
to summarize error patterns, and then use machine learning
techniques to synthesize good-quality error reports and fixes.

\end{itemize}
