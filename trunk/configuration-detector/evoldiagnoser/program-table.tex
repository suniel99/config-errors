
\begin{figure*}[t]
\vspace{1mm}
\centering
\small{
\setlength{\tabcolsep}{.26\tabcolsep}
\begin{tabular}{|l|c|c|c|c|c|c|c|}
\hline
 Error ID.& Error Description & Root-Cause  & \#Options & \multicolumn{3}{|c|}{Rank of the Root-Cause Configuration Option}\\
 \cline{5-7}
 Program& & Configuration Option& & \ourtool & \prevtool~\cite{Zhang:2013:ADS}  & \conftool~\cite{Rabkin:2011:PPC} \\
 %&  Options &  Options&  Options \\
 \hline
 \hline
 1. Randoop& Poor performance in test generation & \CodeIn{usethreads} & \randoopoptnum & \randooprank & \n & \x \\
 2. Weka & A different error message when Weka crashes & \CodeIn{m\_numFolds} & \wekaoptnum &  \wekarank & 9 & 1 \\
 3. Synoptic&  Initial model not saved& \CodeIn{dumpInitialGraphDotFile} & \synopticoptnum & \synopticrankfirst & \n & \x \\
 4. Synoptic& Generated model not saved as JPEG file& \CodeIn{dumpInitialGraphPngFile} &\synopticoptnum & \synopticranksecond & \n & \x \\
 5. JChord& Bytecode parsed incorrectly & \CodeIn{chord.ssa} & \jchordoptnum & \jchordrankfirst & 3& \x \\
 6. JChord& Method names not printed in the console& \CodeIn{chord.print.methods} & \jchordoptnum & \jchordranksecond & \n & \x \\
 7. JMeter& Results saved to a file with a different format & \CodeIn{output\_format} & \jmeteroptnum & \jmeterrank & 1 & \x \\
 8. Javalanche& No mutants generated & \CodeIn{project.tests} &  \javalancheoptnum & \javalancherank& 4 & \x \\
\hline
\hline
 Average &  & & 49.1 & \averagerank &15.3 & 47.5\\
\hline
\end{tabular}
}
\vspace{-2mm}
\Caption{{\label{tab:errors} 
All configuration errors used in the evaluation and
the experimental results. Only the 2nd error is
a crashing error, and all the other errors are non-crashing
errors. Column ``Root-Cause Configuration Option''
shows the actual root-cause configuration option.
Column ``\#Options'' shows the number
of configuration options supported in the new program version, taken from
Figure~\ref{tab:experiment-sub}. 
Column ``Rank of the Root-Cause Configuration Option'' shows the
absolute rank of the actual root-cause
configuration option in each technique's output (lower is better).
%``\N'' means the technique does not identify the root casue configuration options.
``\x'' means the technique is not applicable (i.e., requiring a crashing
point), and ``\n'' means the technique does not identify the actual root cause.
When computing the average rank, each ``\x'' or ``\n'' is treated as
half of the number of configuration options, because a user would need to examine
on average half of the avaiable options to find the root cause.
Column ``\ourtool'' shows the results
of using our technique. Columns ``\prevtool'' and ``\conftool'' show
the results of using two existing techniques as described in Section~\ref{sec:existing}.
}
}
\end{figure*}
