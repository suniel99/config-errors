\section{Introduction}
\label{sec:introduction}

Many modern software systems are configurable. They
have a large number of configuration options that can
customize their behaviors. This flexibility has a cost:
a small configuration error can lead to hard-to-diagnose
behaviors.
%when something goes wrong, diagnosing a configuration
%error can be both time-consuming and furstrating.

Software configuration errors, different than
software bugs, are errors in which
the application code is correct, but the software is
configured incorrectly so that it does not behave
as desired. Such errors may lead software to crash,
produce erroneous output, or simply perform poorly.
We are particularly interested in software configuration
errors because they are \textit{prevalent}, \textit{severe},
\textit{hard to debug}, but \textit{actionable for users
to fix}.


In deployed systems, software configuration errors have been
the dominant cause of problems~\cite{}.
A recent analysis of Yahoo's mission-critical Zookeeper service
showed that software misconfigurations accounted for
the majority of all exhibited bugs~\cite{bft}. Another
recent study analyzed reported problems of a commercial
storage company, and found configuration-related issues
caused about 31\% of all failures~\cite{Yin:2011:ESC}.
Not only are configuration errors prevalent, they
can have high, sometimes disastrous impacts. For example,
an outage in Facebook, due to
an incorrect configuration value, left the website 
inaccessible for about 2 hours~\cite{fbout}. As another example,
the entire .se domain of country Sweden was unavailable
for about 1 hour, due to a DNS misconfiguration problem~\cite{sedown}.
Such incidents, resulting from software configuration errors,
affected millions of users. Furthermore, configuration
errors are difficult to diagnose.
As reflected in a recent article about system deployment experience
in Google, the vast majority of production failures (in Google)
arise not due to bugs in the software, but bugs in the
configuration settings (i.e., configuration errors)
that control the software. Debugging
configuration errors becomes a difficult problem, and an escape
from the ``configuration hell'' is highly demanded~\cite{googleconf}.

However, on the other hand, different than software bugs
which can only be fixed by experienced software developers, fixing a software
configuration error is more \textit{actionable} for software end-users
or system administrators. These users are not the software developers,
and do not have the right expertise to understand (or even access to)
the source code.  But they can simplify fix a configuration error by changing
values of certain configuration options.


\subsection{Configuration Errors Caused by Software Evolution}

Continual change is a fact of life for software systems.
Software evolution can dirupt the existing functionality,
making the software exhibit undesired behaviors in the
new version. After upgrading to a new version,
users often need to carefully exam the existing configuration, and may
configure the software properly. \todo{xx}
Otherwise, the software is working exactly as intended, but the
wrong configuration is leading it to exhibit the undesired behavior.
\todo{not the right place}
Inappropriate reconfiguration can lead to serious results.
For example, on July 26 2012, a small configuration
change caused a system error, and resulted in
Microsoft's public cloud hosting and development platform, Azure,
being unavailable for about two and a half hours~\cite{msdown}.

Why do configuration errors still happen even after the
new software version has been carefully
tested? Different than software bugs, configuration errors are mostly user driven.
A configuration error often occurs when the use of software is unexpected
situations, in which it does not behave as \textit{desired} but
exactly as \textit{intended}. For software developers,
it is impossible for them to test software in every possible
situation in which it might be used; in fact, it is usually
impossible even to foresee every such situation. The situation
is further exacerbated by fact that configuration options
are changed frequently. As software evolves, developers may
add new configuration options, delete or modify existing options.
As reflected by our study in Section~\ref{sec:study}, such
configuration-related changes happen in \textit{every} revision
of \textit{each} subject program, and require users to
re-configure the software correspondingly.


\todo{re-write the below}
Take the open-source JMeter performance testing tool as an example, 
the testing output was saved as a CSV file in version 2.8.
However, as it evolves to version 2.9, the testing output
was saved as an XML file instead using the same input.
All regression tests for version 2.9 passed, and the program
behaves just as \textit{intended} but \textit{differently} that
it used to in version 2.8.


In many cases like the JMeter example above, to recover
from a configuration error and obtain the desired behavior,
users often need to seek information from online help forums, the software
change logs (which tend to be incomplete~\cite{}), or
ask experts. This process can be tedious,
laborious, and frustrating.

\todo{re-write below}
Our technique (and its tool implementation \ourtool) can help
diagnose this problem. Users first demonstrated the
desired and undesired behaviors on two \ourtool-instrumented
JMeter versions, respectively. Then, \ourtool analyzes the
recorded execution profiles from two versions, and produces
a ranked list of root cause configuration options.
\todo{show the value}


\subsection{Diagnosing Configuration Errors}

Broadly speaking, diagnosing a configuration
error can be divided into three separate tasks:
reproducing the error, identifying which specific
configuration option is responsible for the unexpected
behavior, and determining a better value for the
configuration option. \ourtool addresses
the second task: identifying the root cause of
a configuration error

\ourtool aims to help two types of users: software end-users
who may have problems with software installed on their
personal computers; and system administrators who are
responsible for maintaining production systems.
They can use \ourtool to troubleshoot an
error they encounter but do not know how to fix it. 

%when they counter
%an error that they do not know how to fix, to troubleshoot a
%configuration error 
%\ourtool's output can help such users resolve the problems
%they encounter.

\ourtool focuses on diagnosing configuration errors
caused by software evolution. 
Its core idea is to analyze differences between
two execution traces, and then determine specific
configuration options that cause that software
to execute an unexpected path and produce an undesired output.
It uses three steps, as illustrated in Figure~\ref{},  to link the undesired
behavior to specific root cause configuration options:

\begin{itemize}

\item \textbf{Instrumentation and Profiling.} \ourtool
instruments two program versions, and asks users to
demonstrate the different behaviors on two versions.

\item \textbf{Execution Trace Comparison.}
\ourtool identifies branches where the desired execution and
the undesired execution diverged, and then
assigns a cost to each path taken from the branch.

\item \textbf{Root Cause Analysis.} For each difference
in two recorded execution traces, \ourtool finds a likely
reason behiind the differences. Further,
\ourtool attributes differences in execution behavior to
specific configuration options and reports them to the users.


\end{itemize}





\ourtool differs from prior error diagnosis approaches~\cite{} in three
key aspects:
\begin{itemize}
\item \textbf{It diagnoses errors across versions}.
Some existing approaches focus on identifying behavioral differences of
the same program with the different inputs or underlying platforms~\cite{},
while \ourtool targets two different versions of the same program with
the same input and configuration. \ourtool uses a correct software version
as a baseline with which to compare program behavior against, and only
reasons about the behavioral differences.


\item \textbf{It requires no testing oracle}.
Some existing work~\cite{} require users to answer the difficult
question ``why is the software not working'', or to
write a testing oracle to check ``is the software
currently working''. By contrast, \ourtool only requires users to
demonstrate the different behaviors on two versions.

\item \textbf{It determines the root cause option}. Other error diagnosis techniques primarily focus on
determining \textit{what} changes are introduced between
two versions -- they leave the more challenging
question of \textit{how} to eliminate the undesired effects
by those changes unanswered. Users must manually infer
the root cause, e.g., a misconfiguration,
of the unexpected behaviors from the
observed results based on their expertise and knowledge
of the software.

\end{itemize}

\subsection{Evaluation}

We implemented \ourtool for Java software, and empirically evaluated
its effectiveness using \subjnum open-source configurable Java
software. We employ \ourtool to identify behavioral differences between
two versions of these software.

We compared \ourtool to an existing technique~\cite{}, which uses XXX.
The existing technique only \todo{the result}.

Overall, we find that \ourtool is highly effective at helping
identify the root cause of cross-version behavioral differences. \todo{xx}

\subsection{Contributions}

This paper makes the following main contributions:

\begin{itemize}
\vspace{-3mm}
\item \textbf{Study.} We describe an empirical
study of 9 configurable software systems.
Our study indicates that software configuration changes
are frequent and persistent during its evolution (Section~\ref{sec:study}).

\item \textbf{Technique.} We present a technique to diagnosis
configuration errors for evolving software. Our technique
uses dynamic profiling, execution trace comparison, and
static analysis to link certain abnormal behaviors to a
specific responsible configuration option (Section~\ref{sec:technique}).

\item \textbf{Implementation.} We implemented our technique
in a tool, called \ourtool, for Java software (Section~\ref{sec:implementation}).
It is publicly available at \url{http://config-errors.googlecode.com}.

\item \textbf{Evaluation.} We applied \ourtool to \errornum configuration
errors from \subjnum configurable software systems,
and compared it with existing techniques.
The results show the accuracy and efficiency of \ourtool (Section~\ref{sec:evaluation}).
\end{itemize}
