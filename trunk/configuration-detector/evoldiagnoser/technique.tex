
\section{Technique}
\label{sec:technique}

\ourtool models a configuration as a set of key-value
pairs, where the keys are strings and the values have
arbitrary type. 
This key-value abstraction
is used by the POSIX system environment, the Java
Properties API, and the Windows Registry.

%As an example, in the error-fixing
%configuration setting \CodeIn{output\_format = XML} for JMeter in Section~\ref{sec:evolerror},
%\CodeIn{output\_format} is the configuration option name,
%and \CodeIn{XML} is the value. 

\subsection{Overview}

\ourtool is based on two key insights. First,
a program's control flow, rather than data flow,
often propagates the majority of
the effects of a configuration option.
In other words, a configuration option is mainly used
as a ``flag'' that affects the program behavior by
changing the runtime execution path.
% and determines
%a program's execution path.
Second, the control flow differences of two execution
traces can approximate the program behavioral differences
of two versions, and provide evidence
for which parts of the program might be behaving
abnormally and why.
%undesirably.

Based on these two insights, \ourtool uses three
steps to link different behaviors across program
versions to specific configuration options that cause the difference.
Figure~\ref{fig:overview} sketches the high-level workflow of
\ourtool. 
%To recommend configuration options that
%fix the undesired behavior, 
In the first step, \ourtool asks the
user to demonstrate the different behaviors, using the same input and
configuration, on two \ourtool-instrumented program versions
(Section~\ref{sec:profiling}).
Then, \ourtool analyzes the two execution traces produced
by user demonstration, and identifies the control flow differences between
them. In particular, \ourtool identifies program predicates
that behave differently across the two executions
(Section~\ref{sec:comparison}).
After that, \ourtool uses a lightweight dependence
analysis technique, called thin slicing~\cite{Sridharan:2007},
to statically reason about which configuration
options may cause the control flow differences.
Finally, \ourtool reports a ranked list of 
suspicious configuration options to the user (Section~\ref{sec:rootcause}).

\subsection{Instrumentation and Demonstration}
\label{sec:profiling}

\ourtool first instruments both the old and new program versions
to monitor the program  execution at runtime. \ourtool directly
instruments the bytecode. The instrumentation
consists of two parts:

\vspace{-2mm}

\begin{itemize}
\item For each program predicate (i.e., a branch instruction
in bytecode), \ourtool inserts one
probe before and one probe after it
to monitor how frequently the predicate is executed and
how often the predicate evaluates to true. In our
context, a predicate is a Boolean expression in a
conditional or loop statement,
whose evaluation result affects the program
control flow by determining whether to execute the
following statement or not.


\item For each of the other statements, \ourtool inserts
one probe before it to monitor whether the statement
gets executed or not at runtime. The statement execution
information is used to calculate the number of executed
statements controlled by a predicate (Section~\ref{sec:rootcause}).

%\todo{This seems excessively inefficient.  The probe should only be
%  necessary at the start of each basic block, unless you are concerned
%  about exceptions.}

%\todo{Where is this information about non-predicate statements used?  I see
%  that the algorithm uses the true ratio for predicates, but not where it
%  uses the number of executions for arbitrary statements.  Oh, maybe it's
%  used in getExecutedStmtNum, but that isn't clear.}

\end{itemize}


\vspace{-1mm}

After instrumentation, \ourtool asks the user to demonstrate the different
behaviors on the two instrumented program versions, using
the same input and configuration. Demonstration is
one of the simplest ways for an end-user to describe her problem;
and it is easier than writing specifications or scripts of any form.
%a better way is to count the basic block numbers

Executing the instrumented program produces an execution trace,
which consists of a sequence of executed statements as well
as the execution count and evaluation result of each predicate.
The execution trace captured by \ourtool is by no means complete
in recording the full program behavior; it only
captures the control flows a program is taking. As demonstrated
in our experiments, such control flow information serves as a
good approximation to diagnose the undesired program behavior.

%determine which parts of the program behave
%abnormally.
% are
%responsible for the behavioral difference and why.


%predicates and their
%evaluation results. Such captured predicate behaviors are by
%no means complete in recording the full execution trace. However,
%they capture control flows a program is taking. Thus, using the
%recorded predicate execution result, \ourtool could faithfully
%derive the full execution path. 


%  LocalWords:  getExecutedStmtNum


\subsection{Execution Trace Comparison}
\label{sec:comparison}

\begin{figure}[t]
\textbf{Axuiliary methods}:

sameStatement($\mathit{s}$, $\mathit{s'}$): return whether two statements
$\mathit{s}$ and $\mathit{s'}$ are the same statements. \todo{What does
  ``same statement'' mean?  The same source code?  The same subexpressions
  too? (For example, does this return true for if statement only if the
  predicate, then clause, and else clause are all identical?)}

BFS($\mathit{s}$, $\mathit{cfg}$): return an ordered list of reachable successive statements from statement $\mathit{s}$ in $\mathit{cfg}$ by Breath-First Search (BFS).

firstMatch($\mathit{stmtList_1}$, $\mathit{stmtList_2}$): return the first matched statement pair between $\mathit{stmtList_1}$ and $\mathit{stmtList_2}$ (iterating $\mathit{stmtList_1}$ first).
\todo{What is a ``matched statement pair''?}


\textbf{Input}: two methods from two software versions: $\mathit{m_{old}}$ and $m_{new}$,

\quad a maximum lookahead value $\mathit{lh}$ (Our experiment uses $\mathit{lh}=5$.).\\
\textbf{Output}: matched statements between old and new versions.
\vspace{-4mm}%
matchStatements($\mathit{m_{old}}$, $\mathit{m_{new}}$, $\mathit{lh}$)\\
\begin{algorithmic}[1]
\STATE $\mathit{matchedStmts}$ $\leftarrow$ new Map$\langle$Stmt, Stmt$\rangle$
\STATE $\mathit{cfg_{old}}$ $\leftarrow$ constructControlFlowGraph($\mathit{m_{old}}$)
\STATE $\mathit{cfg_{new}}$ $\leftarrow$ constructControlFlowGraph($\mathit{m_{new}}$)
\STATE $\mathit{stack}$ $\leftarrow$ new Stack$\langle$Pair$\langle$Stmt, Stmt$\rangle$$\rangle$
\STATE $\mathit{stack}$.push($\mathit{cfg_{old}}$.$\mathit{entry}$, $\mathit{cfg_{new}}$.$\mathit{entry}$)
\WHILE{$\mathit{stack}$ is not empty}
\STATE $\langle$$\mathit{stmt_{old}}$, $\mathit{stmt_{new}}$$\rangle$ $\leftarrow$ $\mathit{stack}$.pop()
%\IF{$\mathit{stmt_{old}}$ or $\mathit{stmt_{new}}$ has already been matched}
\IF{$\mathit{matchedStmts}$.keys().contains($\mathit{stmt_{old}}$) \\ \quad || $\mathit{matchedStmts}$.values().contains($\mathit{stmt_{new}}$)}
\STATE \textbf{continue}
\ENDIF
\IF{sameStatement($\mathit{stmt_{old}}$, $\mathit{stmt_{new}}$)}
\STATE $\mathit{matchedStmts}$[$\mathit{stmt_{old}}$] $\leftarrow$ $\mathit{stmt_{new}}$
\ELSE
\STATE $\mathit{stmtList_{old}}$ $\leftarrow$ BFS($\mathit{stmt_{old}}$, $\mathit{lh}$)
\STATE $\mathit{stmtList_{new}}$ $\leftarrow$ BFS($\mathit{stmt_{new}}$, $\mathit{lh}$)
\STATE $\langle$$\mathit{stmt_{old}}$, $\mathit{stmt_{new}}$$\rangle$ $\leftarrow$ firstMatch($\mathit{stmtList_{old}}$, $\mathit{stmtList_{new}}$)
\IF{$\langle$$\mathit{stmt_{old}}$, $\mathit{stmt_{new}}$$\rangle$ $\neq$ null}
\STATE $\mathit{stack}$.push($\langle$$\mathit{stmt_{old}}$, $\mathit{stmt_{new}}$$\rangle$)
\ENDIF
\ENDIF
\ENDWHILE
\RETURN $\mathit{matchedStmts}$
\end{algorithmic}
\vspace{-4mm}
\caption{Algorithm for matching statements from two methods.
\label{fig:matching}
\todo{This algorithm is suspicious to me.  For one thing, the size of the
  stack is always $\le 1$ --- why use a stack in that case?  For another
  thing, as soon as \emph{any} statement satisfies sameStatement(), then
  the algorithm terminates.  Is there a bug in the algorithm?}
}
\end{figure}


In this step, \ourtool compares two execution traces
from two program versions, and identifies the
control flow differences. \ourtool focuses on the
behavior of each recorded predicate. It first 
matches each predicate recorded in the old
execution trace to the new version (Section~\ref{sec:match_predicate}),
and then identify all predicates recorded in both execution traces that
behave differently (Section~\ref{sec:identify_diff}).


\subsubsection{Matching Predicates across Versions}
\label{sec:match_predicate}

For each predicate recorded in the old execution trace,
\ourtool matches it in the new program version.
To match a predicate, \ourtool first matches its declaring method,
by using two strategies. \ourtool uses
the first strategy that succeeds.

%, and then
%identifies its runtime behaviors in the undesired execution profile.


\begin{enumerate}
\item \textbf{Identical method name.} Return a method with the identical
fully-qualified name in the new version.
\item \textbf{Similar method content.} Return the method with
the most similar content in the new version. Given
a method in the old program version, \ourtool
uses the algorithm shown in Figure~\ref{fig:matching} (discussed
in detail below) to \textit{every} instruction in it to
\textit{each} method in the new program version, and then
chooses the method with the most matched matched instructions
(in the new program version).
\end{enumerate}

The algorithm in Figure~\ref{fig:matching} is inspired by
a well-established program differencing algorithm, called
JDiff~\cite{}. \todo{describe JDiff and the difference here.}

If there is no matched method in the new program
version, \ourtool concludes that the predicate cannot be
matched. Otherwise, \ourtool uses the following two strategies
to find the corresponding predicate.

\begin{enumerate}
\item If the declaring method is matched by using the first
``Identical method name'' heuristic, \ourtool runs the algorithm
in Figure~\ref{fig:matching} to establish the mapping between
each instruction, and returns the matched instruction of the
predicate.
\item If the declaring method is matched by using the second
``similar method content'' heuristic,  \ourtool
uses the result of the algorithm in Figure~\ref{fig:matching}
by returning the matched instruction of the given predicate.
\end{enumerate}

%For both cases, if the matching algorithm fails to identified

\begin{figure}[t]

\textbf{Auxiliary methods}: 

\quad getPredicates($\mathit{T}$): return all predicates in the execution trace $\mathit{T}$.

\textbf{Input}: two execution traces: $\mathit{T_{old}}$ and $T_{new}$

\quad the instruction map: $\mathit{stmtMap}$ produced by Figure~\ref{fig:matching}

\quad a threshold: $\delta$ (default value: 0.1, used in our experiments)\\
\textbf{Output}: all pairs of behaviroally-deviated predicates between two versions, and their deviation degree\\
\vspace{-4mm}%
identifyDeviatedPredicates($\mathit{T_{old}}$, $\mathit{T_{new}}$, $\mathit{stmtMap}$)\\
\begin{algorithmic}[1]
\STATE $\mathit{predMap}$ $\leftarrow$ new Map$\langle$$\langle$Predicate, Predicate$\rangle$, Float$\rangle$
\FOR{each predicate $\mathit{p_{old}}$ in getPredicates($\mathit{T_{old}}$)}
\STATE $\mathit{p_{new}}$ $\leftarrow$ $\mathit{stmtMap}$[$\mathit{p_{old}}$]
\STATE $\mathit{v}$ $\leftarrow$ $\phi$($\mathit{p_{old}}$, $\mathit{T_{old}}$)
\IF{$\mathit{p_{new}}$ $\neq$ $\mathit{null}$}
\STATE $\mathit{v}$ $\leftarrow$ $|\phi$($\mathit{p_{old}}$, $\mathit{T_{old}}$) - $\phi$($\mathit{p_{new}}$, $\mathit{T_{new}}$)$|$
\ENDIF
\IF{$\mathit{v}$ $\geq$ $\delta$}
\STATE $\mathit{predMap}$.put($\langle$$\mathit{p_{old}}$, $p_{new}$$\rangle$, $\mathit{v}$)
\ENDIF
\ENDFOR
\FOR{each predicate $\mathit{p_{new}}$ in getPredicates($\mathit{T_{new}}$)}
\IF{$\neg$$\mathit{stmtMap}$.values().contain($\mathit{p_{new}}$)}
\STATE $\mathit{v}$ $\leftarrow$ $\phi$($\mathit{p_{new}}$, $\mathit{T_{new}}$)
\IF{$\mathit{v}$ $\geq$ $\delta$}
\STATE $\mathit{predMap}$.put($\langle$$\mathit{null}$, $p_{new}$$\rangle$, $\mathit{v}$)
\ENDIF
\ENDIF
\ENDFOR
\RETURN $\mathit{predMap}$
\end{algorithmic}
\caption{Algorithm for identifying behavioral-deviated
predicates. The $\phi$ function is defined in Section~\ref{sec:identify_diff}.
\label{fig:identify}
}
\end{figure}


\subsubsection{Identify Control Flow Differences}
\label{sec:identify_diff}

With the predicate matching information, \ourtool further
identifies predicates that behave differently
between two versions. 

The algorithm is shown in Figure~\ref{fig:identify}. \ourtool
\todo{describe the algorithm ..}

\ourtool outputs three categories of predicates:
(1) predicates that are only executed in the
old version, (2) predicates that are only executed
in the new version, and (3) predicates that
are executed in both versions but exhibit different
behaviors. For each category, \ourtool sorts the
predicates based on their deviation scores in the
descreasing order.


Differences in predicate behaviors indicate different executed paths
between two versions. Such differences provide evidence of
which part of the program might be behaving unexpectedly and why.

%is any path not taken by the actual
%program execution that starts at a conditional
%branch instruction for which the branch condition
%is affected by one or more configuration options.



\subsection{Configuration Option Recommendation}
\label{sec:rootcause}

%An alternate path is any path not taken by the actual
%program execution that starts at a conditional
%branch instruction for which the branch condition
%is affected by one or more configuration options.


\begin{figure}[t]
\textbf{Auxiliary methods:}

\quad getAffectingOptions($\mathit{p}$, $\mathit{V}$): return all configuration options that affect predicate $\mathit{p}$ in the software version $\mathit{V}$

\quad getExecutedStmtNum($\mathit{p}$, $\mathit{V}$, $\mathit{T}$): return the number of executed statements (determined by predicate $\mathit{p}$) in trace $\mathit{T}$

\textbf{Input}: two software version: $\mathit{V_{old}}$ and $\mathit{V_{new}}$, 

\quad the map of all behaviorally-deviated predicates: $\mathit{predMap}$, produced by Figure~\ref{fig:identify}.

\textbf{Output}: {a ranked list of likely root cause configuration options}
\vspace{-4mm}%
recommendOptions($\mathit{V_{old}}$, $\mathit{V_{new}}$, $\mathit{predMap}$)\\
\begin{algorithmic}[1]
\STATE $\mathit{optionMap}$ $\leftarrow$ new Map$\langle$Option, Float$\rangle$
\FOR{each $\langle$$\mathit{p_{old}}$, $\mathit{p_{new}}$$\rangle$ in $\mathit{predMap}$.keys()}
\STATE $\mathit{v}$ $\leftarrow$ $\mathit{predMap}$[$\langle$$\mathit{p_{old}}$, $\mathit{p_{new}}$$\rangle$]
\IF{$\mathit{p_{old}}$ $\neq$ $\mathit{null}$}
\STATE $\mathit{options_{old}}$ $\leftarrow$ getAffectingOptions($\mathit{p_{old}}$, $\mathit{V_{old}}$)
\STATE $\mathit{v}$ $\leftarrow$ $\mathit{v}$ $\times$ getExecutedStmtNum($\mathit{p_{old}}$, $\mathit{V_{old}}$, $\mathit{T_{old}}$)
\FOR{each Option $\mathit{option}$ in $\mathit{options_{old}}$}
\STATE $\mathit{optionMap}$[$\mathit{option}$] = $\mathit{optionMap}$[$\mathit{option}$] + $\mathit{v}$
\ENDFOR
\ENDIF
\IF{$\mathit{p_{new}}$ $\neq$ $\mathit{null}$}
\STATE $\mathit{options_{new}}$ $\leftarrow$ getAffectingOptions($\mathit{p_{new}}$, $\mathit{V_{new}}$)
\STATE $\mathit{v}$ $\leftarrow$ $\mathit{v}$ $\times$ getExecutedStmtNum($\mathit{p_{new}}$, $\mathit{V_{new}}$, $\mathit{T_{new}}$)
\FOR{each Option $\mathit{option}$ in $\mathit{options_{new}}$}
\STATE $\mathit{optionMap}$[$\mathit{option}$] = $\mathit{optionMap}$[$\mathit{option}$] + $\mathit{v}$
\ENDFOR
\ENDIF
\ENDFOR
\RETURN $\mathit{optionMap}$.sortedKeys()
\vspace{-2mm}
\end{algorithmic}
\caption{Algorithm for recommending configuration options.
\label{fig:recommend}
}
\end{figure}


In this step, \ourtool links the control flow differences
to one or more root cause configuration options.
The key idea is to identify configuration options that
may affect the behaviorally-deviated predicates, and then rank
these options by the degree of how they may cause
the differences.

%\subsubsection{Attributing Trace Differences to Configuration Options}

To identify the configuration options that can affect
a predicate, a straightforward way is to use program slicing~\cite{}
to compute a forward slice from the initialization statement
of a configuration option, and then check whether the predicate is
in the slice. Unfortunately, traditional
full slicing~\cite{} is infeasible due
to its conservatism, such as the need of handling pointers
and the need of following both data and control dependences.
%As been experimentally demonstrated in our
%previous work~\cite{}, traditional full slicing includes
%too much of the program, and can significantly affect an analysis's
%accuracy. 
\todo{an example}


To address this limitation, \ourtool uses thin slicing~\cite{}
to identify configuration options that \textit{directly} affect
a predicate. Different from traditional full slicing~\cite{},
thin slicing \textit{only} follows the data flow dependencies
from the seed (i.e., the initialization statement of a
configuration option), and ignores control flow dependencies
as well as uses of base pointers. This property separates
pointer computations from the flow of configuration option
values and naturally connects a configuration option with its
directly affected statements. \todo{an example}. Section~\ref{sec:alternative}
empirically demonstrates that traditional full slicing includes
too much of the program, and can significantly affect an analysis's
accuracy; while thin slicing is a better
choice.


To distinguish the likelihood of each configuration
option being the root cause, \ourtool associates each
configuration option with a weight, which represents the strength of
the causual relationship between the configuration option
and the execution differences.
A larger weight value indicates that a configuration option
can potentially affect more statements wihthin the control
flow differences as its value propagates in the program, and thus
the configuration option is more likely
to be the root cause.
%which is the number
%of affected statements, as its effects propagates
%in the program.  The weight. 
%If a configuration option
%potentially affects more statements that are decided
%by a behaviorally-different predicate, it is more likely
%to be the root cause.

\todo{explain with the pseduo code}
Figure~\ref{fig:recommend} sketches the configuration option
recommendation algorithm.
For each predicate with different behaviors between two
versions, \ourtool first attributes the control flow difference 
to its affecting configuration options. Then,
\ourtool calculates a cost for each predicate by computing
the number of statements whose execution are determined
by the predicate's evaluation result. Specifically,
\ourtool statically exams the source code to compute
the immediate post-dominant statement of a predicate, and then
analyzes the execution trace to count the number of statements that
are executed between the predicate and its post-dominant
statement. After that, the algorithm multiples the predicate's deviation
degree (computed by \todo{xx}) by the number of executed statements,
and updates the weight of all affecting options.
%attributes the divergence to the affecting options.

\ourtool outputs a ranked list of
configuration options that might be responsible for the
behavior differences. In the output list, if an option only appears
in the old program version, it indicates that option may not
take any effect in the new version (e.g., the option has
been renamed), and the user should switch to a different option.
If an option only appears in the new version or in both
versions, it indicates that the user should change its
existing value.
\todo{rephrase the above to avoid confusion.}

%Finally, the algorithm sums
%first summing
%the total number of instructions executed via this
%predicate. It attributes the divergence
%to root cause configuration options by multiplying
%the cost of the divergence by the weights of the configuration
%options that are relevant to the divergence.


%\ourtool statically analyzes
%the effects of each configuration option when it is
%assigned to a different value.
%Specifically, \ourtool propagates the effect of
%a configuration option to other program statements
%based on control flow dependencies in the program. 



%Two configuration options are considered equal
%root causes even if one has a direct causal
%relationship to a location (e.g., the value
%in memory was read directly from the configuration value)
%and another has a nebulous relationship (e.g., 
%its effects is propagated along a long chain
%of conditional assignments).

%Data flow dependencies are treated to be more likely
%to lead to the root cause than control flow
%dependencies. Control flow dependencies are assumed
%to be more likely to the root cause if they occur
%later in the execution (i.e., closer to the
%deviated execution paths).

%Assign control flow dependence only half the weight
%of the weight introduced by data flow dependencies.
%Further, each nested conditional branches reduces
%the weight by prior branch in the nest one half.

%what about 
%if(a) \{
%
%    if(b) \{
%        //should a and b equally important
%     \}
%    \}



%\todo{mention implementation details about recursive, avoid double count}


%\ourtool also tracks implicit control flow dependencies.

%A predicate's execution depends on the value of
%the configuration option, and the associated weight
%indicates the strength of the dependency.

%\ourtool assumes that control flow dependencies are more likely
%to lead to the root cause if they
%occur closer to the predicate being executed.

%This represents the belief that the execution of
%the basic block is affected by XXX.
%Since these are two independent probabilities:
%potentially changing either of the two options
%might cause the basic block to not have been executed.
%Thus, the weights of configuration options associated
%with a basic block need not sum to one.

%\ourtool determines the root cause of each behaviorally-different
%predicate.  



%Essentially, this step answers the question:
%``how likely is a configuration
%option cause the execution differences?''.

%\ourtool uses two heuristics: xxx.
%These heuristics cause real root causes to rank
%higher than false positives.


%\ourtool next determines why each different path
%gets executed. \ourtool associates each block
%with a set of root causes, More specifically,
%it uses thin slicing to identify a
%set of configuration options for each deviated
%execution path as the root causes.


\subsection{Discussion}
\label{sec:tech_discuss}

We next discuss some design issues in \ourtool.

\vspace{1mm}

\noindent \textbf{\textit{Fixing configuration errors 
vs. Localizing regression bugs.}}
The problem addressed in this paper is significantly different
than the traditional regression bug localization problem~\cite{dd, autoflow}.
A regression bug occurs when developers have made a mistake,
which causes the software to violate its specification after a session of code changes.
By contrast, in our problem, the software behavior on the new version
is still as \textit{designed} by the developers
but \textit{undesired} by the users. 
%\ourtool addresses
%this problem from a specific angel by recommending configuration options
%to fix the undesired behavior.
Further, compared to
%writing correct code,
regression bugs introduced by software developers,
%software configuration lies in the gray zone between
%software developers and software users.
the responsibility for creating
correct configurations lies with both developers and end-users; the developer should create
intuitive configuration logic, build logic that detects
errors, and convey configuration knowledge to users
effectively. This shared responsibility makes diangosing configuration errors
different than localization a regression bug.
%makes recommending configuration options
%different than localizing a regression bug, and causes many existing
%techniques to be inapplicable. We discuss related techniques below
%and in Section~\ref{sec:related}.
%, and describe an empirical comparison
%in Section~\ref{sec:evaluation}.


\vspace{1mm}
\noindent \textbf{\textit{Why not use a dynamic analysis to recommend
configuration options?}}
\ourtool uses thin slicing to statically identify responsible configuration
options for a behaviorally-deviated predicate. Another possible way is to use a pure
dynamic analysis to assess the causality of how a configuration option
may affect the control flow. State-of-the-art
techniques such as Delta Debugging~\cite{dd}, value replacement~\cite{failuredoc},
and dual slicing~\cite{Sumner:2013:CCE}
use a similar idea: they repeatedly replace a variable value with other alternatives,
and then re-execute the program to check whether the outcome is desired.
There are two major challenges that prevent these dynamic analyses
from being used. First, it can be
difficult to find a valid replacement value for a non-Boolean
configuration option, such as a string or regular expression.
Second, automatically checking program outcomes requires
a testing oracle, which is often not available in practice, and end-users
should not be  expected to provide it. To address these challenges,
\ourtool approximates the program behavioral differences by the
control flow differences of two executions, and then statically reasons
about the responsible configuration options.

%However, a major challenge is that it is difficult for
%\todo{illustrate more clearly above}. Investigating how
%to combine static and dynamic analyses 


\vspace{1mm}
\noindent \textbf{\textit{\ourtool's current limitations.}}
There are three major limitations in the our \ourtool technique.
First, \ourtool assumes the different behaviors
of two program versions are not caused by non-determinism.
For non-deterministic behaviors, \ourtool
could potentially leverage a deterministic replay
system~\cite{Huang:2013:CRL, Jin:2012:BRF} to faithfully reproduce the behaviors.
Second, \ourtool only matches one predicate in the old
program version to one predicate in the new program version.
If a predicate evolves into multiple predicates in the new
version, \ourtool may output less useful results. 
%We did not see such cases
%in our experiment, but we speculate that \ourtool
%may not produce useful matching results.
Third, \ourtool focuses on identifying root-cause
configuration options that can change the functional behaviors of
the target program.
% rather than the underlying OS
%or runtime system. 
Configuration options that essentially affect the underlying
OS or runtime system, such as the \CodeIn{-Xmx} option used to
specify JVM's heap size when launching a Java program,
are not supported in \ourtool.

%  LocalWords:  Xmx JVM's
